\documentclass{cslthse-msc}
\usepackage[utf8]{inputenc}
\usepackage{csquotes}
\usepackage[english]{babel}
\usepackage{appendix}
\usepackage{multicol}
\usepackage{titlesec}
\usepackage[nottoc]{tocbibind}
\usepackage{listings}
\usepackage{multicol}
\usepackage{titling}
\usepackage{graphicx}
\usepackage{tabularx}
\usepackage{float}
\usepackage{arydshln}
\usepackage{soul}
\usepackage{enumitem}
\usepackage{natbib}
\usepackage{amsmath}
\usepackage{amsfonts}
\usepackage{amssymb}
\usepackage{amsthm}
%\usepackage{makeidx}
\usepackage[titletoc, header, page]{appendix}
\usepackage{transparent}
\usepackage{pdfpages}
\newcommand{\source}[1]{\caption*{Source: {#1}} }
%\geometry{showframe}
%better like this?
\student{Christoffer MacFie}{dat13cma@student.lu.se}
%\students{Flavius Gruian}{Flavius.Gruian@cs.lth.se}{Camilla Lekebjer}{Camilla.Lekebjer@cs.lth.se}

\thesisnumber{LU-CS-EX: 2019-XX} % Birger Swahn will provide this number to you, once the thesis is ready for publication
% default is Master. Uncomment the following for "kandidatarbete"/Bachelor's thesis
%\thesistype{Bachelor}{Kandidatarbete}

\title{What is required by software platforms in order to give a good developer experience?}
%\onelinetitle
%twolinestitle
\threelinestitle

\subtitle{A Study in Qlik Core's Developer Experience}
\company{Qlik AB}
\supervisors{Martin Höst, \href{mailto:martin.host@cs.lth.se}{\texttt{martin.host@cs.lth.se}}}{Andrée Hansson, \href{mailto:andree.hansson@qlik.com}{\texttt{andree.hansson@qlik.com}}}
%\supervisor{John Deer, \href{mailto:jdeer@company.com}{\texttt{jdeer@company.com}}}
\examiner{Ulf Asklund, \href{mailto:ulf.asklund@cs.lth.se}{\texttt{ulf.asklund@cs.lth.se}}}

\date{\today}
%\date{January 16, 2015}

\acknowledgements{
There are several people I want to thank who made this paper possible.\\\\
I want to thank the interviewees who took time out of their day to be interviewed for this research paper. I also want to thank all the anonymous people who took the surveys conducted in this research paper.\\ \\
I want to thank Andrée Hansson and Johan Bjering, who both served as mentors for me at Qlik. I also want to thank Martin Höst, who was my mentor at Lund University. Finally, I also want to thank Ulf Asklund, who is the exterminator of this paper.
}

\theabstract{
This document describes the Master's Thesis format for the theses carried out at
the Department of Computer Science, Lund University.

Your abstract should capture, in English, the whole thesis with focus on the problem and solution in 150 words. It should be placed on a separate right-hand page, with an additional \textit{1cm} margin on both left and right. Avoid acronyms, footnotes, and references in the abstract if possible.

Leave a \textit{2cm} vertical space after the abstract and provide a few keywords relevant for your report. Use five to six words, of which at most two should be from the title.
}

\keywords{MSc, BSc, template, report, style, structure}

%% Only used to display font sizes
\makeatletter
\newcommand\thefontsize[1]{{#1 \f@size pt\par}}
\makeatother
%%%%%%%%%%

\begin{document}
    \makefrontmatter
    \chapter*{Preface}
    \addcontentsline{toc}{chapter}{Preface}
    To start this research paper off, I will tell you a story.\\\\
    Some years ago I believed I was too dumb to understand thermodynamics. I was about 18 years old and was sitting in my class room trying to solve a problem surrounding thermodynamics. I just could not understand it. My classmates were doing fine, talking and joking with each other while they were solving one problem after another. Meanwhile I just sat there, scratching my head, struggling with the first task.\\
    Then one day, I decided to go and sit by myself instead of in the classroom. I had brought a book, not the one given by the school, but another I had borrowed from the library. I sat in total silence and carefully read the chapter about thermodynamics. I stopped, reflected over the text, and then went back into it. And suddenly, I just understood what I could not understand before.\\
    I think about this from time to time whenever I feel too dumb to learn something. It is not that I was too dumb, it is that the needs I had to do it were not fulfilled. The lesson of the story is that people have different needs in order to have a good experience.\\\\
    This story can be applied to a lot of things in life. Some time ago when trying to use a software platform when programming, I encountered the feeling off not being smart enough. But then I remembered this story, and realized that it is not that I am too dumb, it is that my needs are not fulfilled. And this is what this research paper will be about. What needs do different developers have from software platforms in order to have a good experience using them?
    \chapter{Background and Purpose}
    \section{Goal Of This Research Paper}
    This research paper is done in collaboration with the company Qlik. The goal of it is to define the needs developers have from software platforms in order for them to have a good experience. It also evaluates how well Qlik's software platform Qlik Core satisfy these needs.
    \section{What is User Experience?}
    To start off we have to figure out how we can even measure experience. One field we can look into, that has been researched a lot, is User Experience (UX). UX is the collective term of many disciplines merged into one that evaluates the overall experience delivered to a user of a
    system, product or service. The coining of the term is often attributed to \citet{norman}, who has who has a background in the fields of cognitive science and usability engineering. There isn't one definitive of what UX is. They one who created the term, Norman, defines UX as:
    \begin{quote}
        "All aspects of the end-user’s interaction with the company, its services, and its products. The first requirement for an exemplary user experience is to meet the exact needs of the customer, without fuss or bother. Next comes simplicity and elegance that produce products that are a joy to own, a joy to use. True user experience goes far beyond giving customers what they say they want, or providing checklist features. In order to achieve high-quality user experience in a company’s offerings there must be a seamless merging of the services of multiple disciplines, including engineering, marketing, graphical and industrial design, and interface design."
    \end{quote}
    User Experience has since it's emergence in the 1990s gotten its own ISO-standard: ISO-9241-210. It is defined by \citet{iso9241}, part
    of "Ergonomics of human system interactions", as \begin{quote}
                                                         "A person's perceptions and responses that result from the use or anticipated use of a product, system or service"
    \end{quote}
    World Wide Web Consortium (W3C) is the main international standards organization for the World Wide Web. \citet{w3c} defines it as following:
    \begin{quote}
        "A set of material rendered by a user agent which may be perceived by a user and with which interaction may be possible."
    \end{quote}
    One common denominator is that UX is related to how something is \textit{perceived}. It can therefor be considered a somewhat
    subjective quality of a product, system or service. Because of this it can be hard to measure with exact numbers. \citet{iso9241} has a four-part process how to evaluate a system or service. The process can be seen in figure \ref{fig:iso9241}. It's an iterative process, and depending if the design solution  meets the user requirements or not, an earlier phase in the process needs to be revisited.
    \begin{figure}[H]
        \centering
        \includegraphics[width=0.5\linewidth]{iso9241-210.png}
        \caption{The UX evaluation process as defined by ISO 9241-210}
        \label{fig:iso9241}
    \end{figure}
    \section{What is Developer Experience?}
    Okay, so we now have a better understanding that experience can be measured. What we are interested however is the experience for one certain group: developers. This is where come across the concept of "Developer Experience". Developer Experience, or DX, is similar to the more well known User
    Experience (UX), but with the user being a software developer. Developer Experience has yet to be defined by any standards organization. There is also limited peer-reviewed academic research around the subject. That doesn't however mean that there do not exist articles about it. DX is
    defined by \citet{jarman} as
    \begin{displayquote}
        "The experience developers have when they use your product, be it
        client libraries, SDKs, frameworks, open source code, tools, API,
        technology or service."
    \end{displayquote}
    \citet{dhide} has written an article around how he and a team of developer tries to define DX. As he puts it:
    \begin{quote}
        "Developer Experience (DX) is inspired by the User Experience practice and sees developers as a special case of users. Developer Experience Design is the practice of understanding how developers get their work done, and optimizing that experience."
    \end{quote}
    In this paper we will use the following definition, inspired by UX-defintions:
    \begin{quote}
        "Developer Experience is the perceived feelings and thoughts of a developer, generated by an interaction with a software or service that is meant to be used by a software developer"
    \end{quote}
    \section{How Can One Define "Good" DX?}
    So how do we figure out what exactly is "Good" DX?
    There are many potential factors for defining what constitutes "Good"
    DX. The website \citet{everydeveloper} has developed a \textit{DX Index} from
    1-10, where they consider four factors:
    \begin{enumerate}
        \item Are the libraries available in popular languages?
        \item How prominent, in-depth are the starting guides?
        \item Are the solutions self-serving, without need of demos or 'call us'?
        \item Is the pricing clearly stated?
    \end{enumerate}
    \citet{jarman} has other factors that he uses to evaluate
    if something gives a good DX. He for example puts emphasis on
    communication between the product provider and the developer. The dialog
    between the product provider and the community needs to be authentic,
    open and honest in order the give a good developer experience, according
    to Jarman.
    \\ \\
    \citet{dhide} did a workshop, which generated a list of things that they think should be considered when designing software or services with good DX. Their key aspects are simplicity, usability, innovation and "delightfulness". They put emphasis on "Less is more", stating that "Any element that isn't helping the user achieve their goal is working against them". Some other things they say are important are "Always keep the target users in mind as the product is designed", "Know what type of problem you’re solving" and "Use concepts familiar to the user rather than system-oriented terms".
    \\ \\
    It has been researched by \citet{unhappy} what makes a developer
    happy and unhappy, and they found several indicators, both
    internal- and external factors. They pointed both to internal factors, the developers own well-being. These factors were things like stress, fatigue, low motivation, etc. They also pointed to external factors. These were things like low productivity, delays, broken flow, and more. According to their findings, low productivity is the most common cause of unhappiness among developers. This is something that may be improved by software giving better DX. This research shows that there is a clear need for good DX.
    \section{Who are Qlik and what is Qlik Core?}\label{Qlik}
    This reasearch paper is done in collaboration with the company Qlik.
    Qlik is a software company in Lund, founded in 1993\cite{}. They offer several products, with two most popular ones being QlikView and Qlik Sense. These two programs are used for discovery of insights about data that the user has. The programs takes large quantities of data and link them together into a data-model, which then can be easily interacted with to help the user find insights, patterns and anomalies. This new information can then help companies make decisions for the next step for the company.\\ Let's take an example. A telemarketing company is trying to figure out what employee deserves a Christmas bonus. The company has several thousands of excel sheets of all sold products during the year and who sold what. Going through all these sheets would take many days, and would be very difficult. However, when they load it into Qlik Sense, they can within minutes see the sales per person and date, see figure \ref{fig:SenseExample}. They find that two employee has the same sale amount. However, one of the employees had zero days throughout the year where they had zero sales, whereas the other had 95 days. The company decides to give the Christmas bonus to the employee who had the most consistency in sales.
    \begin{figure}[H]
        \centering
        \includegraphics[width=\linewidth]{salesExample.png}
        \caption{An example of data presented in Qlik Sense}
        \label{fig:SenseExample}
    \end{figure}
    This is all possible because of the Qlik Associative Engine (QAE). The engine links all the data together and makes a data model, which can then be easily displayed in different ways. The target audience for QlikView and Qlik Sense are businesses.\\ \\
    Qlik have now launched a new product called Qlik Core. The target for this product is developers, rather than "ordinary people". The main selling point of this new product is that it gives the powerful QAE to developers to use, and make their own products on top of it.
    Qlik Core is, as described on the official website, "an analytics development platform built around Qlik Associative Engine and Qlik-authored open source libraries"\citep{qlikwebsite}. The Qlik Core platform consists of several components, with it's central part being the QAE. To protect intellectual property (IP), the QAE is close-source.\\ The Qlik Core platform also provides libraries to make it possible to use the QAE, which almost all of them are open-source. The libraries that are included with Qlik Core are "Mira", "Halyard", "Enigma" and a licensing service.\\
    Mira is an open-source JavaScript library which is used to generate insights about the data. Halyard is a JavaScript library which makes loading data into the QAE easier. Enigma is an open-source service makes it possible to communicate with the QAE. This library is offered in both JavaScript and Go. Lastly, the licensing service is a close-source service that authenticates that the user has paid for the use of the platform.\\
    It should be mentioned that QC takes use of a third-party software called Docker. As described by \cite{whatisdocker}, Docker works a bit like virtual machine, but instead of having a whole operating system run, it simulates just a folder-system. They work with something they call "Containers", which makes it possible to package an application with all its dependencies, libraries, etc, into one package. This can then be run anywhere.
    \section{Popularity of Programming Languages}\label{sec:popLang}
    It should said a few things about the state of programming languages in 2019. There are new languages emerging all the time. Some disappear just as quickly as they appeared, while others are here to stay. \citet{github} releases a yearly review of peoples usage of the web-based hosting service. In their review the talk about, amongst other things, what languages are the most popular. In figure \ref{fig:githubLang} we can see the most popular languages from the end of 2014, to the end of 2018. As mentioned, some languages come and go. One example is Ruby, which has dropped from 5th place, to 10th in just three years. The undisputed winner however is JavaScript, which is also what Qlik Core mainly uses. It should also be mentioned that Go, which QC also offers one of their libraries in, is on the rise and saw an increase by 150\% during 2018.\citep{github}
    \begin{figure}[H]
        \centering
        \includegraphics[width=0.5\linewidth]{githubLang.png}
        \caption{The most popular programming languages in GitHub repositories over time.}
        \label{fig:githubLang}
        \source{\cite{github}}
    \end{figure}
    \section{Kinds of APIs}

    Application Program Interfaces (APIs) are, simply put, a software that
    lets one application interact with another application's inner data and
    services. Applications are in need of an interface to interact with it's
    inner parts to create, read, update and read (CRUD) as well as execute
    commands. APIs provide this link between the two pieces of software that let's
    them communicate.Because of APIs broad nature, there are many types of
    APIs.

    \subsection{Internal, Public and Partner APIs}
    APIs have different level of openness, depending on who is going to access them. They are
    usually divided into three groups: Internal APIs, Public APIs and Partner APIs. These
    three groups are explained below.
    \subsubsection{Internal APIs}
    Internal APIs are APIs that har meant to be used in production and
    within an organisation or company. They are often developed to be used
    between different teams in the company to be able to connect software
    components in the application, without having to actually know the code
    of the component. The benefit of this is that the team can open up certain
    needed functionality of the software to other teams while still being in
    control of their own code. This kind of APIs are protected and require
    internal API keys to access to ensure that people outside of the company
    are not able to access them.
    \subsubsection{Public APIs}
    Public APIs is another kind of API. This is a way for the company to
    open up functionality of the software's inner workings to the world so
    that anyone may build new applications that are built upon the original
    software. This kind of interface often only has a small percentage of
    the functionality that the internal API has, since the circuitry of the
    software must be protected for security reasons as well as from business
    intelligence theft. If the internal API was open to the public anyone
    could build their own copy of the program. These kind of APIs either do
    not require any API key to access, or have an API key that is open for
    anyone to acquire (either through payment or for free).
    \subsubsection{Partner APIs}
    Partner APIs are a third interface that can be shared
    business-to-business (B2B), with strategic partners to the company.
    Partner APIs often put some restraints on what can be exposed so that
    the inner workings of the software is still protected, but is able to be
    more open than a public API. These kinds of APIs require an API key that
    is often contracted with terms and condition to protect the company's
    business intelligence.\citet{levin}

    \subsection{Web APIs}

    When using services over the internet, there are many different protocols that can be used to communicate.
    And just like with many other things in computer science, there are many valid approaches whom all have their
    pros and cons. The world wide web (WWW) is largely built upon the application protocol
    hypertext-transfer-protocol (HTTP) which, amongst other things, provides CRUD (Create, Retrieve, Update, Delete)
    methods to be applied on resources. Resources in this context is refers to any \textit{thing}: file, object, document,
    text, etc, that is provided by a web service. There are however many ways of utilizing this protocol to let a
    client access and manipulate server-side data, as well as execute commands. Below there are two methods described.

    \subsubsection{REST}

    Representational State Transfer (REST) is a software architectural style
    that is used in web services which acts as a communication bridge
    between computer systems and the internet. It let's the system interact
    and manipulate the service it's interacting with. REST solves many
    issues that had been present in previous implementations of
    communication between computer systems and the web.
    \\ \\
    One of the key factors in REST is that it's stateless. Statelessness in
    this context means that the two communicating parties does not need to
    know anything about each other or have seen previous messages to
    understand future ones. This feature is possible by limiting it to the
    use of resources instead of commands. REST-APIs can therefor not ask the
    server side to execute a specific custom command, but is limited to
    using CRUD methods.
    \\ \\
    A REST request consists of an HTTP method, a header containing
    information about the request, the path to the resource and lastly and
    optional message body consisting of data.\citet{rest1}\citet{rest2}
    \\ \\
    Statelessness makes it possible to separate the client and the server.
    Code changes to the server will not require changes to the client's
    code, and vice versa, as long as the message format between the two are
    kept the same.
    \\ \\
    Since REST does not use sessions, but simply responds to any incoming
    requests, it's easy to scale up. It simply requires more bandwidth and
    processing power to be able to handle more requests per second.
    \\ \\
    If you want to for example post a message as a user with the \texttt{userID 1}, it
    could look something like this when using REST:

    \begin{lstlisting}
        POST /users/1/messages HTTP/1.1
        Host: example.com
        Content-Type: application/json
        {"msg": "Test123"}
    \end{lstlisting}

    Here, the resource of \texttt{/users/1/messages} is fetched and then the
    new message is created and put there. The server does not work in
    sessions and cannot tell clients that a new message is available. The
    clients has to periodically ask the server if there are any new messages
    to retrieve.
    \\ \\
    REST may be appropriate to use when you mostly want to do CRUD-commands
    or manipulate data.

    \subsubsection{RPC}

    %    https://en.wikipedia.org/wiki/Remote_procedure_call
    %    https://www.smashingmagazine.com/2016/09/understanding-rest-and-rpc-for-http-apis/
    Remote Procedure Call (RPC) is a protocol used to execute commands on remote systems. RPC is, like REST, also
    built on HTTP, but uses mostly just the \texttt{GET} and \texttt{POST} commands. RPC is a request-response protocol and is, unlike
    REST, stateful. Ergo, the protocol works with sessions between a client and a server, and previous messages may
    be needed in order to understand future ones.
    \\ \\
    An upside of RPC is that it let's a client request the server to execute
    a custom command. Making an RPC-call is much like making a normal
    function call, in that you simply provide the name of the method and the
    parameters. A consequence of this is that code changes on server-side,
    such as method-name or parameter input, may require code changes on the
    client side as well.
    \\ \\
    Since RPC has two-way communication, the server can tell the client when
    something has changed, whereas in REST-based communication the client
    has to ping the server to check if there are any changes. An RPC based
    server needs to have a unique session for each client, which can cause
    problems with scalability.
    \\ \\
    If we go back to example used in the previous section: posting a message
    may look something like this:

    \begin{lstlisting}
        POST /SendMessage HTTP/1.1
        Host: example.com
        Content-Type: application/json
        {"userId": 1, "msg": "Test123"}
    \end{lstlisting}

    Here, the server has a custom method called \texttt{SendMessage}. If the
    method call is not made asynchronous, the client is put in wait until
    the server responds with an acknowledgement or the call reaches a
    timeout. Since RPC uses sessions and custom commands, the method can be
    implemented as such that other sessions should be notified that a new
    message has been sent, and the server can send it out to appropriate
    clients.
    \\ \\
    RPC may be appropriate to use when you have functionality that can
    benefit from two-way communication or is mainly command-oriented.

    \subsection{What Type of API is Qlik Core?}

    Qlik Core consists of several components which utilizes different types
    of APIs. Since Qlik Associative Engine is mostly action based, it uses
    JSON-RPC: a remote procedure call protocol in JSON format. Qlik Core
    also provides a discovery service called Mira, which let's the user make
    insights about their data. Mira is a REST-API, since this is a about
    retrieving data and not performing actions.\citet{qlikwebsite}

    \section{API User Personas}
    There are many types of people using platforms, whom all have different
    requirements. One of dividing people is into the two groups
    'Decision makers' and 'Non-Decision Makers'. These both need to be catered to in order
    to have a successful platform: if the decision makers are ignored the
    platform will not be implemented by companies in the first place. If the
    users are ignored, the platform will be quickly dropped since it's usage
    is not good enough.
    \\ \\
    \citet{personas} lists several personas that will affected by how HTTP-based APIs are developed. A persona in this context is a group of people who share certain characteristics. He tries to cover all thinkable personas that would be affected by APIs. He looks both for the perspective of how to satisfy the consumer of APIs, and the creator of APIs. For this research paper, the view is that of what makes consumers of software platforms satisfied, and it is therefore the viewpoint of the consumer that's interesting to us.
    \begin{itemize}[label={}]
        \item \textbf{Established Company Employees} - These people are limited in their freedom of what they can adopt, prohibited by the company's policies, ways of working and legal requirements. They are more likely to use more established programming languages, rather than the latest fads or up-and-coming languages. They are more likely to demand good documentation with examples in their languages, and less likely to spend time trying to understand new ways of working.
        \item \textbf{Startup Company Employees} - This person is in many ways the opposite of the person described above. This person is more likely to use newer programming languages, closely following what's "hot and new". This person also prefers quick solutions, going for existing implementations rather than developing their own. They're pragmatic, looking for fast results and does not have time to spend too much time on "fancy" solutions.
        \item \textbf{Mobile Developers} - This person has concerns that developers for computer-based API consumers does not have to the same extent. Even though it was more relevant a few years ago, when mobile phones were considerably less powerful and PCs, the mobile developer has to take into account CPU and memory usage more than the developers for PCs. They also care about energy usage, since mobiles uses batteries. API creators can help this developer out by specifying how heavy API calls are to make.
        \item \textbf{Poorly Connected Users} - These users has concerns that the others don't. With poor connection, they're taking into account how big the responses from API calls are, how many calls has to be made, etc. This developer needs specifications on these things when developing.
    \end{itemize}


    \section{Standardisation}
    International Organisation for Standardization (ISO) has yet to present a
    standard for developer experience. There are however other standards from ISO
    that are interesting to have a look at. One is \textit{ISO 9126 Software engineering - Product Quality}.
    One part of the ISO-standard concerns quality. This part of the ISO standardizes how to measure the quality of software.
    It has six different characteristics: functionality, reliability, usability, efficiency, maintainability and
    portability. Each of these characteristics have sub-characteristics. The definition of each
    characteristic is listed, and their sub-characteristics, is in table \ref{tabl:standard1} and table \ref{tabl:standard2}


    \begin{table}[H]
        \centering
        \caption{Part 1: ISO-9216}
        \label{tabl:standard1}
        \begin{tabularx}{\columnwidth}{|l|l|X|}
            \hline

            \multicolumn{3}{c}{\textbf{Functionality}} \\ \hline
            F1  &   Suitability &The capability of the software product to provide an appropriate set of
            functions to specified tasks and objectives \\ \hline
            F2  &   Accurateness &  The cap... / / ... to provide the right or agreed results or effects with the needed
            degree of precision \\ \hline
            F3  &   Interoperability& The cap... / / ... to interact with one or more specified systems \\ \hline
            F4  &   Security& The cap... / / ... to protect information and data so that unauthorised persons or systems
            cannot read or modify them and authorised persons or systems are not denied access to them \\ \hline

            \multicolumn{3}{c}{\textbf{Reliability}} \\ \hline
            R1  & Maturity    & The cap... / / ... to avoid failure as a result of faults in the software \\
            \hline
            R2  &   Fault tolerance & The cap... / / ... to maintain a specified level of performance in cases of the
            software faults or of infringement of its specified interface \\ \hline
            R3  &   Recoverability  & The cap... / / ... to re-establish a specified level of performance and recover the
            data directly affected in the case of a failure \\ \hline

            \multicolumn{3}{c}{\textbf{Usability}} \\ \hline
            U1  &   Understandability   &   The cap... / / ... to enable the user to understand whether the
            software is suitable, and how it can be used for particular tasks and conditions of use \\ \hline
            U2&Learnability & The cap... / / ... to enable the user to learn it application \\ \hline
            U3&Operability & The cap... / / ... to operate and control it \\ \hline
            U4&Attractiveness& The cap... / / ... to be attractive to the user [visually] \\ \hline

            \multicolumn{3}{c}{\textbf{Efficiency}} \\ \hline
            E1&Time Behaviour & The cap... / / ... to provide appropriate response and processing times and
            throughput rates when performing its function, under stated conditions \\ \hline
            E2&Resource Utilisation & The cap... / / ... to use appropriate amounts and types of resources when the software
            performs its function under stated conditions \\ \hline
        \end{tabularx}
    \end{table}

    \begin{table}[H]
        \centering
        \caption{Part 2: ISO-9216}
        \label{tabl:standard2}
        \begin{tabularx}{\columnwidth}{|l|l|X|}
            \hline
            \multicolumn{3}{c}{\textbf{Maintainability}} \\ \hline
            M1&Analysability & The cap... / / ... to be diagnosed for deficiencies or causes of failures in the
            software, or for the parts to be modified to be identified \\ \hline
            M2&Changeability & The cap... / / ... to enable a specified modification to be implemented \\ \hline
            M3&Stability & The cap... / / ... to avoid unexpected effects from modifications of the software \\ \hline
            M4&Testability &The cap... / / ... to enable modified software to be validated \\ \hline

            \multicolumn{3}{c}{\textbf{Portability}} \\ \hline
            P1& Adaptability &The cap... / / ... to be adapted for different specified enviroments without applying
            actions or means other than tose provided for this purpose for the software considered \\ \hline
            P2&Installability& The cap... / / ... to be installed in a specified environment \\ \hline
            P3&Co-existence &The cap... / / ... to co-exist with other independent software in a common enviroment sharing
            common resources \\ \hline
            P4&Replaceability &The cap... / / ... to be used in a place of another specified software product for the same
            purpose in the same environment \\ \hline

            \multicolumn{3}{c}{\textbf{All characteristics}} \\ \hline
            AC&Compliance & The cap... / / ... to adhere to standards and conventions relating to the
            characteristic \\ \hline
        \end{tabularx}
    \end{table}

    \chapter{Methodology and Preparations}

    There could have been many approaches to this thesis to try to find what was needed in order to find out
    what people require from software platforms in order to get a good developer experience. The process chosen in this research paper was to make
    a survey followed by interviews to triangulate the results.
    \\ \\
    First of all it was decided what aspects were going to considered. This was followed by an initial test survey to find what
    parts to investigate further. Then the major survey was conducted. After that the results were analyzed to try and find
    patterns and interesting aspects. This lead to a series of questions, which was then used when conducting interviews with people with
    different experience and job titles. The result from the interviews were put together and related to the finding in the survey.
    This finally resulted in a list of recommendation on what is needed by a software platform in order to give a good DX.
    Lastly, this list was used to analyze how well Qlik Core follow these recommendations.\\ \\
    It should be emphasised that these surveys and interviews were of an exploratory nature. Rather than trying to confirm already proven things, new things were searched for.
    \section{Deciding consideration aspects}

    There are a lot of aspects that could have been considered as requirements for
    good DX. It had to be limited down however, and ended up with 14 aspects relating to software, and six aspects relating ot the
    creators of software.
    For the second survey, two more aspects were added, namely aspect 13 and 14 in the list in talbe \ref{tab:aspects}. Aspects relating to
    creators of software were not part of the second survey.
    The reasoning for adding two new aspects and removing aspects relating to creators is
    discussed later in the paper. The fourteen original aspects were decided in a combination
    of reading literature, my own experience of what I would consider when
    picking software platforms, and a brainstorm meeting with more experienced
    people at Qlik, consisting of an architect and developers.
    The list that was ended up with can be seen in table \ref{tab:aspects}.
    \begin{table}[H]
        \centering
        \begin{tabularx}{\columnwidth}{l X}
            \multicolumn{2}{c}{\textbf{List of Aspects}} \\
            \hline
            1    & How often the software is updated    \\
            2    & I can have working code quickly  \\
            3    & The API documentation gives thorough explanations on how it works    \\
            4    & The API has code examples    \\
            5    & The documentation doesn't assume any prior expertise \\
            6    & The documentation has consistent language    \\
            7    & The documentation is easy to navigate    \\
            8    & The official website looks professional  \\
            9    & The pricing of the software  \\
            10    &  The release- and change notes are thorough \\
            11   &  The software has the same features on all different platforms   \\
            12    &  The software is compatible with different platforms    \\
            13  &  The software is offered in more than one programming language    \\
            14*    &  The software is open source    \\
            15*    &  The software uses the programming language I am most comfortable with  \\
            16    &  There exists an active online community around the software    \\
            \hline \hline
            17** & The creator of the software has good communication with it's users \\
            18** & The creator of the software has high transparency with it's issues, ways of working, future plans, etc. \\
            19** & The creator of the software seems professional \\
            20** & The creator of the software has a good reputation online \\
            21** & I have heard of the creator of the software before \\
            22** & I have heard of other software the creator of the software has made
            & \\
            \multicolumn{2}{l}{\textit{*Not part of the first survey, **Not part of the second survey}}
        \end{tabularx}
        \caption{List of aspects considered to give good DX}
        \label{tab:aspects}
    \end{table}

    \citet{jarman} is the source for some of these
    aspects. One of the points he makes is the importance of a great documentation.
    He says the documentation should always be written as if the developer is
    a beginner, which lead to aspect "5 - The documentation doesn't assume any prior expertise" in the list.
    He also says great documentation is consistent, ergo it does not use different
    words to mean the same thing, which lead to aspect "6 - The documentation has consistent language".
    A third aspect he says is needed for great documentation is that is has a
    logical structure, which lead to aspect "7 - The documentation is easy to navigate".
    The last part he considers important for a great documentation is verbosity.
    As he puts it, "You can never say too much". This resulted in "3 - The API documentation gives thorough explanations on how it works".
    Jarman also thinks it's important to have good release notes. He goes on
    to present what release notes should consist of. According to him, not
    only should the release notes consist of the expected, such as what's new,
    updated, deprecated, fixed, etc, but also point out possible risks of the new release,
    such as things that might break with it. Although these sub-features of release notes
    might be interesting to list as their own aspects, the list had to be kept short
    and the aspect that was ended up with was "10 - The release- and change notes are thorough".
    \\ \\
    Jarman also talks about pricing. He puts emphasis on that pricing of the software
    should be easy to find for the developer. During the brainstorming, this
    aspect was also discussed. The aspect was ended up with was the somewhat vague
    "9 - The pricing of the software". This was deliberately chosen
    to be a somewhat open-ended aspect, since there's a lot of things that you can
    consider around the pricing. The information that was searched for was simply if pricing was something
    that was often considered in general. In hindsight, it might have been better to have
    divided this into several aspects, since it's difficult to know how the
    survey taker interpreted the aspect.
    \\ \\
    When this short list existed, I sat down and thought of things that I
    consider myself when picking software. The list was extended further,
    with aspects related to API examples, online community and  platform compatibility.
    \\ \\
    After this, a had brainstorming meeting was had with two senior developers and a
    senior architect to further extend the list of aspects. The meeting attendees can arguably be considered
    to have expertise within the field, having worked for many years within the industry.
    Firstly, all three
    were given five minutes to write down as many aspects as they could think of.
    These aspects were then presented in turn and written on a whiteboard. Any further ideas
    that the three experts got when they saw the others' ideas were also added.
    I then presented the list I had created myself earlier and the aspects not yet listed were added
    as well on the whiteboard. The list was then discussed, as well as the phrasing of each aspect and the importance
    of them.
    \\ \\
    Finally the list contained fourteen aspects, as seen in table \ref{tab:aspects}.
    Being open source is pointed out as important by \citet{jarman}, and was discussed in the meeting.
    It was however not included in the final list. After the first survey was conducted, it was
    also pointed out by survey takers as an aspect that they considered. It was then
    added to the list to be used in the main survey.
    \\ \\
    Aspect number 15 on the list was
    added for the main survey as well. The reasoning here being that the aspect
    "13 - The software is offered in more than one programming language" felt
    like it needed a parallel question: "15 - The software uses the programming language I am most comfortable with",
    to see if the importance of several language was solely based on the fact that
    people wanted their favourite programming language.


    \section{Linking considerations to ISO-9216-1}

    As said before, there is no standard for DX. ISO-9216-1 is however a standard
    to measure the quality of software, so comparing the aspects to this list can
    be interesting. In table \ref{tabl:iso}, the aspects are linked to characteristics they relate to. The aspect ID's can
    be found in table \ref{tab:aspects} and the references to the sub-characteristics can be seen in table \ref{tabl:standard1} and \ref{tabl:standard2}.

    \begin{table}[H]
        \centering
        \caption{Relation between ISO-9216-1 and DX-aspects}
        \label{tabl:iso}
        \begin{tabularx}{\columnwidth}{r|l:l:l:l:l:l}
            \rotatebox{90}{\textbf{Aspect ID}}& \rotatebox{90}{\textbf{Functionality}} & \rotatebox{90}{\textbf{Reliability}} & \rotatebox{90}{\textbf{Usability}} &
            \rotatebox{90}{\textbf{Efficiency}} & \rotatebox{90}{\textbf{Maintainability}} & \rotatebox{90}{\textbf{Portability}} \\ \hline
            1	&		&	R1	&		&		&	M3	&	P4\\ \hline
            2	&	F1, F2	&		&	U1, U2, U3	&		&	M2	&		\\ \hline
            3	&	F1, F2, AC	&	AC	&	U1, U2, U3, AC	&	AC &	M1, AC	&	AC	\\ \hline
            4	&	F1, F2, AC	&	AC	&	U1, U2, U3, AC	&	AC	&	M1, AC	&	AC	\\ \hline
            5	&	AC	&	AC	&	U1, U2, U3, AC	&	AC	&	AC	&	AC	\\ \hline
            6	&	AC	&	AC	&	U1, U2, U3, AC	&	AC	&	AC	&	AC	\\ \hline
            7	&		&		&	U1, U2, U3, AC	&		&		&		\\ \hline
            8	&		&		&	U4	&		&		&		\\ \hline
            9	&		&		&	U1	&		&		&		\\ \hline
            10	&	F1	&	R1	&	U1	&		&		&	P4	\\ \hline
            11	&		&		&		&		&		&	P4	\\\hline
            12	&	F3	&		&		&		&		&	P1, P2	\\ \hline
            13	&		&		&	U2	&		&		&		\\\hline
            14	&	AC	&	AC	&	AC	&	AC	&	M1, M2, M4, AC	&	AC	\\ \hline
            15	&	F1, F2	&		&	U1, U2, U3	&		&	&		\\ \hline
            16	&		&		&	U1, U2	&		&		&		\\\hline
        \end{tabularx}
    \end{table}

    \section{Surveys}

    \subsection{Reasons To Do Surveys}
    Surveys are a popular way to collect quantitative data.
    There are several situations when surveys, or questionnaires as they are also referred to, are a good
    method to use when one wants to collect quantitative data. \citet{denscombe} has described it as a good method when you are
    working data that is of non-sensitive subjects, when the data sources are spread out and when the data pool is big.
    He also states that questionnaires are a good way to collect data of both factual nature, as well as opinion-based.
    Denscombe goes on to say that a questionnaire is likely to contain collection of both of these kinds of data and that it is
    important to make it clear to the person taking the questionnaire whether or not the question is asking for an opinion, or a 'fact'.
    Since this research project concerns itself with a mixture of both factual data, such as people's work experience, job titles and responsibilites, as well as opinion-based data relating to developer experience, questionnaires are a good methodology to use.

    \subsection{Construction of a Questionnaire}
    \citet{denscombe} has a section where he lays out the vital parts of a questionnaire. According to him, a questionnaire should always contain the following parts.

    \begin{itemize}
        \item \textbf{The Sponsor} - Clearly state who this questionnaire is from and for.
        \item \textbf{The Purpose} - Why is this questionnaire being made and for what will the data be used. He warns however that one should not go into too much detail, as to lead the questionnaire taker into answering in a certain way.
        \item \textbf{Confidentiality} - Assert the questionnaire taker that the data collected will not be publically available or be directly linked to the him or her, if him or her so not wishes.
        \item \textbf{Voluntary Responses} - Convey that the questionnaire is completely voluntary to take.
        \item \textbf{Thanks} - Make sure to extend your thanks to the questionnaire taker for voluntarily taking the questionnaire.
    \end{itemize}

    \citet{denscombe} also states that there is no good, defined number of questions that should exist in a questionnaire. However, it should be
    \textit{as brief as possible}. He warns of trying to ask too many questions, for anything that \textit{might} be important. This is not a good approach. A questionnaire constructor should always try to keep the scope as tight as possible. He gives four tips for when trying to do this.
    \begin{itemize}
        \item Only ask questions that are absolutely vital for the research
        \item Proof-read to make sure you do not ask any duplicate questions
        \item Make it as fast and easy as possible to answer the questionnaire
        \item Have a test-round for your questionnaire
    \end{itemize}
    Furthermore, \citet{denscombe} talks about phrasing of questions and knowing your target audience. He gives a comprehensive list of things to think of when doing this. Some key parts are that using words that are suitable for your target audience, avoid leading questions and make sure to include sufficient alternatives. He also puts emphasis on the ordering of questions. He states that "easy" questions, that don't require much consideration from the questionnaire taker (Such as job title) should come early in the questionnaire.

    In his paper, \citet{denscombe} also talks about consistency in how you state your questions.
    There pros and cons of using a variety of style for your questions, according to Denscombe. He suggests that using a variety of styles will stop the questionnaire taker from becoming bored, and also stops them from falling into a pattern, where they simply answer the same way every time without considering the question. The upside of using the same question style is that will make the questionnaire taker used to how the questions should be answered and limits the likelihood for confusion.

    \subsection{Making a Test Questionnaire - Survey 1}
    In the early stages of the research project, it was still not decided what part of DX should be thoroughly researched.
    A test-survey was therefor constructed with what was known to be too broad of a scope. The goal of the survey was not to collect
    useful data per say, but rather to find what sub-field of DX was interesting to pursue. The goal was also to see how well defined the list of aspects was (see table \ref{tab:aspects}), if there were better way of phrasing the questions and if people even knew what developer experience was. I was weary of Denscombes warning that people are hesitant to do more than one survey, but knew that the main survey would be much later. I also made sure, when the survey was sent out, to keep the tone casual to only catch people whom were actually interested in taking this survey, to "save" the more hesitant people for the main survey, hoping that the interested people might take the main survey later as well. A last step to prevent this feeling of "Only having one shot" was to only keep the survey open for 24 hours.

    \subsubsection{Survey 1 Subjects, Questions and Takeaways}
    As stated before, the scope was kept very broad. The survey consisted of three main parts: Background information, how developers find new software and lastly their relationship to DX around software. Both close- and open-ended questions were mixed in this survey. From this, two things were learned:
    \begin{enumerate}
        \item More aspects suggestions of things to consider around DX and how people find new software.
        \item The process needs to be more streamlined in the main survey with close-ended questions to more easily be able to compare the data.
    \end{enumerate}

    Exploring how people find new software would have been interesting, but the directly DX-related questions on what people want from a software was more interesting to research further. The test survey also found that the majority of developers had not heard of the concept of "Developer Experience" before, or could not define it.

    \subsection{Follow-up Interview to Survey 1}

    After the results from the initial survey had been evaluated, two follow-up interviews were
    conducted in order to get some insights if there were any issues with
    the survey. Results from the research paper are not normally presented in this section. However, as the results of the follow-up interview affected how the main survey was constructed, some results will be presented here right away anyway.\\\\
    The overall takeaway from the interviews was that
    mindset, context and definitions was a concern. The two interviewees realized during the interview that they had not given consistent answers, due to having thought of different situations for different survey questions. The survey also used technical jargon, that while it had been explained in the survey, had not been read by the interviewees. This leads to questioning the integrity of the data collected in the first survey, since one cannot know if the survey takers had used their own understanding of the jargon, or read the definition given in the survey. During the two interviews some questions were discussed as well as being too open-ended to be answered with close-ended alternatives. Lastly, the interviewees had an issue with the questions having too little context. As stated by one of the interviewees, "Well, it depends on the context.", was the answer to a lot of the survey questions according to him. He stated that it depends if he's working on a hobby project, or professionally. He also stated that it depends on if he's the only one that will use the platform, or if his co-workers will as well.

    \subsubsection{Takeaways from Follow-up interview}

    Some takeaways are that the context is key to many of the questions that are trying to be asked, and it needs to be clearly stated in the main survey. Another insights is that the survey taker it needs to be confirmed that the survey taker has read the definitions of certain jargon to give comparable data when analyzing the results. Lastly, an example to the context would be good to have, to put all the survey takers in the same mindset.

    \subsection{Making the Main Questionnaire - Survey 2}

    Having made a pilot survey and a follow-up interview, the main survey for the research project was ready to be constructed.
    The questionnaire can be seen in appendix \ref{}.

    \subsubsection{Re-scoping and Changes}
    Overall, the width of the project had to be scoped down, and some
    things had to be dropped from exploration in the project. The
    first survey found that the creator behind a software was less
    considered than had been anticipated. Although it would have been interesting to explore, the decision was to drop these questions for the main survey.
    The questions about how they find new software
    and how long time they spend on this was also be dropped from
    exploration. The re-scoped focus of the main survey instead became what developers are considering when looking at software platforms.
    \\ \\
    It was clear that most people had not heard of, or were not sure about,
    what developer experience was. The main survey therefor included even more comprehensive definition of exactly what DX is.
    The follow-up interview made it clear that people may skip texts about definitions and context, which lead to the change of making these texts more in focus and even having confirmation questions, assuring that they had indeed read the text.
    \\ \\
    With the need for more context, three different situations were defined.
    \begin{itemize}[label={}]
        \item \textbf{Group:} - When working professionally and choosing a software platform for a group of people.
        \item \textbf{Single:} - When professionally choosing a software platform solely for yourself.
        \item \textbf{Hobby:} - When working non-professionally on a hobby project.
    \end{itemize}
    From now on, these three situations will be referred to as 'Group', 'Single' and 'Hobby'.
    \\ \\
    As many of the questions as possible were made close-ended to give easily comparable data. In the first survey there was also the very broad phrase of referring to "Tools and Framework". This term felt too broad, and since Qlik Core is a software platform, in the end people's thought around only this concept was pursued.
    \\ \\
    Furthermore, when extending the questioning around software considerations, it was split it into two parts: How often they consider something, as well as how it affects them emotionally, since this is more central to how DX is measured. Lastly, as mentioned before, two new entries were added to the list of aspects (See table \ref{tab:aspects}).

    \subsubsection{Survey 2 Structure}

    The survey consisted of three parts:
    \begin{itemize}[label={}]
        \item \textbf{Part 1: Background} - A screener around who the survey taker is.
        \item \textbf{Part 2: Software Considerations} - A three-part section around of how often they consider the different aspects when choosing a software platform
        \item \textbf{Part 3: Developer Experience} - A two-part section around how likely the different aspects are to cause a positive/negative feeling
    \end{itemize}
    \\ \\
    This first part of the survey contained definitions and explanations. It also consisted of a background-check of whom the respondent was, asking questions like job title, years of experience and size of the company he/she is working at. It also had a question on whether or not the respondent was in a position to make decisions on what software other people would use. This question was relevant out of two reasons. The first was to be able to have data on whether or not decision makers have other priorities. It also made sure that non-decision makers would not answer the part about the context of 'Group', since they did not have the authority to make those decisions anyway.
    \\ \\
    The second part of the survey focuses on what people consider when choosing a
    software platform. This part was divided into three categories, which
    from now on will be called 'Group', 'Single' and 'Hobby'.
    \\ \\
    The questions were the same in these three prats, the only
    different was the context given. The question was:

    \begin{quote}
        "When $<context>$, which of these traits or aspects do you usually consider when using a software platform?"
    \end{quote}
    The respondent would then rank each aspect (see table \ref{tab:aspects}) on a scale. They had the following alternatives.
    \begin{itemize}[label={-}]
        \item Never consider
        \item Rarely consider
        \item Sometimes consider
        \item Often consider
        \item Always consider
    \end{itemize}
    \\ \\
    The last part of the survey focused on developer experience. It had
    two sub-parts, how likely a factor is to cause them to leave an
    interaction with a software platform with a positive feeling, and how
    likely a factor is to cause them to leave an interaction with a software
    platform with a negative feeling. The question were closely linked to
    the questions asked in the consideration part, but focused on the
    \textit{feeling} rather than if they usually consider the factor when choosing
    a software platform.

    \subsubsection{Data Pool and Implications}
    The survey was sent out internally to Qlik employees. It was also shared with customers of Qlik through their twitter account. This is a quite homogeneous group of people, all working with a certain type of software. While this could be seen as concerning for the data, it was decided that a wider audience than this should not tried to be reached. The reasoning behind this was that there were no reliable or guaranteed way of widening the data pool by any big numbers. However, if the survey was to be shared online and ask 'random' people to answer, one could not know who is answering. By keeping it limited to these two sources, one can have control of who are in the data pool and can take this into account when drawing conclusions.

    \subsubsection{Analyzing Survey 2}
    With all the questions, a part from people's job title, being close-ended it was quite easy to streamline and compare the data collected from the survey. The data was loaded into Qlik Sense, a data visualisation program from Qlik which makes it easy to find patterns and insights of data. In order to compare the questions where people had to choose alternatives on a scale, each answer was given a number, see table \ref{tabl:points}.
    This made it possible for us to get an average score between 0.00 and 1.00 of each aspect. This in turn made it possible to rank each aspect of how important or often considered it was.
    \begin{table}[H]
        \caption{Scaled answers and their given points when analyzing the data}
        \label{tabl:points}
        \centering
        \begin{tabular}{l l|r}
            \multicolumn{2}{l}{\textbf{Answer}} & \textbf{Points} \\ \hline
            Never Consider & 1 - Not very important & 0.00  \\ \hline
            Rarely Consider & 2 & 0.25 \\ \hline
            Sometimes Consider & 3 - Neutral & 0.50 \\ \hline
            Often Consider & 4 & 0.75 \\ \hline
            Always Consider & 5 - Very important & 1.00 \\ \hline
        \end{tabular}
    \end{table}

    \subsubsection{Considerations Questions and DX Questions}\label{consisQuestions}
    In the second survey there were one section with consideration questions, and one section with DX-related questions. These were closely related, but some differ in what they're asking. In table \ref{tab:allQPart1} and \ref{tab:allQPart2} you can see all the questions grouped together. The most noteworthy differences between DX-question and consideration question is the pricing of the software, where the consideration question is very broad whereas the DX-question specifically asks how easy it was to find the price. Another is the broad consideration question of how often the software is updated, whereas the DX-question specifically asks about how quickly the software addresses bugs.

    \begin{table}[H]
        \centering
        \caption{Part 1: The questions asked in the main survey, grouped by aspect category.}
        \label{tab:allQPart1}
        \begin{tabularx}{\columnwidth}{X|l} \hline \hline
        \multicolumn{2}{c}{\textbf{	AMOUNT OF PROGRAMMING LANGUAGE OFFERS	}} \\ \hline
        The software was offered in more than programming language	&	P		\\ \hline
        The software was only offered in one programming language	&	N		\\ \hline
        The software is offered in more than one programming language	&	C		\\ \hline \hline
        \multicolumn{2}{c}{\textbf{	ONLINE COMMUNITY	}} \\ \hline
        The online community was helpful with up-to-date discussion threads	&	P		\\ \hline
        The online community was not helpful and the discussion threads were out-of-date	&	N		\\ \hline
        There exists an active online community around the software	&	C		\\ \hline \hline
        \multicolumn{2}{c}{\textbf{	RELEASE NOTES	}} \\ \hline
        The release notes for what's new/updated/deprecated/etc in an update were well-written	&	P		\\ \hline
        The change- and release logs for what's new/updated/deprecated/etc in an update were poorly written	&	N		\\ \hline
        The release- and change notes are thorough	&	C		\\ \hline \hline
        \multicolumn{2}{c}{\textbf{	DOCUMENTATION LANGUAGE CONSISTENCY	}} \\ \hline
        The documentation had a consistent language, not using different words to mean the same thing	&	P		\\ \hline
        The documentation did not have a consistent language, using different words to mean the same thing	&	N		\\ \hline
        The documentation has consistent language	&	C		\\ \hline \hline
        \multicolumn{2}{c}{\textbf{	DOCUMENTATION NAVIGATION	}} \\ \hline
        The documentation was easy to navigate	&	P		\\ \hline
        The documentation was hard to navigate	&	N		\\ \hline
        The documentation is easy to navigate	&	C		\\ \hline \hline
        \multicolumn{2}{c}{\textbf{	PLATFORM COMPATIBILITY	}} \\ \hline
        The software was compatible on different platforms	&	P		\\ \hline
        The software was compatible on only one platform	&	N		\\ \hline
        The software is compatible with different platforms	&	C		\\ \hline \hline
        \multicolumn{2}{c}{\textbf{	BEING OPEN SOURCE	}} \\ \hline
        The software was open source	&	P		\\ \hline
        The software was not open source	&	N		\\ \hline
        The software is open source	&	C		\\ \hline \hline
        \multicolumn{2}{c}{\textbf{	OFFICIAL WEBSITE LOOK	}} \\ \hline
        The official website looked professional	&	P		\\ \hline
        The official website did not look professional	&	N		\\ \hline
        The official website looks professional	&	C		\\\hline \hline
        \multicolumn{2}{l}{P: Positive DX Impact Question} \\
        \multicolumn{2}{l}{N: Negative DX Impact Question} \\
        \multicolumn{2}{l}{C: Consideration Question}

        \end{tabularx}
    \end{table}

    \begin{table}[H]
        \centering
        \caption{Part 2: The questions asked in the main survey, grouped by aspect category.}
        \label{tab:allQPart2}
        \begin{tabularx}{\columnwidth}{X|l} \hline \hline
        \multicolumn{2}{c}{\textbf{	DOCUMENTATION PRIOR EXPERTISE	}} \\ \hline
        The documentation did not assume that I had any prior expertise with the software, not referencing software-specific things without explaining them	&	P		\\ \hline
        The documentation assumed I had prior expertise with the software, referencing software-specific things without explaining them	&	N		\\ \hline
        The documentation doesn't assume any prior expertise	&	C \\ \hline \hline
        \multicolumn{2}{c}{\textbf{	FEATURES ON ALL PLATFORMS	}} \\ \hline
        The software's features existed and acted the same on different platforms	&	P		\\ \hline
        The software's features did not exist or acted differently on different platforms	&	N		\\ \hline
        The software has the same features on all different platforms	&	C		\\ \hline \hline
        \multicolumn{2}{c}{\textbf{	UPDATES	}} \\ \hline
        The software quickly released updates to address bugs	&	P		\\ \hline
        The software was slow to release updates to address bugs	&	N		\\ \hline
        How often the software is updated	&	C		\\ \hline \hline
        \multicolumn{2}{c}{\textbf{	WORKING CODE QUICKLY	}} \\ \hline
        I could quickly have working code when starting from scratch	&	P		\\ \hline
        I took a long time before I had working code when starting from scratch	&	N		\\ \hline
        I can have working code quickly	&	C		\\ \hline \hline
        \multicolumn{2}{c}{\textbf{	API DOCUMENTATION THOROUGHNESS	}} \\ \hline
        The API was thoroughly explained so that you could understand how it worked	&	P		\\ \hline
        The API was poorly explained so that you could not understand how it worked	&	N		\\ \hline
        The API documentation gives thorough explanations on how it works	&	C		\\ \hline \hline
        \multicolumn{2}{c}{\textbf{	API CODE EXAMPLES	}} \\ \hline
        The code examples for the API were good	&	P		\\ \hline
        The code examples for the APIs were bad	&	N		\\ \hline
        The API has code examples	&	C		\\ \hline \hline
        \multicolumn{2}{c}{\textbf{	FAVOURITE PROGRAMMING LANGUAGE	}} \\ \hline
        The software used a programming language I am skilled in	&	P		\\ \hline
        The software used a programming language I am not very skilled in	&	N		\\ \hline
        The software uses the programming language I am most comfortable with	&	C		\\ \hline \hline
        \multicolumn{2}{c}{\textbf{	SOFTWARE PRICING	}} \\ \hline
        The pricing of the software was easy to find	&	P		\\ \hline
        The pricing of the software was hard to find	&	N		\\ \hline
        The pricing of the software	&	C		\\ \hline \hline
        \multicolumn{2}{l}{P: Positive DX Impact Question} \\
        \multicolumn{2}{l}{N: Negative DX Impact Question} \\
        \multicolumn{2}{l}{C: Consideration Question}

        \end{tabularx}
    \end{table}

    \section{Interviews}

    \subsection{Reasons To Do Interviews}
    There are several reasons why interviews is a suitable method use in this research project.
    The quantitative data collected
    by the two survey gives a good idea of what is needed, does not answer \textit{why} it is needed.
    Interviews will therefor give a depth to the quantitative data that has been collected.
    \citet{denscombe} recommends using interviews as a follow-up to questionnaires. As he puts it,
    questionnaires can generate some interesting results that interviews can pursue
    in greater detail and depth. He also states that interviews should be seen as a good way
    to corroborate data found with other methods. By using interviews, one can triangulate data
    collected by the questionnaires to confirm the facts once more with another approach.
    Furthermore, Dencsombe talks about interviews being well-suited for certain kinds of data.
    He gives three main data types when interviews are a good method,
    two of which are applicable to this research project. The first reason is
    when data is based on emotions. DX is, as described before, based on the \textit{feeling}
    of a good interaction, something that is hard to get a understanding of through
    surveys. The second reason given by Denscombe is when you have access to 'key players'
    with in a field. That is, when you have the possibility to interview people
    that have great insight into a field, interviews are a good way to collect that data.
    With these interviews, one have access to people of different experience, job titles
    and level of decision power. By doing interviews, one can get an understanding
    of why different job titles desire different things, why more experienced people
    want certain aspects and why persons with a lot of power within a company
    have needs that others don't.

    \subsubsection{Interview Instead of Observation Study}
    An observation study was discussed as a method to be used to evaluate DX. It would seem as a quite natural choice, since it's often used when evaluating UX. When evaluating UX, observation studies are used to see how people interact with software. It is mainly used to see if an UI is intuitive. The conductor of the research (the observer) let's a user (the observed) interact with a piece of software. Meanwhile, the observer asks non-intrusive questions, such as "What are you thinking now?" and "What do you expect to happen when you click this?". The observations, such as how long people are stuck or if they made any assumptions that were not expected, are written down. A discussion around this approach for evaluating DX as well was had. However, when diving deeper into what could be observed, it was found that this approach was quite limiting. In figure \ref{fig:obsVSinter} you can see what could have been explored using interviews, compared with observation tests. Since interview had no limitations on what could be explored, it was decided that this approach was superior.
    \begin{figure}[H]
        \centering
        \includegraphics[width=\linewidth]{obsVSinterview.png}
        \caption{What was easy to explore with interviews, compared with observation studies}
        \label{fig:obsVSinter}
    \end{figure}
    \subsection{Interview Decisions}
    There are several things that had to decided when interviews were to be conducted. This included what type of interview style was to be used, what structure the interview should have, what subjects should be explored and what questions should be asked.
    \subsubsection{Interview Style}
    There are many different ways of conducting interviews. \citet{denscombe} groups it into
    three different approaches: one-on-one interviews, group interviews and focus groups.
    These methods all have pros and cons, described by Denscombe.
    \\ \\
    One-on-one interviews are easy to arrange, since
    only two people's (Interviewer and interviewee) schedules have to coincide.
    Compared to group interviews and focus groups, it's also easy to control the interview,
    dive deeper into questions when needed and it's easy to link the results
    to the specific person.
    \\ \\
    Group interviews have advantages as well. It helps to find what a consensus around
    a topic is, where people's opinions can be challenged right away by other group members.
    There are also risks of group interviews, where 'quieter' people may be silenced by
    members of the group who are more dominant. Group interviews are also less suited
    for topics where answers that are more 'accepted' than others.
    \\ \\
    Finally, focus groups have the same issues as group interviews does. It also
    requires a more skilled moderator, since focus groups is more of an ongoing
    discussion of experts within a field, that can easily get out of hand if the
    moderator does not know how to steer the group.  Transcribing focus groups are also
    more challenging since it is natural that people talk over each other. Linking opinions
    to certain people may also be harder when everyone's statements are blurred together.
    \\ \\
    For this research project, one-on-one interviews were chosen.
    The disadvantages of them are small, and one the most challenging things
    about the interviews is finding time to arrange them. It was not feasible
    to try to find enough people for a group interview where everyone's schedules
    could coincide.
    \subsubsection{Interview Structure}
    Dencsombe describes three different types of interviews: structured, semi-structured and
    unstructured interviews. Structured interviews are much like questionnaires, where
    the interviewer holds a tight grip on the interview and does not expect free form answers.
    Structured interviews looks to standardize the results for easy comparison.
    This interview structure is more for 'checking' rather than 'discovery' and is
    therefor more of a quantitative data collection rather than qualitative.
    \\ \\
    Semi-constructed interviews has, just like constructed interviews, a clear set
    list of questions. However, in a semi-structured interview, the interviewee
    is not presented with yes or no questions, or questions with alternatives.
    The questions are instead open-ended. As described by \citet{denscombe}, the interviewee
    is expected to develop their own ideas and speak widely about the given issue or question.
    Here, the emphasis is more on 'discovery' than checking.
    \\ \\
    Unstructured interviews is the third alternative and is the loosest of the three styles.
    With this structure, the interviewer wants to get the interviewees general thoughts
    on a subject or issue and tries to be as unobtrusive as possible. As described by \citet{denscombe},
    the interviewer presents a specific issue or subject and hopes to get a ball rolling.
    Semi-constructed and unstructured exist on a continuum. The more open-ended the questions
    are, the more you move from semi-structured to to unstructured.
    \\ \\
    For this research project, the style of semi-constructed was used.
    The goal is to discover new things, rather than check and confirm, which removes the option
    of a structured interview. However, there is a very clear set of questions
    that need answers.

    \subsection{Interview Subjects}
    There are many things that be explored for these interviews. Due to limited
    time, it had to be narrowed down to a few subjects. The three subjects
    that were chosen are related to release notes, APIs and online community.
    \\ \\
    Release notes is noted as a very important aspect by literature, but
    the survey results show that this is one of the least considered aspects.
    My theory is that this is because people don't look
    at release notes until they are needed, and don't usually consider this
    when choosing platforms. However, a more surprising part is that it does \textit{not}
    affect them negatively DX-wise either if the release notes are poorly written.
    To get a deeper understanding on why this is, this subject was chosen for the interviews.
    \\ \\
    The API documentation and examples are pointed as very important by the
    surveys. In the survey, the vague term of 'Good' and 'Bad' was used, and
    the survey takers applied their own definition of these two terms. What is interesting here is how one can define 'Good' and 'Bad'. One of the most important
    aspects according the survey is also to have working code quickly. Upon exploring
    possible aspects to explore, I found myself asking if working code quickly
    is just a consequence of good API documentation and examples. Maybe it's not
    its own aspect at all? So this question also got merged into this question subject.
    \\ \\
    The last subject that was chosen to be explored is online communities.

    \subsection{Interview material}
    I concluded, after discussion with mentors, that it would be good to have reference points during the interview,
    that there should be some material that the interviewee could use when talking.
    It was discussed using some API documentation and release notes from existing software platforms,
    more precisely Qlik Core. However, Qlik Core is a very advanced platform which takes a long time
    to grasp. Since the interviewees have limited time to put aside for the interviews, it seemed
    like making them read and understand documentation of a whole new software platform was too much to ask.
    Instead an API documentation and release notes for a made up software platform
    called 'MyBakery' was created. Everyone already have a good mental picture of how a bakery works,
    so even if you give them limited amount of documentation they still have a mental picture
    of what this software platform does. To make the request of making them read documentation less boring,
    I tried to make a gamification of the task. The interviewees were given the material, and
    three questions they needed to answer. This forced them to read and understand the material, rather
    than just read through it. The material can be seen in appendix X.

    \subsection{Interview Questions}

    Ultimately, the question that needs to be answered is "Why are the aspects needed or not needed
    to give a good DX and what happens if they (don’t) exist or are poorly/well implemented?"
    The questions for the interviews were constructed to be able to answer this question.
    Since the interview is semi-constructed, all questions will not necessarily be asked.
    Depending on how the interviewee answer, some questions may already have been answered,
    some questions may be uninteresting to dive into, etc.
    \subsubsection{A - APIs}
    For the API documentation, API examples and to have working code quickly, it was already known that was important. It wasn't very surprising that it was.
    For this, the goal was rather trying to understand how one can define "Good" API related things.

    \begin{enumerate}[label={A\Alph*}]
        \item \textbf{General}
        \begin{enumerate}[label={AA\arabic*}]
            \item How important would you say API documentation and examples are?
        \end{enumerate}
        \item \textbf{Api Documentation}
        \begin{enumerate}[label=AB\arabic*]
            \item When you look at API documentation, what are you usually looking for?
            \begin{itemize}[label={-}]
                \item What should the documentation look like?
                \item Is there anything in the material you always look for, or something that is missing?
                \item When you come across an API documentation, are there any red flags you look for?
            \end{itemize}
            \item	How does the API documentation quality affect you in your work?
            \begin{itemize}[label={-}]
                \item Do you abandon a software platform if the documentation is poor?
            \end{itemize}
        \end{enumerate}
        \item \textbf{Api Examples}
        \begin{enumerate}[label={AC\arabic*}]
            \item	What is your goal when looking at API examples?
            \begin{itemize}[label={-}]
                \item For copy-pasting?
                \item To understand underlying structure?
                \item To simply see how it's used?
            \end{itemize}
            \item   What should API examples look like?
            \begin{itemize}[label={-}]
                \item Short examples or long examples?
                \item Should it be runnable or concise?
                \item Within a big context or concise?
            \end{itemize}
            \item	How does the API examples quality affect you in your work?
            \begin{itemize}[label={-}]
                \item Do you abandon a software if the examples are poor?
            \end{itemize}
        \end{enumerate}

        \item\textbf{Working Code Quickly}
        \begin{enumerate}[label={AD\arabic*}]
            \item Is it important for you to have working code quickly?
            \begin{itemize}[label={-}]
                \item Why is it/is it not?
            \end{itemize}
            \item When can you accept to not have working code quickly?
            \item Can you have working code quickly if the API examples and documentation is bad?
        \end{enumerate}
    \end{enumerate}

    \subsubsection{B - Release Notes}
    For release notes, it is known through the survey that they are \textit{not} considered.
    The questions have been divided into three groups: 'When are [release notes] needed",
    "What should [release notes] look like?". A final question to see if the interviewee had any ideas about anomalies in the surveys was also asked.

    \begin{enumerate}[label=B\Alph*]
        \item \textbf{BA - When are they used?}
        \begin{enumerate}[label=BA\arabic*]
            \item How often do you look at release notes?
            \item In what circumstances do you look at release notes?
            \begin{itemize}[label={-}]
                \item Do you look at release notes before deciding to use a platform?
            \end{itemize}
            \item What do you look for when looking at release notes?
        \end{enumerate}
        \item \textbf{What should they look like?}

        \begin{enumerate}[label=BB\arabic*]
            \item Would you say it is important that software platforms have release notes?
            \begin{itemize}[label={-}]
                \item Do you find that information in some other way?
            \end{itemize}
            \item How detailed should they be? / What should they look like?
            \begin{itemize}[label={-}]
                \item How important is it for you that the release notes are thoroughly written?
            \end{itemize}
            \item Is it worth a company's time to make thorough release notes?
            \item How do poor release notes affect you?
        \end{enumerate}
        \item \textbf{Survey Anomaly}
        \begin{enumerate}[label=BC\arabic*]
            \item Release notes is ranked as the least, or amongst the least important aspect for all groups. Why do you think that is?
        \end{enumerate}
    \end{enumerate}


    \subsubsection{C - Online Communities}
    For online community, the goal is to try and understand why it was ranked
    so differently in the two surveys. A goal is also try to understand why, in general,
    an online community matters.

    \begin{enumerate}[label=C\Alph*]
        \item \textbf{General}
        \begin{enumerate}[label={CA\arabic*}]
            \item	When talking about software platforms, what is an online community to you?
            \item	If the documentation was flawless, would you not need an online community?
        \end{enumerate}
        \item \textbf{When are they used?}
        \begin{enumerate}[label={CB\arabic*}]
            \item	How often would you say you take help from online communities?
            \item   Do you check out the online community before choosing a platform?
        \end{enumerate}
        \item \textbf{What should it look like?}

        \begin{enumerate}[label=CC\arabic*]
            \item	What do you want from an online community / what should it look like?
            \begin{itemize}[label={-}]
                \item How important are online communities?
                \item What do communities that you like have in common?
                \item Is it important that a community feels alive?
                \item Is it important the community feels helpful?
                \item Is it important that the tone used in the community is positive?
                \item Is it important that the company behind the software are part of the community?
            \end{itemize}
            \item	If we compare software platforms to something smaller, such as a library. Would you say it is more important or less important to have a software community around it?
            \begin{itemize}[label={-}]
                \item Why is it more/less important?
            \end{itemize}
        \end{enumerate}
        \item \textbf{Survey Anomaly}
        \begin{enumerate}[label={CD\arabic*}]
            \item In the first survey, having an online community was ranked as the second most important aspect. In the second one, it’s ranked in the middle. Why do you think that is?
        \end{enumerate}
    \end{enumerate}

    \section{Evaluating Qlik Core}
    One of the goals for this research paper is to evaluate how good of a software platform Qlik Core is. In section \ref{sec:recommendations}, the recommendations I have for companies who provide a software platform is presented. These recommendations then served as a basis for evaluating Qlik Core. I went through each aspect and determined how well Qlik Core followed the recommendations given. I discussed in what ways QC does, or does not, follow the recommendations and what issues it had. Finally I gave a verdict of either "Pass" or "Failed". In the case of giving the verdict "Failed", I presented what needs to be done in order to get a passing grade.
    \chapter{Results and Discussion}

    \section{Initial Survey}
    Although the first survey only was a pilot version, it can be interesting to evaluate the results of it. There were also things asked in the first survey, that wasn't in the main one. The survey got 38 responses in total.
    Initially it was intended to look at the outcome from four different perspectives, and compare them to the overall group. The four perspectives were:
    \begin{itemize}[label={}]
        \item People with more than 5 years experience in the industry
        \item People who are developers within the industry
        \item People who are architects within the industry
        \item People in companies with more than 200 employees
    \end{itemize}
    The last perspective with the larger companies was
    scrapped since the majority of all answers were from Qlik, and the group
    was therefor a too big majority of the whole group.

    The survey found that the majority of people do not know what DX is. In table \ref{tab:knowdx} we can see how many people felt they could define DX. As we can see, it's quite an unknown term. Almost half had not heard of it, and only 7.9\% felt they fully knew what it was.

    \begin{table}[H]
        \centering
        \caption{Percent of people who knew what Developer Experience was.}
        \label{tab:knowdx}
        \begin{tabularx}{\columnwidth}{X|l}
            \multicolumn{2}{c}{\textbf{Have you heard of the term 'Developer Experience'?}} \\ \hline \hline
            \textbf{Answer} & \textbf{Percent} \\ \hline
            Yes, and I could comfortably give a definition of it & 7.9\%  \\ \hline
            Yes, and I think I could give a definition of it & 	21.1\%  \\ \hline
            Yes, but I could not give a definition of it &	18.4\%  \\ \hline
            No &	47.4\%  \\ \hline
            I don't know & 5.3\% \\ \hline
        \end{tabularx}
    \end{table}

    \subsection{Consideration Factors}
    The scores from consideration question in the first survey in figure \ref{fig:scopresByPoints}. The table is divided into the different perspectives. "Everyone" is simply the average result, "Experienced" is people with more than 5 years of experience in the industry, "Developers" are those who had the job title of developer or software engineer, and "Architects" are those who were some for of software architect.
    As we can see, all but 3 factors scored equal to or above 0.50 points, meaning that they
    were factors that were more often than not considered.
    \begin{figure}[H]
        \centering
        \includegraphics[width=\linewidth]{ScoresByPoints.png}
        \caption{Scores from consideration aspects in survey 1, divided into different perspectives.}
        \label{fig:scopresByPoints}
    \end{figure}
    If we start by looking at the software consideration aspects, we can see that the most important
    factor was 'The API has code examples', followed by the importance of an
    active community around the software, and thirdly that the API
    explanations were thorough. The least two important factors were 'The
    documentation has consistent language' and 'The release- and change
    notes are thorough'.
    This is an interesting find, since literature highlights these to factors as being very important. Many companies also spend a lot of resources to keep documentation up-to-date and release notes thorough. Here we see that most people, more often than not, do \textit{not} consider this factor. A potential explanation to this could be that the question is phrased so that it puts the emphasis on \textit{try} a new software, whereas these two factors may be only relevant in long term use of a software. As we'll see in the main survey however, where the question was phrased as \textit{use a software platform}, release notes were once again listed as one of the least important aspects.
    \\ \\
    For the more experienced users, the result differ some when compared to the 'Everyone' group. They take less consideration to factors surrounding APIs, documentation and time to get started, and care more about pricing,
    cross-platform compatibility and the thoroughness of release notes.
    \\ \\
    With the group consisting of mainly developers, we see almost the
    inversion of some trends in the more-experience-group. The developer
    group cares about documentation and time-to-get-started, and do not care
    about release notes and cross-platform compatibility.
    \\ \\
    If we look at the part about the creator behind the software, we can deduct a few things.
    Overall, the creator behind the software seem to be not completely non-important,
    but also not very important, with all factors scoring close to 0.50. The most
    commonly considered factor was that the creator seemed professional, but
    that only ranked 0.63 on average. The least considered factor was 'I have heard of
    the creator of the software before', which ranked only 0.39.
    \\ \\
    With the more experienced users, they considered the factors even more
    rarely than the 'Everyone' group, apart from transparency, which increased
    slightly. Still, the scores are close to 0.50, and the creator once again
    seem to be only a somewhat important factor.
    \\ \\
    With the group consisting of only developers, it's much the same as
    before, with the one exception that 'I have heard of the creator of the
    software before' had an increase by 10\%. It is however still the least
    considered factor.
    \subsection{Deal Breakers}
    The first survey also had a section of what they considered to be a deal breaker for trying out a new software.
    In table \ref{tab:dealbreaker} we can see the result of this.

    \begin{table}[H]
        \centering
        \caption{Percentage of people who found aspects to be a deal breaker}
        \label{tab:dealbreaker}
        \begin{tabularx}{\columnwidth}{r X}
            & \textbf{Aspect}	\\ \hline
            70\%	&	I have to pay to be able to fully evaluate the software	\\
            65\%	&	It takes a long time to get initially started	\\
            65\%	&	The API is poorly explained	\\
            62\%	&	The API has poor or no code examples	\\
            59\%	&	The online community around the software is dead or has little activity	\\
            27\%	&	The creators behind the software feel like they cater to businesses, not developers	\\
            27\%	&	The online community around the software is unappealing	\\
            16\%	&	The website for the documentation is hard to navigate	\\
            11\%	&	The documentation uses inconsistent language	\\
            11\%	&	The software is not open source	\\
            8\%	&	The creators behind the software are not transparent	\\
            8\%	&	The documentation assumes prior experience with the software	\\
            3\%	&	The release notes are poorly written	\\ \hline
        \end{tabularx}
    \end{table}

    As we can see much of the things that people consider when choosing a software is also a deal breaker if it's poorly implemented. A very clear result from this part is that most people are \textit{not} willing to pay for a software before they can try it out.

    \subsection{Ways of Finding New Software}
    In the test survey there was also a part about how people find new software. The answers can be seen in table \ref{tab:discover}. As we can see, the most common way of finding new software is either through people you know, or online discussions. Since people spend the majority of their working time online and with coworkers, this is quite a natural result. Conferences and online courses are not part of people's daily lives in the same way, and naturally it's not a common source of new discovering new software.

    \begin{table}[H]
        \centering
        \caption{The most common ways of finding new software}
        \label{tab:discover}
        \begin{tabularx}{\columnwidth}{X|l}
            \multicolumn{2}{c}{\textbf{How do you usually discover new tools and frameworks?}} \\ \hline \hline
            \textbf{Answer} & \textbf{Percent} \\ \hline
            Friends or coworkers telling me about it	&	86.8\% \\
            Online communities and forums	&	86.8\% \\
            Reading blog posts or articles	&	55.3\% \\
            Searching for related key words online	&	50.0\% \\
            Social media	&	34.2\% \\
            Conferences	&	26.3\% \\
            Online Courses	&	13.2\% \\ \hline
        \end{tabularx}
    \end{table}

    \section{Survey 2 Results}
    There are many ways you can view the results. In the research it has been looked at from many different angles, and compared with different groups. \\
    In total, the main survey yielded 39 responses. The results were loaded into Qlik Sense
    where one could find different groups and patterns. The results were looked from three angels: The overall average result, result depending
    on your job title and result depending on how much experience you have
    in the industry.
    \\ \\
    The job titles were divided into four groups: Architects, Developer and
    Engineers, Managers and Other. The quantity and percentage of total can
    be seen in the table \ref{tabl:jobs}.

    \begin{table}[H]
        \centering
        \caption{Main survey responses grouped by job titles}
        \label{tabl:jobs}
        \begin{tabularx}{\columnwidth}{r l l}
            \textbf{Grouped Job Titles} &    \textbf{\# of answers} &     \textbf{\% of total} \\\hline
            Architects              & 10           & 25.6\%  \\ \hline
            Developer and Engineers & 21           & 53.8\%  \\\hline
            Managers                & 4            & 10.3\%  \\\hline
            Other                   & 4            & 10.3\%  \\\hline
        \end{tabularx}
    \end{table}

    In table \ref{tab:allPoints} we can see all the aspects, what points each job title gave it and
    their ranking.

    \begin{table}[H]
        \centering
        \caption{List of aspects with their ID, and points and rankings of aspects, grouped by job title}
        \label{tab:allPoints}
        \begin{tabular}{c|r:l|r:l|r:l|r:l|r:l}
            \multicolumn{11}{c}{\textbf{Aspects with their ID}} \\  \hline
            1   & \multicolumn{10}{l}{How often the software is updated}                                        \\
            2   & \multicolumn{10}{l}{I can have working code quickly}                                           \\
            3   & \multicolumn{10}{l}{The API documentation gives thorough explanations on how it works}        \\
            4   & \multicolumn{10}{l}{The API has code examples}                                                \\
            5   & \multicolumn{10}{l}{The documentation doesn't assume any prior expertise}                      \\
            6   & \multicolumn{10}{l}{The documentation has consistent language}                                 \\
            7   & \multicolumn{10}{l}{The documentation is easy to navigate}                                       \\
            8   & \multicolumn{10}{l}{The official website looks professional}\\
            9   & \multicolumn{10}{l}{The pricing of the software}\\
            10  & \multicolumn{10}{l}{The release- and change notes are thorough}\\
            11  & \multicolumn{10}{l}{The software has the same features on all different platforms}\\
            12  & \multicolumn{10}{l}{The software is compatible with different platforms}\\
            13  & \multicolumn{10}{l}{The software is offered in more than one programming language}\\
            14  & \multicolumn{10}{l}{The software is open source}\\
            15  & \multicolumn{10}{l}{The software uses the programming language I am most comfortable with} \\
            16  & \multicolumn{10}{l}{There exists an active online community around the software}\\ \hline
            \multicolumn{11}{c}{} \\
            \multicolumn{11}{c}{\textbf{Scores and rankings}} \\  \hline
            \rotatebox{90}{Aspect ID}                                & \rotatebox{90}{Everyone Rank } & \rotatebox{90}{Avg Everyone } & \rotatebox{90}{Architects Rank } &\rotatebox{90}{Architects Score } & \rotatebox{90}{D\&E* Rank } & \rotatebox{90}{D\&E* Score } &  \rotatebox{90}{Managers Rank } & \rotatebox{90}{Managers Score } & \rotatebox{90}{Other Rank } & \rotatebox{90}{Other Score }\\ \hline
            4                                             &             1 & 0.90         &         1 &   0.92       &            1 & 0.91       &        1 & 0.79     &        1-2 & 0.94  \\ \hline
            3     &             2 & 0.82         &         3 &   0.82       &            2 & 0.87       &        3 & 0.73     &          4 & 0.74  \\ \hline
            2                                       &             3 & 0.80         &         2 &   0.85       &            4 & 0.80       &      4-5 & 0.71     &        1-2 & 0.94  \\ \hline
            9                                           &             4 & 0.80         &         5 &   0.76       &            3 & 0.84       &        2 & 0.75     &          3 & 0.85  \\ \hline
            15 &             5 & 0.68         &         6 &   0.69       &            5 & 0.73       &      6-7 & 0.63     &          5 & 0.69  \\ \hline
            14                                           &             6 & 0.68         &         4 &   0.77       &            6 & 0.69       &        8 & 0.60     &          8 & 0.58  \\ \hline
            16           &             7 & 0.65         &         8 &   0.63       &            7 & 0.67       &     9-11 & 0.54     &          6 & 0.65  \\ \hline
            7                                 &             8 & 0.60         &         7 &   0.66       &            8 & 0.61       &       12 & 0.52     &         10 & 0.52  \\ \hline
            8                               &             9 & 0.60         &         9 &   0.63       &           10 & 0.56       &      4-5 & 0.71     &         12 & 0.44  \\ \hline
            12                   &            10 & 0.58         &        10 &   0.58       &           11 & 0.52       &     9-11 & 0.54     &          7 & 0.59  \\ \hline
            1                                    &            11 & 0.58         &        11 &   0.52       &            9 & 0.57       &      6-7 & 0.63     &         11 & 0.51  \\ \hline
            5                  &            12 & 0.47         &        13 &   0.47       &           13 & 0.44       &     9-11 & 0.54     &          9 & 0.52  \\ \hline
            11         &            13 & 0.46         &        14 &   0.41       &           12 & 0.45       &    13-15 & 0.44     &         13 & 0.42  \\ \hline
            6                             &            14 & 0.41         &        12 &   0.49       &           14 & 0.34       &       16 & 0.42     &         14 & 0.39  \\ \hline
            13         &            15 & 0.36         &        15 &   0.37       &           15 & 0.31       &    13-15 & 0.44     &         16 & 0.23  \\ \hline
            10                            &            16 & 0.33         &        16 &   0.26       &           16 & 0.29       &    13-15 & 0.44     &         15 & 0.35  \\ \hline \hline
            \multicolumn{11}{l}{*D\&E: Developers and Engineers}
        \end{tabular}
    \end{table}


    \subsection{Overall Result}
    Overall, there were quite a lot of interesting findings when analyzing the results from the main survey.
    In general, people are more likely to consider things when they are choosing for
    a group rather than for themselves. For the three categories, Single is the
    closest to the average result, with Group and Hobby existing as
    opposites on either side of Single. In all but one case, if it's often
    considered when choosing for a group, it's \textit{not} often considered when choosing
    for a hobby project, and vice versa.
    \\ \\
    For the part of the survey regarding DX and people's feelings around software platforms, it can be said that there is a direct correlation between if an aspect in it's good form has a positive impact, that same aspect in it's
    bad form will have about the same level of negative impact.
    However, in all but once case, the positive effect is greater than the negative
    effect, if only slightly. In general it is closely related to how often
    something is considered. That is to say, if something will cause someone to have a non-neutral feeling regarding an aspect, they will also consider it when choosing a platform. Ergo, there is no uncoupling between what causes them to have a good DX, and what they consider when choosing a platform. There are some
    outliers to this, but that is the general case.
    \\ \\
    The top three most considered aspects for each context can be seen in
    the table \ref{tabl:top3}.

    \begin{table}[H]
        \centering
        \caption{Top 3 considered aspects by each context}
        \label{tabl:top3}
        \begin{tabular}{r l}
            \hline\hline
            \multicolumn{2}{c}{\textbf{Average}} \\ \hline
            1 &The API has code examples \\
            2 & The API documentation gives thorough explanations on how it works \\
            3 &I can have working code quickly \\
            \hline\hline
            \multicolumn{2}{c}{\textbf{Group}} \\ \hline
            1 & The API has code examples                                        \\
            2 & The API documentation gives thorough explanations on how it works \\
            3 & The software is compatible with different platforms              \\
            \hline\hline
            \multicolumn{2}{c}{\textbf{Single}} \\ \hline
            1 &  The API has code examples                                         \\
            2 &  The API documentation gives thorough explanations on how it works  \\
            3 &  I can have working code quickly                                   \\
            \hline\hline
            \multicolumn{2}{c}{\textbf{Hobby}} \\ \hline
            1 & The pricing of the software    \\
            2 & The API has code examples       \\
            3 & I can have working code quickly\\ \hline
        \end{tabular}
    \end{table}
    The overall result concluded that \textit{the} most considered aspect overall
    is 'The API has code examples'. For hobby, it's the second most
    important aspect, losing the first place by only 0.03 points.
    Further, good code
    examples had the biggest positive impact on developer experience, and
    bad code examples had the biggest negative impact on DX. The negative
    impact if the examples are bad are not as extreme as the positive impact
    is if they are good however. In conclusion, API examples are a key
    factor for software platforms' quality.
    \\ \\
    The thoroughness of the documentation is also one of the most important
    aspects, coming in as the second most important aspect for all
    categories except for Hobby. The positive DX-impact is as big as the
    negative one, and it's also reflected in that it's consideration points
    are as high as the DX-impact points.
    \\ \\
    Having working code quickly is also in the top three for all but the
    group category, where it's in fourth place. The positive DX-impact if
    you can have working code quickly is slightly higher than the negative
    DX-impact if it takes a long time before you have working code. The
    positive impact is also stronger than if you compare it to how often
    it's considered.
    \\ \\
    Having the software be compatible with different platforms is a big
    divider. It's the third most important aspect for Group, but one the
    least important aspects for Hobby, and somewhere in the middle for
    Single. The positive impact is lower than how often it's considered, and
    the negative DX-impact is even less. If you exclude the Group-category,
    it places itself on average as the 11\textsuperscript{th} most important
    aspect, out of 16.
    \\ \\
    The pricing of the software is important to everyone, placing itself on
    average as the 4\textsuperscript{th} most important aspect overall, but it is \textit{the}
    most important aspect for Hobby. For group and single, it's both the
    5\textsuperscript{th}  and 4\textsuperscript{th}  most important aspect respectively. In table \ref{tab:allGroups} we can see the full score and rankings for the different contexts.

    \begin{table}[H]
        \centering
        \caption{Scoring for each aspect by each context}
        \label{tab:allGroups}
        \begin{tabularx}{\columnwidth}{l|X:c:c:c:}
            \textbf{\rotatebox{90}{Aspect ID}} & \textbf{Aspect}	&	\textbf{\rotatebox{90}{Group Score}}	&	\textbf{\rotatebox{90}{Single Score}}	&	\textbf{\rotatebox{90}{Hobby Score}}	\\ \hline
            1	&	How often the software is updated	&	0.69	&	0.59	&	0.46	\\ \hline
            2	&	I can have working code quickly	&	0.75	&	0.80	&	0.86	\\ \hline
            3	&	The API documentation gives thorough explanations on how it works	&	0.85	&	0.82	&	0.79	\\ \hline
            4	&	The API has code examples	&	0.93	&	0.88	&	0.88	\\ \hline
            5	&	The documentation doesn't assume any prior expertise	&	0.41	&	0.47	&	0.53	\\ \hline
            6	&	The documentation has consistent language	&	0.42	&	0.42	&	0.39	\\ \hline
            7	&	The documentation is easy to navigate	&	0.61	&	0.60	&	0.60	\\ \hline
            8	&	The official website looks professional	&	0.67	&	0.60	&	0.53	\\ \hline
            9	&	The pricing of the software	&	0.74	&	0.76	&	0.91	\\ \hline
            10	&	The release- and change notes are thorough	&	0.40	&	0.31	&	0.27	\\ \hline
            11	&	The software has the same features on all different platforms	&	0.63	&	0.47	&	0.28	\\ \hline
            12	&	The software is compatible with different platforms	&	0.81	&	0.58	&	0.37	\\ \hline
            13	&	The software is offered in more than one programming language	&	0.48	&	0.35	&	0.25	\\ \hline
            14	&	The software is open source	&	0.64	&	0.66	&	0.74	\\ \hline
            15	&	The software uses the programming language I am most comfortable with	&	0.56	&	0.71	&	0.78	\\ \hline
            16	&	There exists an active online community around the software	&	0.69	&	0.62	&	0.65	\\ \hline


        \end{tabularx}
    \end{table}

    In figure \ref{fig:dxandcosnid} you can see how the consideration points compares to how much of a positive and negative DX impact it has. In table \ref{tab:dx} you can see the exact numbers for the DX impacts, and the average scores. As we can see, the difference in positive- and negative DX impact is quite small.\\ \\

    \begin{table}[H]
        \centering
        \caption{The average scoring, its positive DX impact and its negative DX impact and their difference, by each aspect category}
        \label{tab:dx}
        \begin{tabularx}{\columnwidth}{l|X r r r r}
            \textbf{\rotatebox{90}{Aspect ID}} & \textbf{Aspect Category}	&	\textbf{\rotatebox{90}{Average Score}}	&	\textbf{\rotatebox{90}{Negative DX}}	&	\textbf{\rotatebox{90}{Positive DX}}	&	\textbf{\rotatebox{90}{DX Diff}}	\\ \hline
            4   & Api Code Examples	&	0.89	&	0.85	&	0.89	&	\pm 0.04	\\ \hline
            3  & Working Code Quickly	&	0.85	&	0.83	&	0.82	&	\pm 0.01	\\ \hline
            2  & Documentation Thoroughness	&	0.83	&	0.80	&	0.81	&	\pm 0.01	\\ \hline
            9  & Pricing	&	0.76	&	0.74	&	0.81	&	\pm 0.07	\\ \hline
            15  & Documentation Navigation	&	0.72	&	0.64	&	0.70	&	\pm 0.06	\\ \hline
            14 & Updates	&	0.72	&	0.63	&	0.69	&	\pm 0.04	\\ \hline
            16  & Online Community	&	0.71	&	0.62	&	0.65	&	\pm 0.03	\\ \hline
            7  & Favourite Programming Language	&	0.67	&	0.62	&	0.60	&	\pm 0.02	\\ \hline
            8  & Being Open Source	&	0.66	&	0.59	&	0.59	&	0.00	\\ \hline
            12 & Official Website Look	&	0.63	&	0.56	&	0.56	&	0.00	\\ \hline
            1  & Documentation Language Consistency	&	0.63	&	0.53	&	0.55	&	\pm 0.02	\\ \hline
            5  & Features On All Platforms	&	0.63	&	0.53	&	0.48	&	\pm 0.05	\\ \hline
            11 & Documentation Prior Expertise	&	0.57	&	0.47	&	0.43	&	\pm 0.03	\\ \hline
            6   & Platform Compatibility	&	0.47	&	0.42	&	0.41	&	\pm 0.01	\\ \hline
            13 & Release Notes	&	0.44	&	0.40	&	0.34	&	\pm 0.06	\\ \hline
            10 & Number of Programming Languages	&	0.39	&	0.34	&	0.32	&	\pm 0.02	\\ \hline
        \end{tabularx}
    \end{table}


    \begin{figure}[H]
        \centering
        \begin{tabular}{r l}
            \hline
            1    & How often the software is updated    \\
            2    & I can have working code quickly  \\
            3    & The API documentation gives thorough explanations on how it works    \\
            4    & The API has code examples    \\
            5    & The documentation doesn't assume any prior expertise \\
            6    & The documentation has consistent language    \\
            7    & The documentation is easy to navigate    \\
            8    & The official website looks professional  \\
            9    & The pricing of the software  \\
            10    &  The release- and change notes are thorough \\
            11   &  The software has the same features on all different platforms   \\
            12    &  The software is compatible with different platforms    \\
            13  &  The software is offered in more than one programming language    \\
            14    &  The software is open source    \\
            15    &  The software uses the programming language I am most comfortable with  \\
            16    &  There exists an active online community around the software    \\ \hline
        \end{tabular}
        \hspace*{-0.01\textwidth}\includegraphics[width=1\textwidth]{dxandconsid.png}
        \caption{The scoring of the consideration points, with how much of a positive- and negative DX impact it has.}
        \label{fig:dxandcosnid}
    \end{figure}

    There are several angles you can view the results at. In the following sections, a few are discussed.
    \subsection{Developer and Engineers}
    Developer and Engineers are the majority of people who are going to use a software platform, since they make out the majority of a development team. In the survey, they also made out the majority of the respondents. In table \ref{tab:devs} we can see how they ranked the considerations. According to this, they are people who want thorough documentation and API examples. They want to have working code quickly, and they care about the pricing of the software. They also more often than not care if a project is open-source, and also want to use the programming language they are most comfortable with. Even though they want thorough documentation, less often than not they don't care if the documentation uses inconsistent language or assumes prior expertise. They do not care if release notes are thorough. Developers and engineers have an average score of 0.60, with the difference between the most considered thing, and the least considered thing, being 0.62.

    \begin{table}[H]
        \centering
        \caption{The ranking and scores of developers and engineers}
        \label{tab:devs}
        \begin{tabularx}{\columnwidth}{ l X r}

        {\textbf{R}*} &   \textbf{Aspect}   & {\textbf{S}**} \\ \hline
        1 &      The API has code examples                                             &  0.91             \\ \hline
        2 &      The API documentation gives thorough explanations on how it works     &  0.87             \\ \hline
        3 &      The pricing of the software                                           &  0.84             \\ \hline
        4 &      I can have working code quickly                                       &  0.80             \\ \hline
        5 &      The software uses the programming language I am most comfortable with &  0.73             \\ \hline
        6 &      The software is open source                                           &  0.69             \\ \hline
        7 &      There exists an active online community around the software           &  0.67             \\ \hline
        8 &      The documentation is easy to navigate                                 &  0.61             \\ \hline
        9 &      How often the software is updated                                     &  0.57             \\ \hline
        10 &      The official website looks professional                               &  0.56             \\ \hline
        11 &      The software is compatible with different platforms                   &  0.52             \\ \hline
        12 &      The software has the same features on all different platforms         &  0.45             \\ \hline
        13 &      The documentation doesn't assume any prior expertise                  &  0.44             \\ \hline
        14 &      The documentation has consistent language                             &  0.34             \\ \hline
        15 &      The software is offered in more than one programming language         &  0.31             \\ \hline
        16 &      The release- and change notes are thorough                            &  0.29             \\ \hline \hline
        \multicolumn{3}{l}{*R: Rank, **S: Score}
        \end{tabularx}
    \end{table}

    \subsection{Architects}
    Architects are a key part of a development team. They are responsible for how the software will be structured and therefor also often have a say in, if the software will be using a software platform, what platform will be used. In table \ref{tab:arch} we can see how the architects ranked the aspects. According to this result, architects are (just like developers) people who want code examples and thorough documentation. They find it important that software platforms are open-source, and want to have working code quickly. They also, just like developers, do not care if the release notes are thorough. The average score for architects is 0.61, with the a difference between the most considered thing, and the least considered thing, of 0.66.

    \begin{table}[H]
        \centering
        \caption{The ranking and scores of architects}
        \label{tab:arch}
        \begin{tabularx}{\columnwidth}{l X r}

            \textbf{R}* &  \textbf{Aspect}        & \textbf{S}** \\ \hline
            1 &    The API has code examples                                             &         0.92             \\ \hline
            2 &    I can have working code quickly                                       &         0.85             \\ \hline
            3 &    The API documentation gives thorough explanations on how it works     &         0.82             \\ \hline
            4 &    The software is open source                                           &         0.77             \\ \hline
            5 &    The pricing of the software                                           &         0.76             \\ \hline
            6 &    The software uses the programming language I am most comfortable with &         0.69             \\ \hline
            7 &    The documentation is easy to navigate                                 &         0.66             \\ \hline
            8 &    There exists an active online community around the software           &         0.63             \\ \hline
            9 &    The official website looks professional                               &         0.63             \\ \hline
            10 &    The software is compatible with different platforms                   &         0.58             \\ \hline
            11 &    How often the software is updated                                     &         0.52             \\ \hline
            12 &    The documentation has consistent language                             &         0.49             \\ \hline
            13 &    The documentation doesn't assume any prior expertise                  &         0.47             \\ \hline
            14 &    The software has the same features on all different platforms         &         0.41             \\ \hline
            15 &    The software is offered in more than one programming language         &         0.37             \\ \hline
            16 &    The release- and change notes are thorough                            &         0.26             \\ \hline \hline
            \multicolumn{3}{l}{*R: Rank, **S: Score}
        \end{tabularx}
    \end{table}

    \subsection{Managers}
    Managers have responsibilities that differ quite much from developers, engineers and architects. They often have to think of legal aspects and are in contact with other departments, such as sales. They will often also not have the same level of competence when it comes to software development as architects, developers and engineers have. They are also not the key demographic for this research, since we are talking about \textit{developer} experience. They are however a key part for the selection of software platforms, since they will have the most authority and also are the person in control (or is the contact to the person in control) of the money. In table \ref{tab:managers} we can see how the managers ranked the aspects. There we can see that even managers care about API examples and that they can have working code quickly. The pricing of the software is, maybe not surprisingly, quite important to them. Managers have an average score of 0.59, with the biggest difference between the most considered thing, and the least considered thing, being just 0.37.

    \begin{table}[H]
        \centering
        \caption{The ranking and scores of managers}
        \label{tab:arch}
        \begin{tabularx}{\columnwidth}{l X r}

            \textbf{R}* &  \textbf{Aspect}        & \textbf{S}** \\ \hline
            1 &   The API has code examples                                             &      0.79  \\ \hline
            2 &   I can have working code quickly                                       &      0.75  \\ \hline
            3 &   The pricing of the software                                           &      0.73  \\ \hline
            4-5 &   The API documentation gives thorough explanations on how it works     &      0.71  \\ \hline
            4-5 &   There exists an active online community around the software           &      0.71  \\ \hline
            6-7 &   The software uses the programming language I am most comfortable with &      0.63  \\ \hline
            6-7 &   How often the software is updated                                     &      0.63  \\ \hline
            8 &   The software is open source                                           &      0.60  \\ \hline
            9-11 &   The documentation is easy to navigate                                 &      0.54  \\ \hline
            9-11 &   The software is compatible with different platforms                   &      0.54  \\ \hline
            9-11 &   The software has the same features on all different platforms         &      0.54  \\ \hline
            12 &   The official website looks professional                               &      0.52  \\ \hline
            13-15 &   The documentation has consistent language                             &      0.44  \\ \hline
            13-15 &   The software is offered in more than one programming language         &      0.44  \\ \hline
            13-15 &   The release- and change notes are thorough                            &      0.44  \\ \hline
            16 &   The documentation doesn't assume any prior expertise                  &      0.42  \\ \hline \hline
            \multicolumn{3}{l}{*R: Rank, **S: Score}
        \end{tabularx}
    \end{table}

    \subsection{Developer and Engineers compared with Architects}
    If we compare architects with developer and engineers, they
    are quite similar. On average, the score differences is only 0.06. The
    biggest difference for consideration questions is 'The documentation has
    consistent language', where Architects average out at 0.49/1.00 in
    points, making it 'Sometimes considered', whereas Developer and
    Engineers only score 0.34/1.00, placing it between 'Sometimes consider'
    and 'Rarely consider'. Their ranking does not differ much, with the
    biggest ranking difference being two spots.
    It can also be said that architects on average have a score of 0.61, and developers 0.60. So by a fine margin, architects consider more things than developers and engineers. Architects are also more extreme, where the most considered thing, and least considered thing have difference of 0.66 points, whereas developers and engineers have 0.62. They are however, like stated before, quite close and similiar.
    \\ \\
    In table \ref{tab:arch-devs} we can see the different aspects, sorted by average most
    important, for Architects and Developers and Engineers.

    \begin{table}[H]
        \centering
        \caption{The ranking and scores of architects, compared with developers and engineers}
        \label{tab:arch-devs}
        \begin{tabularx}{\columnwidth}{X|c:c|c:c|c}

            \textbf{Aspect}                                                    & \rotatebox{90}{\textbf{Architects Rank}}    &\rotatebox{90}{\textbf{Architects Score}} & \rotatebox{90}{\textbf{D\&E* Rank}}  & \rotatebox{90}{\textbf{D\&E* Score}}& \rotatebox{90}{\textbf{Diff}}\\ \hline
            The API has code examples                                             &         1 & 0.92             &               1 & 0.91             & 0.01  \\ \hline
            The API documentation gives thorough explanations on how it works     &         3 & 0.82             &               2 & 0.87             & 0.05  \\ \hline
            The pricing of the software                                           &         5 & 0.76             &               3 & 0.84             & 0.08  \\ \hline
            I can have working code quickly                                       &         2 & 0.85             &               4 & 0.80             & 0.05  \\ \hline
            The software is open source                                           &         4 & 0.77             &               6 & 0.69             & 0.08  \\ \hline
            The software uses the programming language I am most comfortable with &         6 & 0.69             &               5 & 0.73             & 0.05  \\ \hline
            There exists an active online community around the software           &         8 & 0.63             &               7 & 0.67             & 0.03  \\ \hline
            The documentation is easy to navigate                                 &         7 & 0.66             &               8 & 0.61             & 0.05  \\ \hline
            The official website looks professional                               &         9 & 0.63             &              10 & 0.56             & 0.07  \\ \hline
            The software is compatible with different platforms                   &        10 & 0.58             &              11 & 0.52             & 0.06  \\ \hline
            How often the software is updated                                     &        11 & 0.52             &               9 & 0.57             & 0.05  \\ \hline
            The software has the same features on all different platforms         &        14 & 0.41             &              12 & 0.45             & 0.04  \\ \hline
            The documentation doesn't assume any prior expertise                  &        13 & 0.47             &              13 & 0.44             & 0.03  \\ \hline
            The documentation has consistent language                             &        12 & 0.49             &              14 & 0.34             & 0.15  \\ \hline
            The software is offered in more than one programming language         &        15 & 0.37             &              15 & 0.31             & 0.06  \\ \hline
            The release- and change notes are thorough                            &        16 & 0.26             &              16 & 0.29             & 0.04  \\ \hline \hline
            \multicolumn{6}{l}{*D\&E: Developers and Engineers}
        \end{tabularx}
    \end{table}

    \subsection{Architects compared with managers}
    If we compare architects and managers, the differences are bit more
    extreme. On average, they differ by 0.09 points. The biggest divider for
    them is 'The release- and change notes are thorough', where they differ
    by 0.18 points. The rank difference is only one spot though. Their
    biggest dividers in rank differ by five spots. They are 'The official
    website looks professional' and 'The documentation is easy to navigate'
    where the first one is more important to Architects, and the second one
    is more important to Managers. We can also see that managers are much less extreme in their opinions than architects, where the difference between the most considered thing, and the least considered thing, being just 0.37, whereas it's 0.66 for architects. Since architects will be working more directly with software platforms, they may have stronger opinions on what is needed, and not needed for them.

    \begin{table}[H]
        \centering
        \caption{The ranking and scores of architects, compared with managers}
        \label{tab:arch-man}
        \begin{tabularx}{\columnwidth}{X|c:c|c:c|c}

            \textbf{Aspect}                                                    & \rotatebox{90}{\textbf{Architects Rank}}    &\rotatebox{90}{\textbf{Architects Score}} & \rotatebox{90}{\textbf{Manager Rank}}  & \rotatebox{90}{\textbf{Manager Score}}& \rotatebox{90}{\textbf{Diff}}\\ \hline
            The API has code examples                                             &         1 & 0.92       &            1 & 0.79           & 0.13  \\ \hline
            The API documentation gives thorough explanations on how it works     &         2 & 0.82       &          4-5 & 0.71           & 0.14  \\ \hline
            The pricing of the software                                           &         3 & 0.76       &            3 & 0.73           & 0.09  \\ \hline
            I can have working code quickly                                       &         5 & 0.85       &            2 & 0.75           & 0.01  \\ \hline
            The software is open source                                           &         4 & 0.77       &            8 & 0.60           & 0.16  \\ \hline
            The software uses the programming language I am most comfortable with &         6 & 0.69       &          6-7 & 0.63           & 0.06  \\ \hline
            There exists an active online community around the software           &         9 & 0.63       &          4-5 & 0.71           & 0.08  \\ \hline
            The documentation is easy to navigate                                 &         8 & 0.66       &         9-11 & 0.54           & 0.09  \\ \hline
            The official website looks professional                               &         7 & 0.63       &           12 & 0.52           & 0.14  \\ \hline
            The software is compatible with different platforms                   &        10 & 0.58       &         9-11 & 0.54           & 0.04  \\ \hline
            How often the software is updated                                     &        11 & 0.52       &          6-7 & 0.63           & 0.11  \\ \hline
            The software has the same features on all different platforms         &        13 & 0.41       &         9-11 & 0.54           & 0.07  \\ \hline
            The documentation doesn't assume any prior expertise                  &        12 & 0.47       &           16 & 0.42           & 0.07  \\ \hline
            The documentation has consistent language                             &        14 & 0.49       &        13-15 & 0.44           & 0.03  \\ \hline
            The software is offered in more than one programming language         &        15 & 0.37       &        13-15 & 0.44           & 0.07  \\ \hline
            The release- and change notes are thorough                            &        16 & 0.26       &        13-15 & 0.44           & 0.18  \\ \hline
        \end{tabularx}
    \end{table}

    \subsection{Developer and Engineers compared with Managers}
    If we compare Developer and Engineer with Managers, we also find that
    they're quite different. On average, they differ in points by 0.10
    points. Their biggest differ in points is for 'The official website
    looks professional', where they differ by 0.15 points, which is also
    their biggest differ in ranking: 6 spots difference. Just like with architects vs managers, they're differences between the extreme are very different. 0.37 difference for managers, and 0.62 for developers and engineers. Once again, this seems quite natural since developers and engineers will be working with the platform more and have stronger opinions on what they need.

    \begin{table}[H]
        \centering
        \caption{The ranking and scores of developer and engineers, compared with managers}
        \label{tab:arch-devs}
        \begin{tabularx}{\columnwidth}{X|c:c|c:c|c|}

            \textbf{Aspect}                                                    & \rotatebox{90}{\textbf{D\&E* Rank}}    &\rotatebox{90}{\textbf{D\&E* Score}} & \rotatebox{90}{\textbf{Manager Rank}}  & \rotatebox{90}{\textbf{Manager Score}}& \rotatebox{90}{\textbf{Diff}}\\ \hline
            The API has code examples                                             &               1 & 0.91       &            1 & 0.79           & 0.13  \\ \hline
            The API documentation gives thorough explanations on how it works     &               2 & 0.87       &          4-5 & 0.71           & 0.14  \\ \hline
            The pricing of the software                                           &               3 & 0.84       &            3 & 0.73           & 0.09  \\ \hline
            I can have working code quickly                                       &               4 & 0.80       &            2 & 0.75           & 0.01  \\ \hline
            The software is open source                                           &               5 & 0.73       &            8 & 0.60           & 0.16  \\ \hline
            The software uses the programming language I am most comfortable with &               6 & 0.69       &          6-7 & 0.63           & 0.06  \\ \hline
            There exists an active online community around the software           &               7 & 0.67       &          4-5 & 0.71           & 0.08  \\ \hline
            The documentation is easy to navigate                                 &               9 & 0.57       &         9-11 & 0.54           & 0.09  \\ \hline
            The official website looks professional                               &              10 & 0.56       &           12 & 0.52           & 0.14  \\ \hline
            The software is compatible with different platforms                   &               8 & 0.61       &         9-11 & 0.54           & 0.04  \\ \hline
            How often the software is updated                                     &              11 & 0.52       &          6-7 & 0.63           & 0.11  \\ \hline
            The software has the same features on all different platforms         &              12 & 0.45       &         9-11 & 0.54           & 0.07  \\ \hline
            The documentation doesn't assume any prior expertise                  &              13 & 0.44       &           16 & 0.42           & 0.07  \\ \hline
            The documentation has consistent language                             &              14 & 0.34       &        13-15 & 0.44           & 0.03  \\ \hline
            The software is offered in more than one programming language         &              15 & 0.31       &        13-15 & 0.44           & 0.07  \\ \hline
            The release- and change notes are thorough                            &              16 & 0.29       &        13-15 & 0.44           & 0.18  \\ \hline \hline
            \multicolumn{6}{l}{*D\&E: Developers and Engineers}
        \end{tabularx}
    \end{table}


    \subsection{Experience}
    The responses were also divided into five groups, depending on how much
    experience in the software industry the had. There were 5 people with
    less than 5 years experience, 10 people with 5 - 10 years experience, 10
    people with 10 - 15 years experience, 12 people with 15 - 25 years
    experience and 2 people with 25+ years of experience in the software
    industry. How people ranked the aspects, depending on years of experience, can be seen in table \ref{exp}. In table \ref{tab:jobExp} we can see the division of how many years of experience related to job title.
    The overall take away is that years of experience is \textit{not} a good indicator of people's needs. There doesn't seem to be any relationship with "The more/less experience you have, the more/less something is important".
    \begin{table}[H]
        \centering
        \caption{The scoring of the aspects, grouped by years of experience.}
        \label{tab:jobExp}
        \begin{tabularx}{\columnwidth}{X|c|c|c|c|c|c}
            \textbf{Aspect}	&\textbf{\rotatebox{90}{	$<5$ years	}}&\textbf{\rotatebox{90}{	$5-10$ years	}}&\textbf{\rotatebox{90}{	$10-15$ years	}}&\textbf{\rotatebox{90}{	$15-25$ years	}}&\textbf{\rotatebox{90}{	$25+$ years}}	\\ \hline
            The API has code examples	&	0.91	&	0.93	&	0.89	&	0.85	&	1.00	\\ \hline
            The API documentation gives thorough explanations on how it works	&	0.91	&	0.86	&	0.87	&	0.72	&	0.83	\\ \hline
            I can have working code quickly	&	0.79	&	0.86	&	0.82	&	0.77	&	0.92	\\ \hline
            The pricing of the software	&	0.73	&	0.86	&	0.81	&	0.83	&	0.67	\\ \hline
            The software uses the programming language I am most comfortable with	&	0.68	&	0.73	&	0.68	&	0.74	&	0.50	\\ \hline
            The software is open source	&	0.67	&	0.62	&	0.74	&	0.71	&	0.67	\\ \hline
            There exists an active online community around the software	&	0.84	&	0.53	&	0.73	&	0.55	&	0.75	\\ \hline
            The documentation is easy to navigate	&	0.60	&	0.68	&	0.64	&	0.48	&	0.71	\\ \hline
            The official website looks professional	&	0.53	&	0.67	&	0.54	&	0.59	&	0.50	\\ \hline
            How often the software is updated	&	0.66	&	0.52	&	0.52	&	0.58	&	0.50	\\ \hline
            The software is compatible with different platforms	&	0.41	&	0.54	&	0.59	&	0.57	&	0.54	\\ \hline
            The documentation doesn't assume any prior expertise	&	0.43	&	0.57	&	0.50	&	0.39	&	0.50	\\ \hline
            The software has the same features on all different platforms	&	0.28	&	0.42	&	0.53	&	0.45	&	0.29	\\ \hline
            The documentation has consistent language	&	0.41	&	0.45	&	0.40	&	0.32	&	0.58	\\ \hline
            The software is offered in more than one programming language	&	0.28	&	0.35	&	0.35	&	0.36	&	0.25	\\ \hline
            The release- and change notes are thorough	&	0.26	&	0.34	&	0.33	&	0.30	&	0.25	\\ \hline
        \end{tabularx}
    \end{table}

    \begin{table}[H]
        \centering
        \caption{Years of experience in relation to people's job title}
        \label{tab:jobExp}
        \begin{tabularx}{\columnwidth}{l l X}
            \textbf{Experience} &  \textbf{Quantity}        & \textbf{Job Title} \\ \hline
            $< 5$ years	    &   5   &	Developers and Engineers \\
            &   0   &	Architects \\
            &   0   &	Managers \\ \
	    &   0   &	Other \\ \hline
            $5 - 10$ years	&   5   &	Developers and Engineers \\
            &   3   &	Architects \\
            &   1   &	Managers \\
            &   1   &	Other \\ \hline
            $10 - 15$ years	&   6   &	Developers and Engineers \\
            &   2   &	Architects \\
            &   1   &	Managers \\
            &   1   &	Other \\ \hline
            $15 - 25$ years	&   5   &	Developers and Engineers \\
            &   3   &	Architects \\
            &   2   &	Managers \\ \
	&   2   &	Other \\ \hline
            $25+$ years	&0&	Developers and Engineers \\
            &   2   &	Architects \\
            &   0   &	Managers \\
            &   0   &	Other \\ \hline
        \end{tabularx}
    \end{table}

    \subsection{Decision Makers}
    The results were also divided into decision makers and non-decision makers. The
    groups are of comparable size: There are 22 decision makers for groups
    and 14 non-decision makers for groups, and 3 who answered that it is not
    applicable. In table \ref{tab:decision} we can see how decision makers and non-decision makers score differently.
    Overall, non-decision makers and decision makers don't differ a lot.
    \begin{table}[H]
        \centering
        \begin{tabularx}{\columnwidth}{X|c:c|c:c|}
            \textbf{Aspect}	&\textbf{\rotatebox{90}{DM* Rank}}&\textbf{\rotatebox{90}{DM* Score}}&\textbf{\rotatebox{90}{NDM** Rank}}&\textbf{\rotatebox{90}{NDM** Score}}	\\ \hline
            The API has code examples	&	1	&	0.91	&	1	&	0.86	\\ \hline
            The API documentation gives thorough explanations on how it works	&	2-3	&	0.83	&	3	&	0.80	\\ \hline
            I can have working code quickly	&	2-3	&	0.83	&	4	&	0.79	\\ \hline
            The pricing of the software	&	4	&	0.80	&	2	&	0.83	\\ \hline
            The software uses the programming language I am most comfortable with	&	6	&	0.69	&	6	&	0.74	\\ \hline
            The software is open source	&	5	&	0.70	&	8	&	0.63	\\ \hline
            There exists an active online community around the software	&	7	&	0.62	&	5	&	0.75	\\ \hline
            The documentation is easy to navigate	&	8	&	0.59	&	7	&	0.66	\\ \hline
            The official website looks professional	&	9	&	0.58	&	9-10	&	0.61	\\ \hline
            How often the software is updated	&	11	&	0.53	&	9-10	&	0.61	\\ \hline
            The software is compatible with different platforms	&	10	&	0.55	&	12	&	0.54	\\ \hline
            The documentation doesn't assume any prior expertise	&	12	&	0.45	&	11	&	0.60	\\ \hline
            The software has the same features on all different platforms	&	13	&	0.43	&	14	&	0.46	\\ \hline
            The documentation has consistent language	&	14	&	0.38	&	13	&	0.50	\\ \hline
            The software is offered in more than one programming language	&	15	&	0.32	&	15	&	0.38	\\ \hline
            The release- and change notes are thorough	&	16	&	0.29	&	16	&	0.35	\\ \hline  \hline
            \multicolumn{5}{l}{*DM: Decision Makers, **NDM: Non-Decision Makers}
        \end{tabularx}
        \caption{The ranking and scorings of decision makers, compared with non-decision makers.}
        \label{tab:decision}
    \end{table}

    It could also be interesting to see who the decision makers are. Divided
    by job title and level, it looks like this:

    \begin{table}[H]
        \centering
        \begin{tabular}{l l l l}
            \textbf{Job Title} & \textbf{Level} & \textbf{DM*} & \textbf{NDM**} \\ \hline
            Architect               & Middle & 0               & 0                   \\
            Architect               & Senior & 7               & 2                   \\ \hline
            Developer and Engineers & Middle & 3               & 4                   \\
            Developer and Engineers & Senior & 5               & 7                   \\ \hline
            Managers                & Middle & 1               & 0                   \\
            Managers                & Senior & 3               & 0                   \\ \hline
            Other                   & Middle & 0               & 0                   \\
            Other                   & Senior & 3               & 1                   \\ \hdashline
            &        &     \textbf{22}          & \textbf{14}              \\ \hline\hline
            \multicolumn{4}{l}{*DM: Decision Makers, **NDM: Non-Decision Makers}

        \end{tabular}
        \caption{Caption}
        \label{tab:level}
    \end{table}
    As we can see in table \ref{tab:level}, and not surprising, all managers are decision makers.
    Somewhat more interesting, two architects claim they are not in a
    position to make decisions on what software others will use. We also see
    that 50\% of Middle level people are decision makers, and 64\% of Senior
    level people are decision makers.


    \subsection{Compared to Survey 1}

    The sample size is almost exactly the same in the two surveys. The first
    survey had 38 responses, whereas the second one had 39. There are two
    major differences between the two surveys. The first one had mostly just
    Developer and Engineers, 74\%, only one manager, 3\%, and just five
    architects, 13\%. As a reminder, the second survey had 53.8\% developer
    and engineers, 10.3\% managers and 25.6\% architects. The second
    difference is that in the second survey, the people were given a context
    for when they were considering the aspects.
    \\ \\
    If we compare it to the first survey, we see some anomalies. To be able
    to compare the rankings, we will exclude the two newly added questions
    in this part. In table \ref{tab:surveyDiff} we see the two surveys compared.

    \begin{table}[H]
        \centering
        \caption{The difference in score and ranking between the first and second survey.}
        \label{tab:surveyDiff}
        \begin{tabularx}{\columnwidth}{X|c:c|c:c|c:c|}
            \textbf{Aspect} & \textbf{\rotatebox{90}{S2* Rank}}&\textbf{\rotatebox{90}{S2* Points}}&\textbf{\rotatebox{90}{S1** Rank}}&\textbf{\rotatebox{90}{S1** Points}}&\textbf{\rotatebox{90}{Difference Points}} & \textbf{\rotatebox{90}{Difference Rank}} \\ \hline
            The API has code examples                                             &                        1 & 0.90                      &                        1 & 0.87                       &       +0.03 &         0    \\ \hline
            The API documentation gives thorough explanations [on how it works] &                        2 & 0.82                      &                        3 & 0.78                       &       +0.05 &        +1    \\ \hline
            I can have working code quickly                                       &                        3 & 0.80                      &                        4 & 0.76                       &       -0.04 &        +1    \\ \hline
            The pricing of the software                                           &                        4 & 0.80                      &                        5 & 0.74                       &       +0.06 &        +1    \\ \hline
            The software uses the programming language I am most comfortable with &                        5 & 0.68                      &                        7 & 0.68                       &        0.00 &        +2    \\ \hline
            \sout{The software is open source}                                    &                        - & 0.68                      &                        - & -                          &           - &         -    \\ \hline
            There exists an active online community around the software           &                        6 & 0.65                      &                        2 & 0.80                       &       -0.15 &        -4    \\ \hline
            The documentation is easy to navigate                                 &                        7 & 0.60                      &                       10 & 0.61                       &        0.00 &        +3    \\ \hline
            The official website looks professional                               &                        8 & 0.60                      &                       11 & 0.59                       &       +0.01 &        +3    \\ \hline
            The software is compatible with different platforms                   &                        9 & 0.58                      &                        6 & 0.72                       &       -0.14 &        -3    \\ \hline
            How often the software is updated                                     &                       10 & 0.58                      &                        8 & 0.63                       &       -0.05 &        -2    \\ \hline
            The documentation doesn't assume any prior expertise                  &                       11 & 0.47                      &                       12 & 0.50                       &       -0.03 &        +1    \\ \hline
            The software has the same features on all different platforms         &                       12 & 0.46                      &                        9 & 0.62                       &       -0.16 &        -3    \\ \hline
            The documentation has consistent language                             &                       13 & 0.41                      &                       13 & 0.45                       &       -0.04 &         0    \\ \hline
            \sout{The software is offered in more than one programming language}  &                        - & 0.36                      &                        - & -                          &           - &              \\ \hline
            The release- and change notes are thorough                            &                       14 & 0.33                      &                       14 & 0.41                       &       -0.09 &         0    \\ \hline \hline
            \multicolumn{7}{l}{*S2: Survey 2, **S1: Survey 1}
        \end{tabularx}
    \end{table}
    An interesting difference is the rank of 'The software is
    compatible with different platforms'. However, this aspect is very
    divided in the second survey depending on the situation. The rankings
    for this aspect look as following: Group ranks it as the third most
    important, whereas single ranks is at the 10\textsuperscript{th} spot and finally hobby at
    place number 13. Similarly, the question 'The software has the same
    features on all different platforms' has a big point difference when
    comparing the first and second survey. However, this is also a divided
    question depending on situation in the second survey. It's ranked by
    Group at 9\textsuperscript{th}, Single is tied between the 11\textsuperscript{th} and 12\textsuperscript{th} spot and for Hobby
    it's 13\textsuperscript{th}. In the first survey, it's ranked as 9 out of 14. The
    change in ranking in both these cases can therefor somewhat be explained
    by the fact that no context was given in the first survey.
    \\ \\
    But maybe most interesting is that in the first survey, 'There exists an
    active online community around the software' ranked as the second most
    important aspect, but is on average ranked as the sixth most important
    aspect in the second survey. There is no difference on the context here,
    in the second survey it's ranked at #6 by both Single and Hobby, and is
    tied between 6th and 7th spot for Group. It \textit{is} slightly more important
    for Developer and Engineers, but only by 0.04 points, so the fact that
    the first survey had them as a majority cannot explain this either.
    There does not seem to be any obvious explanation for this shift.
    \\ \\
    Below we can see the average result from the two surveys, with the
    rankings. The two questions that don't exist in the first survey have
    been excluded, and the ranking for the second survey recalculated.


    \subsection{Not considered}

    Because of the way the data pool was shaped, two groups were not looked at, which could have been interesting. The first one is company
    size. 30 out of 39 people worked in companies with more than 1000
    people, and only 4 out of 39 people worked in companies with less than
    100 people. Therefor the data pool is too small to make any larger
    claims on patterns.
    \\ \\
    The other group that could have been interesting to look at is the
    professional level of the persons. However, 77\% were senior level, and
    23\% were middle, and no one was junior level. The job title level is
    somewhat arbitrary, and it can be argued that looking at years of
    experience, where the answers are more divided, can substitute this
    angle. Most of the people with a middle level have worked less than 5 years.
    How big part of the work experience group are made up of middle and
    senior level respectively can be seen in the table \ref{tab:experience_level}.

    \begin{table}[H]
        \centering
        \caption{The link between level in job title and years of experience}
        \label{tab:experience_level}
        \begin{tabular}{|l|l|l|}
            \hline
            \textbf{Years of experience} & \textbf{Middle level} & \textbf{Senior level} \\ \hline
            \< 5 years           & 100\%         & 0\%           \\ \hline
            5-10 years        & 20\%          & 80\%          \\ \hline
            10-15 years       & 20\%          & 80\%          \\ \hline
            15-25 years       & 0\%           & 100\%         \\ \hline
            25+ years           & 0\%           & 100\%         \\ \hline
        \end{tabular}
    \end{table}

    \section{Interview Results}

    All of the interviews can be found in their full form in appendix X. In this section however
    some interesting findings in the interviews will be discussed.

    \subsection{API Documentation and Examples}
    Just like the survey suggested, API documentation and examples are extremely important. The first people look at, according to the interviews, is the examples. That seems to be the way people get started when encountering a documentation. This is, according to the interview subjects, to get an overview of the software platform. Examples give an insight into what the software can do, what it's applications are.
    \begin{quote}
        "The example is a summary without comments" \\
        --- Quality Architect
    \end{quote}
    So it's an easy way for them to quickly see if the platform is intended to do. Interview subjects also state that examples are a good way to see if it's something that quickly understand, if it's something they're used to. One of the interview subjects stated that the more generic an API is, e.g. the more applications it has, the more examples are needed.
    \begin{quote}
        "... which means that if it's a broad and generic API ... they will have to have a lot of examples" \\
        -- Architect
    \end{quote}
    The interview subjects also said they look at other things to determine if they want to use a software platform. When asked if there were any "red flags" for API documentation, several different answers were given. One stated that language inconstancy was a big warning, another if the documentation felt auto-generated, and a third if the description of the methods were lacking. People also stated they usually look at things outside of documentation when they are deciding, namely how well-known and popular the software platform seems to be.
    \begin{quote}
        "In the github you look at, is it starred, how many has downloaded it, and stuff. Is it used? Is it up-to-date?" \\
        --- Quality Architect
    \end{quote}
    \begin{quote}
        "And googling a lot. See what others are using."\\
        --- Senior Developer \\
    \end{quote}
    \begin{quote}
        "I would look both at the API documentation but also how much it's used and how much it's maintained, what was the latest changes and things like that" \\
        --- Architect
    \end{quote}
    One of the interview subjects were a manager, which turned out to have quite other priorities when it came to API documentation when he was in his working role. His opinion was that developers tend to think about more what's fun to use, rather than what is useful.
    \begin{quote}
        "I think that developers tend to play down the need for production worthy code ... [if] it looks good, if [developers] think it's fun to work with, it's good [to the developers]" \\
        --- Development Manager
    \end{quote}
    The manager was more concerned with reliability of the software platform. How likely was it to break down? And if so, what support can the developers get? Should we buy a support contract?
    \\ \\
    All the interview subjects said that they want to have working code quickly. Some of them attributed this to them being impatient or lazy, but when you started to question a bit more you realized that this is actually their way of figuring out if a software is valuable. When question about why he wants to have working code quickly, the architect stated:
    \begin{quote}
        "I don't get the feeling of the API otherwise." \\
        --- Architect
    \end{quote}
    So examples were used both to understand the underlying structure, how different part of the platform are linked up. But also for copy-pasting into your project to try it out.
    \\ \\
    The interviewees were also asked if they could ever tolerate not having working code quickly. It turns out that it's acceptable if they already know that the software is valuable and is what they are looking for. This however had the prerequisite that they would work with the software platform for a long time, and they knew that the long time it took to get started would be worth it. It however caused irritation. Consensus was though that they would rather try to find an alternative rather than to spend time to get it working, if they had the option.
    \begin{quote}
        "So if there is several different alternatives, I think I start out with another."\\
        --- Quality Architect
    \end{quote}
    \begin{quote}
        "Well, if I have a choice to find something else to use then I would just do that." \\
        --- Software Engineer
    \end{quote}
    \begin{quote}
        "...if you go to ... one API and you found it quite bad, then you skip and go the next API." \\
        --- Senior Developer
    \end{quote}
    So it's quite clear that developers are impatient and not shy of dropping alternatives for something else, if they have the choice. It is therefor important for a software platform to get started quickly, and to quickly convey to the developer what it is that the software platform can do. Developer do not want to spend a lot of time to figure out if a software platform is useful for them.
    \\ \\
    In the interviews it was also talked about what the examples should look like. Not surprisingly, most of the people said it depends on the situation. Two cases that were brought up by several interviewees was that there should be 'getting-started' kind of examples, where the example is runnable. The other case was for examples that exist in the API's method descriptions. These ones should be as isolated as possible, according to interview subjects.
    \begin{quote}
        "... as isolated as possible. So you just show one small feature..." \\
        --- Quality Architect
    \end{quote}
    It was also stated by many that if the concepts are complicated, it's preferable to have smaller, step-by-step examples rather than a big example that does the complicated thing. Especially if the example introduces a lot of new concepts that needs to be understood.
    \begin{quote}
        "...it's important to have more step-by-step things ... [if] you need some prerequisite knowledge in order to [understand] that" \\
        --- Software Engineer
    \end{quote}
    Preferably there should both exist a larger example within a bigger context, and a smaller snippet that just shows how specific methods of the API should be used. If there can only exist one of these, the concise example is preferred over the bigger ones.
    \begin{quote}
        "If I have to choose then it would be [the] more concise examples..."\\
        --- Software Engineer
    \end{quote}
    \\ \\
    One of the interview subjects had a warning on quick-working code and that it can be decieving. They had been working on a project where they found a software platform to use. They got started quickly, and it seemed to be a good platform to use for their purpose. However, once they started to try to do more advanced things using the platform, they realized that the documentation was severely lacking. To do the simple things, and get up and going was no issue. But the documentation was not well documented beyond it's simplest cases. They ended up doing and investigation, and decided it was easier to actually throw away months of work and use another platform with good documentation, rather than trying to use the platform with poor documentation.
    \begin{quote}
        "And you could do the basic things quickly and it worked well. But after a while every time you ran into a problem and you wanted to fix it and you look into the documentation it's like: it doesn't say anything here." \\ \\
        "...then we realized: 'Okay, if this continues happening you just keep losing time, every time you want to fix something'. So we kind of scraped it and did an investigation: 'Okay what are our other options?. And in this case, let's try to look at the documentation straight away and see if all of the problems we had with the older one are now gonna be fixed.'" \\
        --- Software Engineer
    \end{quote}
    \subsubsection{Takeaway from API Documentation and Examples Interviews}
    The takeaway is that examples are \textit{very} important in documentation. There needs to be examples that explains what the software platform can do and give the developer a way to get started quickly. Developers are impatient, so if you cannot provide a way to quickly give these things, they will choose another alternative. The documentation needs to explain advanced concepts step-by-step. It's also important that there exists documentation for both how you do the simple things, as well as more special cases. The language needs to be consistent, the documentation needs to be manually written with care and the methods need to have thorough explanations.
    \subsection{Release note}
    Release notes is part of all serious software platforms. But how much are they used, what do people want from them? According to the interviews, they are not used much at all. All but one interviewee said that they use release notes regularly, and some said they almost never use them, and one had never even looked at release notes in their professional career. When asked how often they read release notes the answers were:
    \begin{quote}
        "Almost never, I can say ... they are not very important to me."\\
        --- Quality Architect
    \end{quote}
    \begin{quote}
        "Never." \\
        --- Senior Developer
    \end{quote}
    \begin{quote}
        "I don't know if I've ever looked at release notes" \\
        --- Software Engineer
    \end{quote}
    The software engineer, who even has part of her working duties to present release notes, stated that she never looks at release notes herself.
    The manager said that he will look at release notes when there are major releases, to get a grip on how much time an upgrade would take.
    \begin{quote}
        "I want to know how big of a change, and how much time do we need to invest in such a change."\\
        --- Developer Manager
    \end{quote}
    But he also said that managers more likely rely on developers and architects to take that roll in general.
    \begin{quote}
        "The manager would probably ask teh architects and developers to tell how much work they needed to put in ... I don't think the manager himself would look into it"\\
        --- Developer Manager
    \end{quote}
    The architect however said that release notes are very important to him. Out of the interview subjects, he's the one with the most experience, over 25 years. He stated that he earlier in his career did not care much about release notes either. He says:
    \begin{quote}
        "A lot of year ago when I was new I would probably say the same thing, to be honest. It's more like afterwards you really realize how important it becomes when you sit there and you have the give the next version ... maybe it's easy to overlook that fact."\\
        --- Architect
    \end{quote}
    The architect says there are three different situation when he looks at release notes. The first is the initial situation, when he first encounters a software platform.
    \begin{quote}
        "...getting the feeling of how well they are documentation things ... it also gives you a feeling [of] how mature or obsolete the function is. If the the last release note is five years old ... that's a bad thing. If there [are] new release notes every day that's also a warning sign, [it] means it's not mature" \\
        --- Architect
    \end{quote}
    The second situation is the frustration phase, as he puts it.
    \begin{quote}
        "The other one is ... the frustration phase. 'Why is this not working, it should work', and then you're going to release notes. It's more like finding the issue"\\
        --- Architect
    \end{quote}
    The last situation he presents that causes him to look at release notes is when there needs to be a decision if software should be upgraded or not.
    \begin{quote}
        "Probably the real reason for release notes, it's like 'Okay, should we upgrade?' ... I think that's the key thing with release notes, that's why you have them. If it's worth to upgrade." \\
        --- Architect
    \end{quote}
    When asked how important they think release notes are, most of the interviewees say that they \textit{are} important, even if they don't use them themselves. The senior developer stated that release notes is not important. But when asked about if they thought it was the general case for developers that they do not care about release notes, the senior developer stated:
    \begin{quote}
        "No. Because my colleagues wants use to write release notes. So think ... others are actually reading them."\\ --- Senior Developer
    \end{quote}
    The software engineer, how had as part of the duties to present Qlik Sense's changes between versions, stated:
    \begin{quote}
        "Looking at how much other people are interested in having us present that, then it seem like it's quite important" \\
        --- Software Engineer
    \end{quote}
    For these two, it seems that release notes is something that they are not interested in, but have learned that others apparently do. The architect even stated that he has been taught that they are important.
    \begin{quote}
        "So I've been taught in that sense that [releaes notes] are very important. \\
        --- Architect
    \end{quote}
    For those interviewees who do look at release notes, it's quite different things they look at. The quality architect and manager states they only care about things that may break anything. The architect, as stated before, looks at several different things. The senior developer stated that she may be interested in new features, but says she finds those things through online discussions and blogs rather than release notes. The quality architect, who is part of the decision if software should be upgraded, stated he doesn't even look at release notes before that decision. Instead, they do test builds with the newer version, and if none of the tests breaks with the new version, they simply upgrade to that version.
    \begin{quote}
        "If there's a new [version], we take it. And then the release notes [are] not important. Unless the new bump breaks something. Then you have  to try to figure what has changed ... if our pipeline is green, we merge to master."\\
        --- Quality Architect
    \end{quote}
    The architect is also the only interviewee that looks at release notes before choosing a software platform.  When asked if they check the release notes before choosing a software platform, they stated:
    \begin{quote}
        "I don't think I have"\\
        --- Quality Architect
    \end{quote}
    \begin{quote}
        "Probably not ... Unfortunately not ... I probably should. But I would rely on the architects and developers to tell me if they were good enough" \\
        --- Development Manager
    \end{quote}
    When asked why they think release notes are not very important to people, according to the surveys, a few theories were presented. Mainly it was release notes are not part of every day work, and you can most of the time use a software platform without looking at release notes.
    \begin{quote}
        "It's not so often you use them ... most things don't change that much" \\
        --- Architect
    \end{quote}
    \begin{quote}
        "You can use an API without the release notes".\\
        --- Senior Developer
    \end{quote}
    When asked how poor release notes affect people, all but the architect naturally said that it does not affect them that much, since they do not use them. They however said it affects their opinion on the company.
    \begin{quote}
        "I think it's bad when it comes to ... the trust ... between whoever is using the API and whoever is developing. I think it reflects poorly on the company."\\
        --- Software Engineer
    \end{quote}
    \begin{quote}
        "You get a more serious feeling about the API if you have release notes. [Poor release notes] reflects poorly on the company"\\
        --- Senior Developer
    \end{quote}
    \subsubsection{Takeaway from Release Notes interviews}
    The takeaway seems to be that most people don't use release notes, even people who "should" use release notes. People go to great lengths to not read them. They manager relies on others to read them and the quality architect relies on his tests. The senior developer and software engineer simply don't do it. The architect cares a lot about it, that is however something that years of experience has taught him. Release notes seem to be used more as a last resort when something breaks, as is to be avoided at every cost. The release notes are however thought of by people as something very important by the interviewees. The mentality is that they should be there, and the lack of them makes the software creator seem unprofessional.
    \subsection{Online Communities}
    Online communities are places on the internet where people can discuss issues or questions they have about software. They can exist on many places. Some software companies host their own forums, other communities naturally emerge on online forums such as Stack Overflow. Sometimes they exist on both of these places. Exactly \textit{what} defines an online community can be somewhat vague. When the interviewees where asked what an online community was to them, there were several different answers given.
    \begin{quote}
        "Active discussions. That people are asking questions and they're getting answers quickly ... not only given by the main developer or developers" \\
        --- Development Managers
    \end{quote}
    \begin{quote}
        "My online community is Google"\\
        --- Senior Developer
    \end{quote}
    \begin{quote}
        "When it comes to online community, it's like anywhere on the internet."\\
        --- Software Engineer
    \end{quote}
    \begin{quote}
        "Well it's somewhere I can find something specific about [the software platform]. It could be their own community like Qlik community, [or] Stack Overflow or something where I can find a lot of information. A lively debate somewhere else where I would topically find discussions, I would say that they're pretty equal to me. In reality I don't care [where I find the information]."
        \\---Architect
    \end{quote}
    \begin{quote}
        "I many look at the github pages ... I look at Stack Overflow [too]."\\
        --- Quality Architects
    \end{quote}
    So online communities can exist on many places, and it seems that it really doesn't matter where it exists. When asked how much they used online communities, it was clear that it was a very central part of their daily life.
    \begin{quote}
        "That's very often"\\ --- Quality Architect
    \end{quote}
    \begin{quote}
        "I use them very often just to find answers. [And] asking questions I would probably say like a couple of times a month."\\
        --- Architect
    \end{quote}
    \begin{quote}
        "All the time" \\
        --- Software Engineer
    \end{quote}
    \begin{quote}
        "Daily."\\
        --- Senior Developer
    \end{quote}
    So an aspect that is used daily by developer is clearly something that is important. Aspects of a community that are important, maybe not surprisingly, is that they need to feel alive and feel helpful.
    \begin{quote}
        "I wouldn't call a dead community a community. That's an aspect of the community, that there really is somebody there. If the community isn't able to answer questions I put there then it's not [a] community I would say. That's to me the key thing about it, that if I reach out and ask something I will get an answer."\\---Architect
    \end{quote}
    \begin{quote}
        "[That] there are people out there helping. That's an active community. There shouldn't be things lying around for a lot of time, then it's a dead community."\\--- Development Manager
    \end{quote}
    \begin{quote}
        "Yeah, it should [feel alive]. But also, alive with important stuff to me. I don't want to be searching for the answers. It's better to have small pieces, than a long blog post. 'Indexible' answers."\\--- Quality Architect
    \end{quote}
    \begin{quote}
        "I get a little bit of a bad feeling when something is old. Like questions asked four years ago." \\--- Senior Developer
    \end{quote}
    \begin{quote}
        "If you see that there are ... either very little people or people that are not active in it then it's really not useful in the end. You [need] to be able to engage with other people."\\--- Software Engineer
    \end{quote}
    The development manager puts emphasis as well that the community needs to have a positive tone in how discussions are conducted.
    \begin{quote}
        "That they are polite answers and they are elaborate answers to questions. [That] there are people answering to the right level. Sometimes there are people who misuse this, like 'I have to show something to my manager, please do this for me'. I don't think you need to answer those but you should least support: 'Here's some help to get you started'"\\--- Development Manager
    \end{quote}
    When asked if they needed to feel like they're apart of the community themselves, there were mixed answers.
    \begin{quote}
        "Generally no, I would say."\\ --- Architect
    \end{quote}
    \begin{quote}
        "No, I don't think so."\\--- Quality Architect
    \end{quote}
    \begin{quote}
        "No. But it's great that others want to be part of the community because otherwise I wouldn't find my answers."\\--- Senior Developer
    \end{quote}
    \begin{quote}
        "Yeah! I would say so. Because then you see that people are actually using it. Like the documentation can be perfect but if no one using this, then why should I?"\\--- Software Engineer
    \end{quote}
    The interviewees were also asked if they felt like it was important that the company behind the software was part of the community. Here, some mixed answers were given as well. The architect stated that if it's a big company, it's needed. But for smaller, open-source things, it was not needed.
    \begin{quote}
        "Not necessarily, if they can. [If] people are there answering all the questions and they are correct, then I think the company has succeeded in something really cool. So for me it would just tell me that, yeah this is a really good product."\\--- Development Manager
    \end{quote}
    \begin{quote}
        "Yeah, I think so. And that you see who is answering what question. If it's the company or it's a user."\\--- Quality Architect
    \end{quote}
    \begin{quote}
        "Yeah, I think that's always good. I don't think it's necessary that they're of the community but I think it would make the consumer feel better about using the product. Again, coming back to the trust and relationship between the product and the user."\\--- Software Engineer
    \end{quote}
    The senior developer stated that if the company behind a software is part of the online community, it gives the feeling of the company actually caring about their users.
    \begin{quote}
        "...that [the company] actually take [their] responsibility for [their] users. For the community." \\--- Senior Developer
    \end{quote}
    However, when asked if it has the opposite effect if they're \textit{not} part of the community, she changes her answer.
    \begin{quote}
        "Actually, I don't think I care who answers the question. [But] if there are questions that are not answered, that you know that a [company]-person can answer, and doesn't do that, that's not good."\\--- Senior Developer
    \end{quote}
    When further asked if it's worth a company's time to answer community questions, the manager elaborated his answer.
    \begin{quote}
        "I think it goes down to a money question, actually. If you can get by without, because there is a lot of people answering, then  you're a lucky company, and you don't need to do that. Then you can back away if you think that the community is working by itself. [Otherwise] you probably need to be there."\\--- Development Manager
    \end{quote}
    In the interviews it was also explored if online community was just a substitute for poor documentation. When asked if they needed a community, even if the documentation was close to perfect, the general answer was yes.
    \begin{quote}
        "Yes, I would say so. Because you will run in to other kind of problems like 'Why can't I do this?'. I would very [hesitant] if there wasn't some kind of community around it"\\---Architect
    \end{quote}
    \begin{quote}
        "It would lower my thoughts about it. It has to be a really good tool to get started without [an online community]. [Online communities] can put the solution in perspective to the question, which a documentation couldn't do. If you have an area which is very generic, a community could catch that person easier than a documentation."\\--- Development Manager
    \end{quote}
    \begin{quote}
        "No, not for my purposes. [But] I want an an entry point for when I don't find my information on the web page. Where could I post? And if that is a community, then my answer is yes."\\ --- Quality Architect
    \end{quote}
    \begin{quote}
        "Maybe it's not that important when you have great documentation. But I mean, you always get stuck on things. So then it's good with a community. [But] documentation is prior one."\\--- Senior Developer
    \end{quote}
    \begin{quote}
        "Even when you have a documentation that's average or even good, it's always nice  to have online communities because you have then people that run into the same issues as you do." \\--- Software Engineer
    \end{quote}
    The software engineer also goes on to say that online communities are comforting, to know that others are using the platform as well.
    \\ \\
    The architect was the only one who checked online communities before deciding if he wanted to use a software platform. He stated that it was quite hard to get a grip on if a community is useful or not, when you're not using the platform yet. But he said that reading threads was still helpful, to see what the community was like.
    \begin{quote}
        "You can get a feeling if you follow the discussion threads and things like that to see that... If most of them end in... like people don't understand anything; that's a bad thing. That's a very important answer [too], that there is no answer."\\---Architect
    \end{quote}
    Several of the other interviewees stated that they usually just use online communities when they are stuck on something, and only then. The quality architect stated that he didn't feel the need to check the online community before deciding.
    \begin{quote}
        "Not if I find good information on the github page or something. Then I don't have any reason to check out the community."\\--- Quality Architect
    \end{quote}
    \begin{quote}
        "No I don't [check the online community before deciding]." \\--- Senior Developer
    \end{quote}
    When finally asked to put concisely what they want from an online community, it was clear that they mostly just want answers to their problems solved easily.
    \begin{quote}
        "Answers."\\
        --- Senior Developer
    \end{quote}
    \begin{quote}
        "Structured, indexible solutions"\\ --- Quality Architect
    \end{quote}
    \begin{quote}
        "I want answers to my questions, I think that's it." \\ --- Development Manager
    \end{quote}
    \begin{quote}
        "Activity and people interacting with each other. And having questions and examples."\\--- Software Engieneer
    \end{quote}
    \subsubsection{Takeaways from Online Communities interviews}
    The takeaway from the interviews around online communities shows a few things. One is that online communities is something primarily used by people when they're stuck and have a specific problem that the documentation may not cover. Where this community exists does not really matter to people, the priority is if the information they need exists or not. Most of them however don't care if \textit{they're} part of the community. It simply acts as a problem solver. They also point out that it's not necessary that the company behind the software is part of the community, but that it doesn't hurt and can build a trust between the company and the user. As much as it is a problem solver for when the documentation doesn't give them the answer, they still want an online community even with a very good documentation. The primary reason for this is that there is always problems and situations that are too specific for a documentation to cover. Online communities are used by the interviewees very often, and is therefor quite important to them. However, documentation is \textit{even more} important.
    \section{Recommendations For Software Platforms}\label{sec:recommendations}
    \subsection{How often the software is updated}
    \begin{figure}[H]
        \centering
        \includegraphics[width=\linewidth]{scorelines/aspect1.png}
        \includegraphics[width=\linewidth]{dxscorelines/dxaspect1.png}
        \includegraphics[width=\linewidth]{contextlines/contextline1.png}
        \caption{Scoring for "How often the software is updated"}
        \label{fig:aspect1}
    \end{figure}
    Updating of the software platform is of course important. Bugs need to be fixed, the compatibility for different platforms improved and new features might be added from time to time. In figure \ref{fig:aspect1} we can see how it's ranked. As we can see it's ranking somewhere in the middle. We also see that the DX impact is bigger than the consideration. As mentioned in section \ref{consisQuestions}, the DX-question and consideration question differed quite a bit for this aspect, and we should careful to correlate these two. The DX impact shows however that being quick to address bugs has a bigger impact than the negative impact of being slow to address bugs.\\ \\The interviews showed that how often the software is updated can be seen as an indicator for how mature the software platform is. Too often and it will scare people away, as it's an indicator that there is a lot of bugs or the software is mature. If the software is updated not often enough it's an indicator that the software platform feels abandoned or not prioritised. The recommendation is to plan your updates carefully, try to lump small updates together into bigger ones, as to not update too often. Exemptions from this is critical bug fixes, such as relating to security or breaking bugs.\\ \\
    \textbf{Verdict: Somewhat important, but keep in mind. Medium effort, medium payoff.}
    \subsection{I can have working code quickly}
    \begin{figure}[H]
        \centering
        \includegraphics[width=\linewidth]{scorelines/aspect2.png}
        \includegraphics[width=\linewidth]{dxscorelines/dxaspect2.png}
        \includegraphics[width=\linewidth]{contextlines/contextline2.png}
        \caption{Scoring for "I can have working code quickly"}
        \label{fig:aspect2}
    \end{figure}
    Working code quickly has been shown to be important to developers. In figure \ref{fig:aspect2} we can see the scoring. With an average score of 0.80/1.00 for the job titles, this is an important aspect that should always be considered when creating a software platform. There's also no major difference depending on the context. We also see that it has a strong DX impact. Not only are developers impatient people that want results quickly overall, working code quickly is developers way to figure out if a software platform is useful. Software developers quickly abandon software if they don't see it's value. Because working code quickly is their way of evaluating new software, it's extremely important to be able to provide this. The recommendation is to easily show how to get started. It should both be front and centre when you visit the website, and the example should be quick without being too simple. The effort to have an example that is easy to follow  and makes the user understand can take time, and be a bit of an effort, but is definitely worth it.\\ \\
    \textbf{Very important, always keep in mind. Medium effort, high payoff.}
    \subsection{The API documentation gives thorough explanations on how it works}
    \begin{figure}[H]
        \centering
        \includegraphics[width=\linewidth]{scorelines/aspect3.png}
        \includegraphics[width=\linewidth]{dxscorelines/dxaspect3.png}
        \includegraphics[width=\linewidth]{contextlines/contextline3.png}
        \caption{Scoring for "The API documentation gives thorough explanations on how it works"}
        \label{fig:aspect3}
    \end{figure}
    Documentation is the heart of a software platform, providing the explanation of how to use it. In figure \ref{fig:aspect3} we can see how it scored. With an average score of 0.82/1.00 for the job titles it's paramount that this aspect is taken care of. We also see that it is important for all contexts. This is also reflected in the DX impact result, where we see that poor documentation has a strong negative impact, and vice versa. Poor documentation was shown to cause developers to quickly abandon software. It doesn't matter how good your software platform is. If you don't have a good documentation that clearly explains how you're suppose to use it, people will not use your platform. You \textit{must} make sure you explain all parts of your platform. The effort to have thorough documentation is big, but is worth it when you see how important it is. While a documentation needs to be thorough, it's important to ease the reader into it. Presenting all information at once will overwhelm the reader. Images are often a good way to explain how things are interacting. The recommendation is to put a lot of effort into this. Before going into too much detail, give the reader an overview of what the documentation will convey, and how things are interconnected. Images are often an effective way to do this. Listen carefully to any questions you get from users. If a lot of users find the same things difficult, it can be an indicator that the documentation isn't thorough enough. A good method could be to read online discussions, and see what people have difficulty with.\\ \\
    \textbf{Verdict: Very important, always keep in mind. High effort, high payoff.}
    \subsection{The API has code examples}\label{hascodeexamples}
    \begin{figure}[H]
        \centering
        \includegraphics[width=\linewidth]{scorelines/aspect4.png}
        \includegraphics[width=\linewidth]{dxscorelines/dxaspect4.png}
        \includegraphics[width=\linewidth]{contextlines/contextline4.png}
        \caption{Scoring for "The API has code examples"}
        \label{fig:aspect4}
    \end{figure}
    You can explain concepts and methods in text, but often an example is the best way to convey something quickly. The interviews showed that examples is the first thing people look at when encountering documentation. It is therefor important that the example is front and centre in documentation. The examples are used for many things too. It's for copy-pasting into people's projects, getting an overview how things work and are linked together as well as to simply see how things should be used. The effort to construct good examples is quite big. The recommendation is to \texit{always} have a simple example with all methods and concepts, and if possible more advanced examples too. The simple example should be concise, and show the standard situation. It could be tempting to show something fancy. This however increases the risk for confusion. If you're going to show advanced situations, do it in step-by-steps as to not confuse the user. In figure \ref{fig:aspect4} you can see how it scored. With an average score of 0.90/1.00 for the job titles and a strong DX impact it's paramount that this aspect is taken care of. It's also important in all contexts. It's the first thing users look at, and if they don't understand you risk losing them. \\ \\
    \textbf{Very important, always keep in mind. High effort, high payoff.}
    \subsection{The documentation doesn’t assume any prior expertise}
    \begin{figure}[H]
        \centering
        \includegraphics[width=\linewidth]{scorelines/aspect5.png}
        \includegraphics[width=\linewidth]{dxscorelines/dxaspect5.png}
        \includegraphics[width=\linewidth]{contextlines/contextline5.png}
        \caption{Scoring for "The documentation doesn’t assume any prior expertise"}
        \label{fig:aspect5}
    \end{figure}
    Software platforms will naturally have concepts that are new to people. Whenever a new concept is used, you risk confusing the user if it's not explained. In figure \ref{fig:aspect5} you can see that it does not score very high, with an average of 0.47/1.00 for the job titles. It's the same for the contexts. We also see that the DX impact scores about the same as the consideration questions. This doesn't mean that it can be completely ignored, but people are not scared away by new concepts. The recommendation is to link to an explanations of new concepts where ever they're used. This effort is not very big, but solves the problem. \\ \\
    \textbf{Verdict: Not very important, but don't ignore. Low effort, medium payoff.}
    \subsection{The documentation has consistent language}
    \begin{figure}[H]
        \centering
        \includegraphics[width=\linewidth]{scorelines/aspect6.png}
        \includegraphics[width=\linewidth]{dxscorelines/dxaspect6.png}
        \includegraphics[width=\linewidth]{contextlines/contextline6.png}
        \caption{Scoring for "The documentation has consistent language"}
        \label{fig:aspect6}
    \end{figure}
    Having consistent language is an indicator of a documentation that has been thoroughly reviewed and worked with. In figure \ref{fig:aspect6} however we can see that it scores quite low, with an average of 0.41/1.00 for the job titles. It's about the same for the contexts. We also see that although it's not quite often considered, it has a stronger DX impact. It's however still close to neutral. A documentation with inconsistent language is still very much usable, it just increases the risk for confusion. The effort to fix an already inconsistent documentation is very high. The recommendation is to define a vocabulary that should be used by the documentation writers before starting to write documentation, or from now on if the documentation has already been written. This will ensure that new documentation has consistency, as long as the documentation writers make sure to use the vocabulary. Because of the high effort, and low importance according to this research, going through documentation and fixing inconsistency should not be of very high priority.\\ \\
    \textbf{Verdict: Not very important, but don't ignore. High effort, low payoff.}
    \subsection{The documentation is easy to navigate}
    \begin{figure}[H]
        \centering
        \includegraphics[width=\linewidth]{scorelines/aspect7.png}
        \includegraphics[width=\linewidth]{dxscorelines/dxaspect7.png}
        \includegraphics[width=\linewidth]{contextlines/contextline7.png}
        \caption{Scoring for "The documentation is easy to navigate"}
        \label{fig:aspect7}
    \end{figure}
    As mentioned before, developers are impatient people. A documentation that is easy to navigate, means developers can find what they're looking for quicker. In figure \ref{fig:aspect7} we see that it scored close scored somewhere in between 'Sometimes consider' and 'Often consider' for both the job titles and the contexts. It also has a DX impact that is above neutral. The recommendation is therefor to have a clear navigation structure, with carefully chosen titles that makes the user understand what each link will lead them to. For larger pages, having a table of contents at the top that shows what sections it contains is also recommended. For smaller sub-pages a short paragraph could be good to explain to the user what he or she can expect to find on the page. This gives the possibility to the user to be more efficient when searching for specific things. The effort to do this can be somewhat high, especially the careful title naming. The titles can easily be too vague or ambivalent. Having an efficient search function for your documentation will also make the navigation more easy. To implement this is a higher effort than structuring the documentation. The recommendation is therefor to start with the structure, and after that work on the search function. \\ \\
    \textbf{Verdict: Somewhat important, keep in mind. Medium-high effort, medium payoff.}

    \subsection{The official website looks professional}
    \begin{figure}[H]
        \centering
        \includegraphics[width=\linewidth]{scorelines/aspect8.png}
        \includegraphics[width=\linewidth]{dxscorelines/dxaspect8.png}
        \includegraphics[width=\linewidth]{contextlines/contextline8.png}
        \caption{Scoring for "The official website looks professional"}
        \label{fig:aspect8}
    \end{figure}
    The official website is often a company's face outwards, and a user's first encounter with the company. You should not downplay the importance of first impressions. In figure \ref{fig:aspect8} we can see that it scored somewhere in between 'Sometimes consider' and 'Often consider' with the average score of 0.60/1.00 for the job titles. It's the same for the contexts. The DX-impact for a good or bad website is however neutral. The recommendation is to spend some time to make sure your official website looks professional. Test it for different browsers and devices. The effort to do this should not be high for a software company who creates software platforms. If a company whom provides software platform solutions cannot provide a professional looking website, the impression it gives is that they won't provide a good platform either.
    \\ \\
    \textbf{Verdict: Somewhat important, keep in mind. Medium effort, medium payoff.}

    \subsection{The pricing of the software}
    \begin{figure}[H]
        \centering
        \includegraphics[width=\linewidth]{scorelines/aspect9.png}
        \includegraphics[width=\linewidth]{dxscorelines/dxaspect9.png}
        \includegraphics[width=\linewidth]{contextlines/contextline9.png}
        \caption{Scoring for "The pricing of the software"}
        \label{fig:aspect9}
    \end{figure}
    Money always plays a roll in any company, and the price of a software cannot be ignored. You can make money from your software in many ways, through licensing, subscriptions, support contract, and more. It's out of the scope of this research paper on how one should make money from software platforms. In figure \ref{fig:aspect9} we see that it's quite important to people. It's especially important for people working on a hobby projects! This is another aspect, as mentioned in section \ref{consisQuestions}, where the DX-questions and considerations questions are a bit different, and one should be careful to draw parallels. We can see however in the consideration scoring that pricing is very important. We can tell by the DX-scoring that clearly showing the pricing of the software has a very strong positive impact, and hiding the pricing away has a strong negative impact. The recommendation for pricing, in order to give a good experience for developers, is to clearly state the pricing of the software platform. Especially if your target audience is people working on hobby projects. Whichever money making model the software platform uses, the cost to inquire it for a developer should not be hidden away. \\ \\
    \textbf{Verdict: Important, keep in mind. Low effort, high payoff.}

    \subsection{The release- and change notes are thorough}
    \begin{figure}[H]
        \centering
        \includegraphics[width=\linewidth]{scorelines/aspect10.png}
        \includegraphics[width=\linewidth]{dxscorelines/dxaspect10.png}
        \includegraphics[width=\linewidth]{contextlines/contextline10.png}
        \caption{Scoring for "The release- and change notes are thorough"}
        \label{fig:aspect10}
    \end{figure}
    Release notes was shown to be a quite non-important aspect, as seen in figure \ref{fig:aspect10}, especially for the people who are supposedly using them: developers and architects. We can see by the DX-impact that poor release notes has a bit of a negative impact. It's however still below neutral. The interviews showed that release notes are a last resort for people, and ignored as much as possible. The recommendation is therefor to not spend too much time on release notes. But make sure to not ignore them. If you make sure to have good commit messages, these can act as a good base for writing the release notes. This minimises the time that has to be spent on writing the release notes.\\ \\
    \textbf{Verdict: Not important, but don't ignore. Medium effort, low payoff.}

    \subsection{The software has the same features on all different platforms}
    \begin{figure}[H]
        \centering
        \includegraphics[width=\linewidth]{scorelines/aspect11.png}
        \includegraphics[width=\linewidth]{dxscorelines/dxaspect11.png}
        \includegraphics[width=\linewidth]{contextlines/contextline11.png}
        \caption{Scoring for "The software has the same features on all different platforms"}
        \label{fig:aspect11}
    \end{figure}
    If your software platform supports many different platforms, such as operation systems and browsers, it takes time and effort to make sure that they work the same in all situations. In figure \ref{fig:aspect11} we see that this scores kind of low, and the DX-impact is neutral. It's somewhat higher in the context of deciding for a group. The effort to make sure that all features exist on all platforms is very high. The recommendation is therefor to clearly state to the users what features are offered on what platforms. If you decide to make sure that the features are consistent on all platforms, be aware that the effort to do this is very high and may not have a big payoff. \\ \\
    \textbf{Verdict: Not very important. High effort, low-medium payoff.}

    \subsection{The software is compatible with different platforms}
    \begin{figure}[H]
        \centering
        \includegraphics[width=\linewidth]{scorelines/aspect12.png}
        \includegraphics[width=\linewidth]{dxscorelines/dxaspect12.png}
        \includegraphics[width=\linewidth]{contextlines/contextline12.png}
        \caption{Scoring for "The software is compatible with different platforms"}
        \label{fig:aspect12}
    \end{figure}
    Having your software platform be compatible with many different platforms is a requirement for some users, a bonus for some. The effort to do this is however high. If you plan to implement this, make sure that the increase in potential users is worth the effort to do this. You need to be aware that is highly increases all future maintenance for your platform. As we can see in figure \ref{fig:aspect12}, the scoring is low to average for both job titles and DX impact. The one outlier is in the context of deciding for a group, where it's very important. The recommendation is to skip this aspect, if you are not sure that the increase in users is worth the effort. \\ \\
    \textbf{Verdict: Not very important. High effort, low-medium payoff.}

    \subsection{The software is offered in more than one programming language}
    \begin{figure}[H]
        \centering
        \includegraphics[width=\linewidth]{scorelines/aspect13.png}
        \includegraphics[width=\linewidth]{dxscorelines/dxaspect13.png}
        \includegraphics[width=\linewidth]{contextlines/contextline13.png}
        \caption{Scoring for "The software is offered in more than one programming language"}
        \label{fig:aspect13}
    \end{figure}
    Having your platform be compatible with more than one language takes a big effort. It takes a long time to develop and doubles the maintenance needed. As we can see in figure \ref{fig:aspect13} it's not ranked high at all, both for the consideration part and the DX part. The recommendation is therefor to ignore this aspect. The effort compared with the payoff is not worth it. \\ \\
    \textbf{Verdict: Not important. High effort, low payoff.}

    \subsection{The software is open source}
    \begin{figure}[H]
        \centering
        \includegraphics[width=\linewidth]{scorelines/aspect14.png}
        \includegraphics[width=\linewidth]{dxscorelines/dxaspect14.png}
        \includegraphics[width=\linewidth]{contextlines/contextline14.png}
        \caption{Scoring for "The software is open source"}
        \label{fig:aspect14}
    \end{figure}
    Being open source has both benefits and limitations. It is out of the scope of this research paper to evaluate these. But as we can see in figure \ref{fig:aspect14} it's quite important to developers and architects. The DX impact shows however that being close-sourced doesn't have a very big impact. The recommendation is therefor to try to be open source if you can, but it's not necessary. If it's not something that the company is used to, be aware that it requires other ways of working than conventional software that is close-source. If you decide to make an open-source software platform, make you have the expertise in the company to be able to handle this type of software. It has a potential to increase the amount of users, but if you don't use the benefits that open-source gives, it may not pay off. \\ \\
    \textbf{Verdict: Important. Medium-high effort, high payoff.}

    \subsection{The software uses the programming language I am most comfortable with}
    \begin{figure}[H]
        \centering
        \includegraphics[width=\linewidth]{scorelines/aspect15.png}
        \includegraphics[width=\linewidth]{dxscorelines/dxaspect15.png}
        \includegraphics[width=\linewidth]{contextlines/contextline15.png}
        \caption{Scoring for "The software uses the programming language I am most comfortable with"}
        \label{fig:aspect15}
    \end{figure}
    Developers tend to be more comfortable working with certain languages, and it's unique for every person. As we can see in figure \ref{fig:aspect15} it is definitely something people consider. The DX impact shows however that they're not immediately deterred by software platforms that is not in their favourite language. As stated before, providing several languages is not worth the effort. The recommendation is therefor to be attentive to what programming languages are popular, and to see how the trends change. Be aware however that these trends are just that, trends. New languages emerge all the time as "the new hot thing". The safer bet is to offer your software platform in a popular language, that is predicted to be popular for quite some time. \\ \\
    \textbf{Verdict: Important. Medium effort, high payoff.}

    \subsection{There exists an active online community around the software}
    \begin{figure}[H]
        \centering
        \includegraphics[width=\linewidth]{scorelines/aspect16.png}
        \includegraphics[width=\linewidth]{dxscorelines/dxaspect16.png}
        \includegraphics[width=\linewidth]{contextlines/contextline16.png}
        \caption{Scoring for "There exists an active online community around the software"}
        \label{fig:aspect16}
    \end{figure}
    It can be difficult, if not impossible, to forcefully create an online community around your software platform. It's not up to the company if people engage in online discussions, it's up to the users themselves. The company can however provide platforms for discussions. In figure \ref{fig:aspect16} we see that it's quite often considered, and a good community has above neutral DX impact on developers.

    The effort to create an online forum for users to discuss doesn't have to be too big, but acts as a starting point for a community to be created. Often communities will naturally emerge on other places, such as Stack Overflow. The company can help the online community prosper by keeping an eye on these communities as well, and answer question there. This also builds a trust between the user and the company.

    Online communities is primarily used for questions and answers to help out when the documentation is not enough. The effort to make sure your online community has the answers they need is somewhat big. The recommendation is to take extra care of your online community when it's new and small, and less when it's big enough that it takes care of itself. Never ignore it though, pay attention to suggestions and issues that your online community has. \\ \\
    \textbf{Verdict: Important, keep in mind. Medium effort, medium-high payoff.}
    \section{Evaluation of Qlik Core}
    Finally, it will be evaluated how well Qlik Core follow the recommendations given in \ref{sec:recommendations}. This is done aspect-by-aspdect, and given a verdict based on how important the aspect is compared with how well they follow the recommendation.
    \subsection{How often the software is updated}
    Qlik Core is a quite new product, which left beta in mid-fall of 2018. At the date of writing this paper, they've made four updates since they left beta. They follow the recommendations, where they don't release too often, but clump them together into major releases.\\ \\
    \textbf{Importance: Somewhat Important.}\\
    \textbf{Verdict: Pass}
    \subsection{I can have working code quickly}\label{workincode}
    When evaluating this aspect, I tried to visit the QC website as if I were new users. The following text is the process in doing so, and what issues I had.\\ \\
    When first visiting the website, my first instinct was to look for the phrases 'Try' or 'Getting Started'. You're greeted by a 'Try For Free', which I assumed was going to lead me to a page how I get started trying it out. That however lead me to a page about licensing. After that I saw the 'Get Started' in the top menu, which was what I was looking for initially. I afterwards realised that there was a button for 'Start Tutorials' on the start page, which leads the me to the same place as 'Getting Started'. I somehow missed this button. While this may be a one-off mistake by just us, it indicates that there's room for improvement. Rephrasing of these buttons may be in order. The start page can be seen in figure \ref{fig:startpage}. \\ \\
    \begin{figure}[H]
        \centering
        \includegraphics[width=\linewidth]{qlikCoreWebsite20190124.png}
        \caption{The Qlik Core start page on the 24\textsuperscript{th} January 2019}
        \label{fig:startpage}
    \end{figure}
    The 'Get Started' has a flow of text, where the actual tutorials are linked in the text flow. This requires the user who just wants to get started to read through the text. While this may be handy to put the user into context, when it comes to getting started quickly it slows the process down. Having these links be more in focus may speed up that process. The page can be seen in figure \ref{fig:gettingstarted}.
    \begin{figure}[H]
        \centering
        \includegraphics[width=\linewidth]{qlikCoreWebsite-GetStarted20190124.png}
        \caption{The Qlik Core Getting Started page on the 24\textsuperscript{th} January 2019}
        \label{fig:gettingstarted}
    \end{figure}
    Already one the first step I was at risk of being confused. They prompt you to run a command, but don't specify where. Clarifying that it should be in the newly cloned folder takes away that risk of confusion.\\
    The first tutorial called 'Hello-Engine' simply sets up the possibility to communicate with the engine. It was not clearly explained in the tutorial what the goal of the tutorial was, so when I finished I did not fully understand what I had finished. I thought that I had done something wrong, my thoughts were "This can't be it?" when I had actually completed the tutorial. Overall I had a lack of achievement, and did not feel I had learned anything about what Qlik Core is or can do. The following tutorial 'Hello Data' suffered a bit from the same issues. The third tutorial 'Hello Visualisation' was the first tutorial where I felt I had learned something about what Qlik Core can do. A tutorial should give the user a sense of achievement, and make the user feel like he or she learned something they did not previously know. In my opinion, these tutorials could be merged into one longer tutorial, divided into three sections. \\\\
    For the next tutorial 'core-orchestration' I got stuck immediately. The tutorial prompts you to get a licence. Even though the process of obtaining one was fairly easy (just follow a link and give your name and email) this may scare users away since it stops the coding for administrative work. Furthermore, the process of figuring out how to configure the license was extremely cumbersome. I had to search for a long, long time on how I should do it, looking through lots of different links. In the end, I had to follow a total of five links away from the tutorial page until I found an example, inside a file in a GitHub project. Then I understood what it should look like. It was also very much a copy-paste way of doing it, without me understanding \textit{why} it should be configured this way. Afterwards, when I talked to the people whom had created the tutorial, it turns out that getting a license was only optional for the tutorial, even though it was stated that it was needed.\\\\
    The 'core-orchestration' tutorial also has a lot of prerequisites on knowledge, requiring the user to know the concept of "Swarm" in the third-party software Docker. I still tried to follow the guide, which ultimately did not work. When talking to the Qlik Core team, it turns out that this is a very special case for using Qlik Core, and will not be used by most people. The way it's presented, right after the simplest tutorials, is problematic. This will scare users away. It needs to be explained that this is a special and advanced case. \\\\
    The fourth tutorial was about authorisation. This was also a special, advanced case which did not teach the user about what Qlik Core can do. It was also hard to follow, with explanations that were brief and hard to understand.\\\\
    The last tutorial is related to data loading. These tutorials did not require as much prerequisite knowledge. It however failed as well. When talking to the tutorial constructor, it turns out that the tutorial was out date with what commands that should be run. There were also errors in what versions should be run. Ultimately, this tutorial also failed to be completed.\\\\
    Overall, getting started felt difficult, cumbersome and irritating. I was stopped by errors, getting a licence that I did not actually need, setting up configurations and out-of-date tutorials. I also felt like I was was not actually learning anything about what Qlik Core can do. The tutorials need to be reworked in my opinion, with more simple use cases. And most importantly, tutorials where the user actually gets to change some code and feel like they learn something, rather than running downloaded scripts. The goal when making the tutorials should be "How do we get the users interested in Qlik Core, and make them feel like they are able to use it."\\\\
    \textbf{Importance: Very important.}
    \textbf{Verdict: Failed.} The tutorials needs to be both reworked, and probably extended with more simple examples. The tutorial needs to be created with the goal of getting new users to \texit{understand} how Qlik Core works and show that them that it solves problems in an easy and efficient way.
    \subsection{The API documentation gives thorough explanations on how it works}\label{APIexplained}
    Qlik Core is a group of different services running together. How these are linked together, is nowhere explained.\\
    On the QC website one can find the API documentation. It turns out that the documentation there however, is not directly under the Qlik Core teams control. It's pulled from other sources and automatically generated. In the end however, it's the Qlik Core team that is responsible for what is on their website. \\ \\
    At the heart of it all is the QAE, which uses several different APIs: Qlik's own QIX API (built on JSONRPC), a REST API, Data Connector and Analytical Connector. The last two are based on gRPC (Google's implementation of RPC). In the introduction to the QAE it is explained that the different APIs are meant to be used in different use cases, and each of these APIs has their own API documentation. An error that exists in all documentation is that they are out of data. At time of writing, the latest release is 12.331.0, and the documentation is for version 12.292.0. Another thing that is true for all API documentations is that there are close to no examples. \\ \\
    \textbf{QIX API Documentation}\\
    For their own QIX API, it's lacking a lot. The documentation has been split up into several different sub-categories. They are "Definitions", "Global", "Doc", "Generic Obj", "Generic Bookmark", "Generic Dimension", "Generic Measure" and "Generic Variable". None of the sub-pages has an introduction or description of what the page contains. \\
    The sub-page "Definitions" is the one that is lacking the most. The definitions page is simply just a long list of a total of 217 definitions. In figure \ref{fig:qixmethod} we can see how an entry in "Definitions" typically looks like.
    \begin{figure}[H]
        \centering
        \includegraphics[width=0.5\linewidth]{APIexampleQIX.png}
        \caption{An example of how a method for the "Definitions" part of QIX API looks like. It has no description for the entry, and one of the fields has no description as well.}
        \label{fig:qixmethod}
    \end{figure}
    This is not an outlier. 36 fields had \textit{No description}, and out of the 217 entries, only 60 has a text explaining what the definition is or does. This does \texit{not} follow the recommendations given in this paper, and need to be taken care of as soon as possible.\\
    For the other API sub-pages of QIX, most of the entries has descriptions, with just a few exceptions.\\ \\
    \textbf{REST API Documentation}\\
    The REST API seems like the most mature of the four APIs. It has description for all parts, with parameters and responses defined. It also states what permissions are needed. In figure \ref{fig:RESTmethod} we can see what an entry in the REST API documentation typically looks like.
    \begin{figure}[H]
        \centering
        \includegraphics[width=0.5\linewidth]{APIexampleREST.png}
        \caption{An example of how an entry in the REST API documentation typically looks like.}
        \label{fig:RESTmethod}
    \end{figure}
    At the top of the page however is a link called "Qlik Associative Engine API specification", which is broken. This immediately gives the user a feeling of an immature documentation.
    \\
    \textbf{Data Connector API Documentation}\\
    In the documentation for Data Connector we once again have the issue of missing descriptions. Its lack of descriptions is however not as bad as the QIX documentation. It's also the only documentation that has a figure that explains how it's interacting with other services.\\
    \textbf{Analytical Connector API Documentation}\\
    This documentation is on par with the REST API documentation in how mature it is. It only has one field without a description. It is however, just like all other documentation, lacking an intro and an explanation on where and when it's suitable to use this.\\
    \textbf{License Documentation} \\
    As discussed in \ref{workincode}, the licensing caused a lot of confusion. How to actually set up the licensing is not really explained on the page. It figure \ref{fig:APIlicenese} we can see how they explain of how to setup your license. What a \texttt{<License service URL>} should look like, is not explained. The best way to fix it was to open an example in GitHub and look in the configuration file.\\
    \begin{figure}[H]
        \centering
        \includegraphics[width=\linewidth]{APIExampleLicense.png}
        \caption{The description on how to pass the licensing to the Qlik Associative Engine.}
        \label{fig:APIlicenese}
    \end{figure}
    \textbf{Importance: Very important.}\\
    \textbf{Verdict: Failed. Overall, the documentation feels overwhelming. It's hard to get a grip on when, where and how the APIs are supposed to be used. The documentation is not really explained at all. An introduction, with images, explaining how things are interconnected would be a great way to start. The licensing page needs rework too.}
    \subsection{The API has code examples}
    The APIs have close to \texit{no} examples at all. The few examples that exists are in the introduction part, and are snippets without any context. The recommendation for this aspect is to always have small examples, and preferably longer examples as well. As stated in \ref{hascodeexamples}, code examples are used for many things: getting an overview as to understand the platform, getting started and seeing how code is used. Even more alarming, the tutorials that is suppose to get people started is without examples. They simply rely on the user opening and project in GitHub and reading through it. API examples is the most important aspect according to this research paper, and the lack of them can been felt heavily. It's hard to get an understandable overview, hard to get started and hard to see how the APIs should be used.\\ \\
    \textbf{Importance: Very important.} \\
    \textbf{Verdict: Failed.} The software platform has almost no examples at all, which makes it hard to use in a lot of ways. This needs to be of very high priority to fix.
    \subsection{The documentation doesn’t assume any prior expertise}
    Qlik states that this documentation requires you to know some things before you begin. Most central is the third-party software Docker. By doing this, Qlik rejects the responsibility to explain how these things work. However, it feels like the documentation take \textit{too} little responsibility in explaining new concepts. The documentation also uses some unexplained expressions. One example of this is the terms to do something "On The Fly". This is a concept used throughout the documentation, but never explained what exactly it means. \\ Their way of dealing with new concepts however follows the recommendations of linking to places where an explanation is given. Qlik should be aware that the threshold of starting with Qlik Core is high for a new person because of the high requirements in prerequisite knowledge.\\ \\
    \textbf{Importance: Not Very Important}\\
    \textbf{Verdict: Pass.} If Qlik Core gives more examples and gets better at explaining its documentation, this aspect does not need fixing. It follows the recommendations. Combined with this aspect's importance being low, this aspects gets a pass.
    \subsection{The documentation has consistent language}
    As mentioned in \ref{APIexplained}, the Qlik Core team are not directly responsible for the documentation for the APIs. When contacting the people responsible for the different parts, it turns out that there is not central vocabulary used. When reading through the documentation, it does however not seem inconsistent in its wording. The effort to produce a vocabulary to be used by many different teams is high, and inconsistency in the language does not seem to be an issue. \\ \\
    \textbf{Importance: Not very important}\\
    \textbf{Verdict: Pass.} The recommendation is to use a central vocabulary. The language is however not inconsitent without it, and because of the high effort to implement it, this gets a pass anyway.
    \subsection{The documentation is easy to navigate}
    There are a few issues with the navigation for the documentation. The first is that on the start page, see figure \ref{fig:startpage}, there is no clear way of how to reach the API documentation. It turns out that the documentation is hidden in "Services", see figure \ref{fig:navgiation}.
    \begin{figure}[H]
        \centering
        \includegraphics[height=\linewidth]{APInavigation.png}
        \caption{The sidebar navigation. The API documentation is found inside "Services".}
        \label{fig:navgiation}
    \end{figure}
    The recommendation is to carefully choose your titles. API documentation being found inside Services is not very obvious, according to us. The sub-pages also have some very vague names. The services' APIs are quite big. For example, the QIX Definitions part having 217 entries in just one long list. Navigating the API documentation feels cumbersome and overwhelming. Restructuring this into smaller sections, with aptly named headings would make it easier.\\
    The Qlik Core website has a search function, which searches both their own documentation and third-party software documentation. The search function however is limited. It does not allow you to search for exact phrases, normally done by putting a phrase within citation marks. You cannot tell the search to \textit{not} include certain words or phrases, normally done by adding a minus before the word or phrase. In the end, it's probably easier to use a search engine like Google, than Qlik Core's own search function.\\\\
    \textbf{Importance: Somewhat Important}\\
    \textbf{Verdict: Failed.} The documentation is hard to find, somewhat hard to navigate and the search function is lacking functionality.
    \subsection{The official website looks professional}
    This aspect is hard to evaluate objectively, since it is up to each user's opinion. According to us, the website does look professional. It's using standard conversions, doesn't have spelling errors or buggy CSS. It works well on mobile as well.\\ \\
    \textbf{Importance: Somewhat important.}\\
    \textbf{Verdict: Pass.}
    \subsection{The pricing of the software}
    Finding the pricing of Qlik Core is not possible. Going to the licensing page, you can easily try the software for free for a period of time. If you want to obtain an actual license however, you have to contact the sales department of Qlik. They don't give any information of any kind of price range or if it's a monthly or yearly (or any other) payment. Any information you want about pricing, you have to contact the sales department. This clearly goes against my recommendations.\\ \\
    \textbf{Importance: Important}\\
    \textbf{Verdict: Failed}. It's up to Qlik themselves how they choose to sell their product. Not giving any sort of information on the cost however will have a huge negative impact on developer experience for their potential buyers.
    \subsection{The release- and change notes are thorough}
    The release notes are in my opinion well done. Since it is a very new product, the release notes are more about new features rather than changes. These are presented in the form of a blog post with images and gifs, which makes it easy to get an overview of what's new. They warn about changes they are making that may cause issues, and explain how one should fix these things.\\ \\
    \textbf{Importance: Not important.}\\
    \textbf{Verdict: Pass.}
    \subsection{The software has the same features on all different platforms}\label{features}
    Qlik Core does not list on their website what features exist on what platforms. They have, according to one of their team members, tested it on different web browsers and found limitations for their services, but have yet to put them on their website. When it comes to the Qlik Associative Enginge, that is run in support of the third-party software Docker. Docker is supported by Mac, Windows and Ubuntu. It's also supports many different servers, such as Fedora. So the heart of Qlik Core, the engine, acting the same on all different platforms is reliant on Docker acting the same on all different platforms.\\\\
    \textbf{Importance: Not very important}\\
    \textbf{Verdict: Failed}. The recommendation is to list your feature support on your website, which Qlik Core does not.
    \subsection{The software is compatible with different platforms}
    As mentioned in \ref{features}, the server side of Qlik Core (the engine) supports many platforms through Docker. The other services offered in Qlik Core works in all browsers, but may have limitations depending on the browser. \\\\
    \textbf{Importance: Not very important.}\\
    \textbf{Verdict: Pass.} I recommend however to make the information of platform support more clear on the Qlik Core website.
    \subsection{The software is offered in more than one programming language}
    As mentioned before, Qlik Core is constructed from a combination of different services. The Engine is close-source, so the programming languages needed to know is for the supporting libraries that comes with Qlik Core. The majority of the libraries are written in JavaScript, which we saw in section \ref{sec:popLang} was a very popular programming language. However, they've actually gone to the length of offering the library that communicates with the engine, Engima, is both JavaScript and Go. This is above and beyond what I even recommend.\\\\
    \textbf{Importance: Not important.}\\
    \textbf{Verdict: Pass.}
    \subsection{The software is open source}
    Qlik Core is partially open source. The data loading software "Halyard", the discovery service "Mira" and the software communicating with the engine "Enigma" are all open source. The Qlik Core website and all the tutorials are also open-source. The Licensing service and the Qlik Associative Engine are however close-source. Their reasoning behind making the Qlik Associative Engine close-source is to protect intellectual property. It is their main selling point, and also serves as the basis for their other products, and needs to be protected. Their licensing service contains some intellectual property as well and logic that would be vulnerable to open up to the public. Qlik has been in talks however to open up this service as well to the public, but would need some refactoring of the software in that case.\\\\
    \textbf{Importance: Important}\\
    \textbf{Verdict: Pass.} Qlik has tried to make Qlik Core as open-source as possible, which follows the recommendations.
    \subsection{The software uses the programming language I am most comfortable with}
    As talked about in \ref{Qlik}, the libraries in QC mainly uses JavaScript. The recommendation for this aspect is not to offer many langugaes, but rather keep an eye on what languages are popular, and will keep on being popular. And as we saw in section \ref{sec:popLang}, JavaScript is the most popular language on GitHub. \\\\
    \textbf{Importance: Important}\\
    \textbf{Verdict: Pass.}
    \subsection{There exists an active online community around the software}
    Being a fairly new product, there doesn't seem to be any community around Qlik Core yet. Searching for Qlik Core on Stack Overflow yields \textit{zero} results. There are no discussions from users directly in the GitHub repository relating to Qlik Core either. Qlik does have a forum site for its users, \texttt{https://community.qlik.com}. This website has different sub-categories for their products. It does not yet however a sub-category dedicated to Qlik Core. On the Qlik Core website, they link to a support channel on Slack (a chat program used commonly by developers). As of writing this paper, this seems to be the best way to get help as a user.\\
    It may be too early to judge Qlik Core for not having an online community. As we have seen however, Qlik has not yet done much to help create a community either. An easy first step would be to make a part of their official community site dedicated to questions regarding Qlik Core.\\ \\
    \textbf{Importance: Important.}\\
    \textbf{Verdict: Failed.} Although the product is in its early stages and will naturally not have a community yet, because of the aspect's importance and how little has been done to try to create a community, I give this aspect a failed grade.
    \chapter{Conclusion}
    More text here...
    \newpage
    \bibliographystyle{plainnat}
    \makebibliography{references}
    \begin{appendices}
        \chapter{Material for Interview}\label{interviewMaterial}
        Before the interviews were conducted, the interviewees got handed this PDF with a part of a fictive software platform documentation. They had to read and solve three questions, to make sure that they had actually read through it properly.
        \includepdf[pages=-]{QuizAndMaterial.pdf}
    \end{appendices}
\end{document}
