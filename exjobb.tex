%%
%% Author: BZM
%% 01/08/2019
%%

% Preamble
\documentclass[11pt]{article}

% Packages
\usepackage{a4wide}
\usepackage{multicol}
\usepackage{csquotes}

% Document
\begin{document}

\begin{multicols}{2}

        \section{A Case Study in Qlik Core's Developer Experience}

        \subsection{Background and Purpose}

        This paper aims to

        \subsubsection{Software platforms and software ecosystems}

%        Text about openness/closeness of a software platform grows
%        dependence, but risks given advantages to competitors __ __ Qlik Core
%        builds on other platforms (Javacript, Docker, etc). Is it it's own
%        platform?

        \subsubsection{What is User Experience?}

        User Experience (UX) is the collective term of many disciplines merged
        into one that evaluates the overall experience delivered to a user of a
        system, product or service. The term was coined by Donald Norman in the
        1990s[Citation needed] who has a background in the fields of cognitive
        science and usability engineering. It is defined by ISO 9241-210, part
        of "Ergonomics of human system interactions", as *"a person's
        perceptions and responses that result from the use or anticipated use of
        a product, system or service"*. It can therefor be considered a
        subjective quality of a product, system or service.


        \subsubsection{What is Developer Experience?}

        Developer Experience, or DX, is similar to the more well known User
        Experience (UX), but with the user being a software developer. DX is
        defined by Sam Jarman as

\begin{displayquote}
"The experience developers have when they use your product, be it
client libraries, SDKs, frameworks, open source code, tools, API,
technology or service." <cite>[Sam Jarman][1]</cite>.
\end{displayquote}

        \subsubsection{Standardizations}
        International Organisation for Standardization (ISO) has yet to present a
        standard for Developer Experience. There are however other standards from ISO
        that are interesting to have a look at. One is \textit{ISO 9126 Software engineering - Product Quality}.
        One part of the ISO-standard concerns quality. This part of the ISO standardizes how to measure the quality of software.
        It has six different characteristics: functionality, reliability, usability, efficiency, maintainability and
        portability. Each of these characteristics have sub-characteristics. The definition of each
        characteristic is listed, and some their sub-characteristics, is in the table below:


    \begin{table}[]
        \centering
        \begin{tabular}{|p{0.05\linewidth}|p{0.2\linewidth}|p{0.55\linewidth}|}
            \hline

            \multicolumn{3}{c}{\textbf{Functionality}} \\ \hline
            F1  &   Suitability &The capability of the software product to provide an appropriate set of
            functions to specified tasks and objectives \\ \hline
            F2  &   Accurateness &  The cap... / / ... to provide the right or agreed results or effects with the needed
            degree of precision \\ \hline
            F3  &   Interoperability& The cap... / / ... to interact with one or more specified systems \\ \hline
            F4  &   Security& The cap... / / ... to protect information and data so that unauthorised persons or systems
            cannot read or modify them and authorised persons or systems are not denied access to them \\ \hline

            \multicolumn{3}{c}{\textbf{Reliability}} \\ \hline
            R1  & Maturity    & The cap... / / ... to avoid failure as a result of faults in the software \\
            \hline
            R2  &   Fault tolerance & The cap... / / ... to maintain a specified level of performance in cases of the
            software faults or of infringement of its specified interface \\ \hline
            R3  &   Recoverability  & The cap... / / ... to re-establish a specified level of performance and recover the
            data directly affected in the case of a failure \\ \hline

            \multicolumn{3}{c}{\textbf{Usability}} \\ \hline
            U1  &   Understandability   &   The cap... / / ... to enable the user to understand whether the
            software is suitable, and how it can be used for particular tasks and conditions of use \\ \hline
            U2&Learnability & The cap... / / ... to enable the user to learn it application \\ \hline
            U3&Operability & The cap... / / ... to operate and control it \\ \hline
            U4&Attractiveness& The cap... / / ... to be attractive to the user [visually] \\ \hline

            \multicolumn{3}{c}{\textbf{Efficiency}} \\ \hline
            E1&Time Behaviour & The cap... / / ... to provide appropriate response and processing times and
            throughput rates when performing its function, under stated conditions \\ \hline
            E2&Resource Utilisation & The cap... / / ... to use appropriate amounts and types of resources when the software
            performs its function under stated conditions \\ \hline
        \end{tabular}
        \end{table}

    \begin{table}[]
        \centering
        \begin{tabular}{|p{0.05\linewidth}|p{0.2\linewidth}|p{0.55\linewidth}|}
            \hline
            \multicolumn{3}{c}{\textbf{Maintainability}} \\ \hline
            M1&Analysability & The cap... / / ... to be diagnosed for deficiencies or causes of failures in the
            software, or for the parts to be modified to be identified \\ \hline
            M2&Changeability & The cap... / / ... to enable a specified modification to be implemented \\ \hline
            M3&Stability & The cap... / / ... to avoid unexpected effects from modifications of the software \\ \hline
            M4&Testability &The cap... / / ... to enable modified software to be validated \\ \hline
            
            \multicolumn{3}{c}{\textbf{Portability}} \\ \hline
            P1& Adaptability &The cap... / / ... to be adapted for different specified enviroments without applying
            actions or means other than tose provided for this purpose for the software considered \\ \hline
            P2&Installability& The cap... / / ... to be installed in a specified environment \\ \hline
            P3&Co-existence &The cap... / / ... to co-exist with other independent software in a common enviroment sharing
            common resources \\ \hline
            P4&Replaceability &The cap... / / ... to be used in a place of another specified software product for the same
            purpose in the same environment \\ \hline

            \multicolumn{3}{c}{\textbf{All characteristics}} \\ \hline
            AC&Compliance & The cap... / / ... to adhere to standards and conventions relating to the
            characteristic \\ \hline
        \end{tabular}
    \end{table}





        There is also a standard for UX,
        namely *ISO-9241-210: Ergonomics of human-system interaction -
        Part 210: Human-centred design for interactive systems*.

        \subsubsection{How do we define 'Good DX'?}

        There are many potential factors for defining what constitutes 'Good'
        DX. <cite>[EveryDeveloper][2]</cite> has developed a \textit{DX Index} from
        1-10, where they consider four factors:

        \begin{itemize}
        \item[-] Are the libraries available in popular languages?
        \item[-] How prominent, in-depth are the starting guides?
        \item[-] Are the solutions self-serving, without need of demos or 'call us'?
        \item[-] Is the pricing clearly stated?
        \end{itemize}
        <cite>[Sam Jarman][1]</cite> has other factors for he uses to evaluate
        if something gives a good DX. He for example puts emphasis on
        communication between the product provider and the developer. The dialog
        between the product provider and the community needs to be authentic,
        open and honest in order the give a good developer experience, according
        to Jarman. He also states that ...

        <cite>[Graziotin, et. al.][3]</cite> researched what makes a developer
        happy and unhappy, and found several indicators. They found both
        internal- and external factors, where external are the most interesting
        for this project. However, the internal unhappiness factor of 'work
        withdrawal' is worth noticing. Being stuck on a task without any
        progress for too long leads to unhappiness.

        \subsubsection{What is Qlik Core?}

        Qlik Core (QC) is, as described on the official website, "an analytics
        development platform built around Qlik Associative Engine and
        Qlik-authored open source libraries". https://core.qlik.com/ The
        platform consists of several components, with it's central part being
        the engine. The platform also provides 'Mira', a software to generate
        insights about the data. Furthermore, it provides the two javascript
        libraries 'Halyard' and 'Enigma'. Halyard helps the user to easily load
        in data into the engine. Enigma helps the user communicate with the
        engine.

        \subsubsection{Kinds of APIs}

        Application Program Interfaces (APIs) are, simply put, a software that
        lets one application interact with another applications inner data and
        services. It's the link between the two pieces of software that let's
        them communicate. Because of APIs broad nature, there are many types of
        APIs. Applications are in need of an interface to interact with it's
        inner parts to create, read, update and read (CRUD) as well as execute
        commands.

        \subsubsubsection{Internal, Public and Partner APIs}

        Internal APIs are APIs that har meant to be used in production and
        within an organisation or company. They are often developed to be used
        between different teams in the company to be able to connect software
        components in the application, without having to actually know the code
        of component. The benefit of this is that the team can open up certain
        needed functionality of the software to other teams while still being in
        control of their own code. This kind of APIs are protected and require
        internal API keys to access to ensure that people outside of the company
        are not able to access them.

        Public APIs is another kind of API. This is a way for the company to
        open up functionality of the software's inner workings to the world so
        that anyone may build new applications that are built upon the original
        software. This kind of interface often only has a small percentage of
        the functionality that the internal API has, since the circuitry of the
        software must be protected for security reasons as well as business
        intelligence theft. If the internal API was open to the public anyone
        could build their own copy of the program. This kind of APIs either do
        not require any API key to access, or have an API key that is open for
        anyone to acquire (either through payment or for free).

        Partner APIs are a third interface that can be shared
        business-to-business (B2B), with strategic partners to the company.
        Partner APIs often put some restraints on what can be exposed so that
        the inner workings of the software is still protected, but is able to be
        more open than a public API. These kinds of APIs require an API key that
        is often contracted with terms and condition to protect the company's
        business intelligence. <cite>[Levin, Guy.][6]</cite>

        \subsubsubsection{Web APIs}

        When using services over the internet, there are many different protocols that can be used to communicate.
        And just like with many other things in computer science, there are many valid approaches whom all have their
        pros and cons. The world wide web (WWW) is largely built upon the application protocol
        hypertext-transfer-protocol (HTTP) which, amongst other things, provides CRUD (Create, Retrieve, Update, Delete)
        methods to applied on resources. Resources in this context is refers to any _thing_: file, object, document,
        text, etc, that is provided by a web service. There are however many ways of utilizing this protocol to let a
        client access and manipulate server-side data, as well as execute commands. Below we describe two methods.

        \subsubsubsubsection{REST}

        Representational State Transfer (REST) is a software architectural style
        that is used in web services which acts as a communication bridge
        between computer systems and the internet that let's the system interact
        and manipulate the service it's interacting with. REST solves many
        issues that had been present in previous implementations of
        communication between computer systems and the web.

        One of the key factors in REST is that it's stateless. Statelessness in
        this context means that the two communicating parties does not need to
        know anything about each other or have seen previous messages to
        understand future ones. This feature is possible by limiting it to the
        use of resources instead of commands. REST-APIs can therefor not ask the
        server side to execute a specific custom command, but is limited to
        using CRUD methods.

        A REST request consists of an HTTP method, a header containing
        information about the request, the path to the resource and lastly and
        optional message body consisting of data.
        https://www.codecademy.com/articles/what-is-rest
        https://www.smashingmagazine.com/2016/09/understanding-rest-and-rpc-for-http-apis/

        Statelessness makes it possible to separate the client and the server.
        Code changes to the server will not require changes to the client's
        code, and vice versa, as long as the message format between the two are
        kept the same.

        Since REST does not use sessions, but simply responds to any incoming
        requests, it's easy to scale up. It simply requires more bandwidth and
        processing power to be able to handle more requests per second.

        If you want, for example, post a message as a user with the userID 1, it
        could look something like this when using REST:


        POST /users/1/messages HTTP/1.1
        Host: example.com
        Content-Type: application/json
        \{"msg": "Test123"\}


        Here, the resource of '/users/1/messages' is fetched and then the
        new message is created and put there. The server does not work in
        sessions and cannot tell clients that a new message is available. The
        clients has to periodically ask the server if there are any new messages
        to retrieve.

        REST may be appropriate to use when you mostly want to do CRUD-commands
        or manipulate data.

        \subsubsubsubsection{RPC}

        %    https://en.wikipedia.org/wiki/Remote_procedure_call
        %    https://www.smashingmagazine.com/2016/09/understanding-rest-and-rpc-for-http-apis/
        Remote Procedure Call (RPC) is a protocol used to execute commands on remote systems. RPC is, like REST, also
        built on HTTP, but uses mostly just the GET and POST commands. RPC is a request-response protocol and is, unlike
        REST, stateful. Ergo, the protocol works with sessions between a client and a server, and previous messages may
        be needed in order to understand future ones.

        An upside of RPC is that it let's a client request the server to execute
        a custom command. Making an RPC-call is much like making a normal
        function call, in that you simply provide the name of the method and the
        parameters. A consequence of this is that code changes on server-side,
        such as method-name or parameter input, may require code changes on the
        client side as well.

        Since RPC has two-way communication, the server can tell the client when
        something has changed, whereas in REST-based communication the client
        has to ping the server to check if there are any changes. An RPC based
        server needs to have a unique session for each client, which can cause
        problems with scalability.

        If we go back to example used in the previous section: posting a message
        may look something like this:

        ```
        POST /SendMessage HTTP/1.1
        Host: example.com
        Content-Type: application/json
        \{"userId": 1, "msg": "Test123"\}
        ```

    Here, the server has a custom method called ```SendMessage```. If the
    method call is not made asynchronous, the client is put in wait until
    the server responds with an acknowledgement or the call reaches a
    timeout. Since RPC uses sessions and custom commands, the method can be
    implemented as such that other sessions should be notified that a new
    message has been sent, and the server can send it out to appropriate
    clients.

    RPC may be appropriate to use when you have functionality that can
    benefit from two-way communication or is mainly command-oriented.

    \subsubsubsection{What type of API is Qlik Core?}

    Qlik Core consists of several components which utilizes different types
    of APIs. Since Qlik Associative Engine is mostly action based, it uses
    JSON-RPC: a remote procedure call protocol in JSON format. Qlik Core
    also provides a discovery service called Mira, which let's the user make
    insights about their data. Mira is a REST-API, since this is a about
    retrieving data and not performing actions.

    \subsubsection{API User Personas}

    There are many types of people using platforms, whom all have different
    requirements. They can be roughly divided into two important groups:
    'Decision makers' and 'Users'. These both need to be catered to in order
    to have a successful platform: if the decision makers are ignored the
    platform will not be implemented by companies in the first place. If the
    users are ignored, the platform will be quickly dropped since it's usage
    is not good enough.

    <cite>[Mark Nottingham][5]</cite> lists 11 personas for HTTP-based APIs.

    \subsection{Execution}


    \subsubsection{Deciding consideration aspects}

    There are a lot of aspects we could have considered as requirements for
    good DX. We had to limit them down however, and ended up with 14 aspects.
    For the second survey, we added two more aspects, namely aspect 13 and 14 in the list below.
    The reasoning for adding this is
    discussed later in the paper. The fourteen original aspects were decided in a combination
    of reading literature, our own experience of what we would consider when
    picking software platforms, and a brainstorm meeting with more experienced
    people at Qlik, consisting of architects and developers.
    The list we ended up with is the following:

    1. How often the software is updated
    2. I can have working code quickly
    3. The API documentation gives thorough explanations on how it works
    4. The API has code examples
    5. The documentation doesn't assume any prior expertise
    6. The documentation has consistent language
    7. The documentation is easy to navigate
    8. The official website looks professional
    9. The pricing of the software
    10. The release- and change notes are thorough
    11. The software has the same features on all different platforms
    12. The software is compatible with different platforms
    13. The software is offered in more than one programming language
    14. The software is open source
    15. The software uses the programming language I am most comfortable with
    16. There exists an active online community around the software

    <cite>[Sam Jarman][1]</cite>'s article is the source for some of these
    aspects. One of the points he makes is the importance of a great documentation.
    He says the documentation should always be written as if the developer is
    a beginner, which lead to aspect `5 - The documentation doesn't assume any prior expertise` in the list.
    He also says great documentation is consistent, ergo it does not use different
    words to mean the same thing, which lead to aspect `6 - The documentation has consistent language`.
    A third aspect he says is needed for great documentation is that is has a
    logical structure, which lead to aspect `7 - The documentation is easy to navigate`.
    The last part he considers important for a great documentation is verbosity.
    As he puts it, "You can never say too much". This resulted in `3 - The API documentation gives thorough explanations on how it works`.
    Jarman also think it's important to have good release notes. He goes on
    to present what release notes should consist of. According to him, not
    only should the release notes consist of the expected, such as what's new,
    updated, deprecated, fixed, etc, but also point out possible risks of the new release,
    such as things that might break with it. Although these sub-features of release notes
    might be interesting to list as their own aspects, we had to keep the list short
    and ended up with the general aspect of `10 - The release- and change notes are thorough`.

    Jarman also talks about pricing. He puts emphasis on that pricing of the software
    should be easy to find for the developer. During the brainstorming, this
    aspect was also discussed, and we ended up with the somewhat vague
    aspect of `9 - The pricing of the software`. This was deliberately chosen
    to be a somewhat open-ended aspect, since there's a lot of things that you can
    consider around the pricing, and we simply wanted to know if pricing is something
    that is often considered in general. In hindsight, it might have been better to have
    divided this into several aspects, since it's difficult to know how the
    survey taker interpreted the aspect.

    When we had this short list, we sat down and thought of things that we
    consider ourself when picking software. The list was extended further,
    with aspects related to API examples, online community and  platform compatibility.
    We then had a brainstorming meeting, where did not present our list, and people
    were free to present things they usually considered. After the people present
    at the meeting had presented their ideas, we showed our list and compared.
    The aspects on the list were discussed, the phrasing of it and the importance
    of them. Finally we ended up with the list of fourteen aspects.
    Being open source was discussed, but after some hesitance dropped. It
    was pointed out as important by Sam Jarman. And after the first survey, when it was
    also pointed out by survey takers as an aspect that they considered, it was
    added to the list in the second survey.
    Aspect number 15 on the list, which was added for the second survey, was
    added for the second survey as well. The reasoning here being that the aspect
    `13 - The software is offered in more than one programming language` felt
    like it needed a parallel question: `15 - The software uses the programming language I am most comfortable with`,
    to see if the importance of several language was solely based on the fact that
    people wanted their favorite programming language.


    \subsubsection{Linking considerations to ISO-9216-1}

    As said before, there is no standard for DX. ISO-9216-1 is however a standard
    to measure the quality of software, so comparing the aspects to this list can
    be interesting. In the table below, the aspects are linked to characteristics they relate to.

    |		|	Functionality	|	Reliability	|	Usability 	|	Efficiency	|	Maintainability	|	Portability	|
    |	----------------	|	----------------	|	----------------	|	----------------	|	----------------	|	----------------	|	----------------	|
    |	How often the software is updated	|		|	R1	|		|		|	M3	|	P4	|
    |	I can have working code quickly	|	F1, F2	|		|	U1, U2, U3	|		|	M2	|		|
    |	The API documentation gives thorough explanations on how it works	|	F1, F2, AC	|	AC	|	U1, U2, U3, AC	|	AC	|	M1, AC	|	AC	|
    |	The API has code examples	|	F1, F2, AC	|	AC	|	U1, U2, U3, AC	|	AC	|	M1, AC	|	AC	|
    |	The documentation doesn't assume any prior expertise	|	AC	|	AC	|	U1, U2, U3, AC	|	AC	|	AC	|	AC	|
    |	The documentation has consistent language	|	AC	|	AC	|	U1, U2, U3, AC	|	AC	|	AC	|	AC	|
    |	The documentation is easy to navigate	|		|		|	U1, U2, U3, AC	|		|		|		|
    |	The official website looks professional	|		|		|	U4	|		|		|		|
    |	The pricing of the software	|		|		|	U1	|		|		|		|
    |	The release- and change notes are thorough	|	F1	|	R1	|	U1	|		|		|	P4	|
    |	The software has the same features on all different platforms	|		|		|		|		|		|	P4	|
    |	The software is compatible with different platforms	|	F3	|		|		|		|		|	P1, P2	|
    |	The software is offered in more than one programming language	|		|		|	U2	|		|		|		|
    |	The software is open source	|	AC	|	AC	|	AC	|	AC	|	M1, M2, M4, AC	|	AC	|
    |	The software uses the programming language I am most comfortable with	|	F1, F2	|		|	U1, U2, U3	|		|		|		|
    |	There exists an active online community around the software	|		|		|	U1, U2	|		|		|		|


    \subsubsection{Initial Survey}

    After gathering potential factors, through litterateur and
    brainstorming, on what would lead to good DX, it was concluded that a
    smaller, initial survey would be relevant to conduct to test the waters
    on the potential factors that had been found. It also intended to find
    more potential factors that had not been thought of. The small survey
    got 38 responses, mostly from Qlik employees. The survey confirmed some
    initial assumtions but also raised some questions. It also highlighted
    that some questions may need to be added, some rephrased and some
    removed for the main survey. The survey questions can be seen in
    Appendix X.

    \subsubsubsection{Survey structure}

    The survey consisted of three parts. The first part was a small screener
    to gather some information about the person taking the survey. The
    second part was about users usage of new software and the third and last
    part was about the DX factors, and how important they were to the user.
    After the survey was done, the data was gathered and evaluated using
    Qlik Sense. Although the data set was too small to make any direct
    affirmations, it gave some indications. In the part about the DX
    factors, the user were to rank each factor depending on how often they
    considered the factor when finding new software. To able to rank the
    answers, each answer was given a value: 'Always Consider' (+4), 'Often
    Consider' (+3), 'Sometimes Consider' (+2), 'Rarely Consider' (+1),
    'Never Consider' (0). After summing them up, I was able to give each
    question a score between 0 and 1, where a 1.0 score would mean that
    everyone answered 'Always consider' and 0 would mean that everyone
    answered 'Never consider'. The result of the survey can be seen in
    Appendix Y.

    \subsubsubsection{Indications of survey results}

    The results were evaluated through initially three different filters,
    which were compared to the overall group, to see if there were any
    trends amongst certain groups. The groups that were filtered by were:
    'People with more than 5 years experience in the industry', 'People who
    are developers within the industry' and 'People in companies with more
    than 200 employees'. The last filter with the larger companies was
    scrapped since the majority of all answers were from Qlik, and the group
    was therefor a too big majority of the whole group. When evaluating the
    data we also found that the people who had the jobtitle of architect
    stood out quite a lot compared to the overall group. Even though the
    sample size was small, only 5 out of the 38, this group could be
    interesting to look at since they are a key group for this survey.

    It should be noted that there is quite a lot of overlap between the
    experienced group and the developer group. 13 people are developers with
    more than five years experience in the industry.

    ##### Software factors

    Overall, all but 3 factors scored a positive score, meaning that they
    are all factors are more often considered than not. The most important
    factor was 'The API has code examples', followed by the importance of an
    active community around the software, and thirdly that the API
    explanations were thorough. The least two important factors were 'The
    documentation has consistent language' and 'The release- and change
    notes are thorough'. This is an interesting find, since literature
    highlights these to factors as being very important. Many companies also
    spend a lot of resources to keep documentation up-to-date and release
    notes thorough. Here we see that most people, more often than not, do
    _not_ consider these factors. A potential explanation to this could be
    that the question is phrased so that it puts the emphasis on _try_ a new
    software, whereas these two factors may be only relevant in long term
    use of a software.

    For the more experienced users, the result differ some to the control
    group. They take less consideration to factors surrounding APIs,
    documentation and time to get started, and care more about pricing,
    cross-platform compatibility and the thoroughness of release notes.

    With the group consisting of mainly developers, we see almost the
    inversion of some trends in the more-experience-group. The developer
    group cares about documentation and time-to-get-started, and do not care
    about release notes and cross-platform compatibility.

    ![Scores](ScoresByPoints.png)

    ##### Creator behind the software factors

    Overall, the creator behind the software seem to be not non-important,
    but not very important, with all factors scoring close to 0/10. The most
    commonly considered factor was that the creator seemed professional, but
    that only ranked 3.0/10. The least consided factor was 'I have heard of
    the creator of the software before', which ranked -2.47/10.

    With the more experienced users, they considered all factors even more
    rarely than the controll group, apart form transparency, which rose
    slightly. Still, the scores are close to 0, and the creator once again
    seem to be a somewhat important factor.

    With the group consisting of only developers, it's much the same as
    before, with the one exception that 'I have heard of the creator of the
    software before' had an increase by 10%. It is however still the least
    considered factor.

    ##### Follow-up interview

    After the results had been evaluated, two follow-up interviews were
    conducted in order to get some insights if there were any issues with
    the survey. The two interviews resulted in some contradicting answers,
    with some things being an issue for interview subject one, and a
    non-issue for subject two.

    The first thing we wanted to see if the people conducting the survey
    read the description of our definition of 'Tools and frameworks'. The
    first subject had not seen this description, whereas the other one did.
    It's however paramount that all subjects use the same definition when
    answering, and in the next survey the definition will be clearly stated
    and made sure it's understood by the test subject.

    The general question if there were any confusion regarding the survey
    questions or alternatives, there did not seem to be much of an issue.
    The first interview subject found one alternative a bit confusing, and
    the second interview subject found no issues. The confusing alternative
    will be made more clear for the next survey.

    Both of the interview subjects were in a position to make decisions for
    a team of what software to use. We asked them if they put themselves in
    the context of making a group decision or a decision for self-use of
    software when they took the survey. We also asked them if they put
    themselves in the context of working professionally or on a 'hobby
    project'. They both took the survey in the context of making decisions
    for a group professionally. The first interview subject had the opinion
    that his survey answers would not differ depending on those contexts,
    whereas the second interview subject pointed out some differences he had
    depending on those two contexts. For the next survey we will explicitly
    tell the people taking the survey to put themselves in a certain
    subject, since it may affect the answers.

    One of the questions were “Which of these traits or aspects do you
    usually consider when deciding if you want to TRY a new tool or
    framework?”, with an emphasis on the word _try_. We asked them is this
    emphasis affected their answer, of if their answers would have been the
    same if the question was stated in the form of: “Which of these traits
    or aspects do you usually consider when deciding to _use_ a new tool or
    framework?”. The first interview subject said he did not see a
    difference between the two questions, and it did not affect his answers.
    The second interview subject went on a side-tangent, talking more about
    the previous question about working professionally versus at home, and
    did not clearly answer if the emphasis did affect his answer. Our
    conclusion is that we should be cautious about drawing too many
    conclusions from this question, since the emphasis may have caused
    survey-takers to interpret the question differently, but that the
    results are still solid.

    We also had concerns around the question of "How quickly do you usually
    decide if the tool or framework is for you?". The interview subjects
    confirmed our concerns. They said that it depends on how complex the
    project and/or software is. This question will have to be rephrased
    and/or given a more specific context in the next survey.

    The first interview subject also had some survey answers that may be
    contradicting, and we asked him to explain his thinking around these
    answers. He had said that he _always_ considers weather or not he can
    have working code quickly. He did _not_ however say that it is a deal
    breaker for a software if he cannot have working code quickly. His
    reasoning around this was that sometimes you have no choice. A software
    may be the only available solution for your project, and then it does
    not matter if it takes a long time to have working code. He also checked
    that he often considers if documentation assumes any prior expertize,
    but it was not a deal breaker. He applied that same reasoning behind
    these choices.

    Our takeaway from this last part is that we may have to rephrase the
    question so that we exclude the extreme cases when developers have no
    choice but to settle for bad DX. We want to find what developers are
    looking for in a software, to find out what they consider to be good DX,
    not what they have to tolerate in bad DX.

    \subsubsubsection{Conclusions from survey}

    The two factors 'The documentation has consistent language' and 'The
    release- and change notes are thorough' scored surprisingly low. This
    goes against the literature. A potential difference could be that
    question only states what would make the user *try* a new software. It
    may not be an initial stopper, but could cause issues once the user has
    decided to use the software. The follow-up interview indicated that so
    may be the case. This will have to be investigated more.

    The non-developer group cares about other things than the developers.
    This groups includes consultants, directors, managers and architects.
    These are positions in which they have power to make major decisions in
    the company. The group therefor have other priorities than the lone
    developer, that only considers his own workflow. In the next survey we
    will make the context more specific so that the two groups are not mixed
    up.

    \subsection{Survey 2}

    \subsubsection{Changes from survey 1}

    It was clear that most people had not heard of, or were not sure about,
    what developer experience was. I will therefor in the next survey
    explain more what DX is about. The follow-up interview made it clear
    that people may have not read the information text in the first survey,
    so I will therefor in the next survey confirm that they have read it. It
    was also not clear in what mindset people were when they answered survey
    1, professional or non-professional. It was also unclear if they were
    taking into consideration that others may use the software. The next
    survey will therefor give a more precise context, so that we can be sure
    that the answers are consistent.

    Overall, the width of the project have to be scoped down, and some
    things will have to be dropped from exploration in this project. The
    first survey found that the creator behind a software was less
    considered than expected. The questions about how they find new software
    and how long time they spend on this will also be dropped from
    exploration.

    In the first survey, we also used the term 'Tools and frameworks'. This
    ended up being too broad of a concept. In the second survey, we solely
    focused on 'Software platform' instead, since Qlik Core is a software platform.

    The focus will be scoped down to solely focus on the software
    considerations required to give a good DX. Two new consideration
    questions were added as well, that did not exist in the first survey.
    The two new questions are 'The software is open source', after getting
    the suggestion from free form answers in the first survey. The second
    new question is 'The software is offered in more than one programming
    language'. The reasoning behind the question was to try to get more
    insight into the question 'The software uses the programming language I
    am most comfortable with', to see if offering more languages is
    something that is worth while for a software platform. The question 'The
    API gives thorough explanations' was also rewritten as 'The API
    documentation gives thorough explanations on how it works' to be less
    vague.

    \subsubsection{Survey 2 Structure}

    The second survey can be divided into two major parts. The first part,
    like in survey 1, focuses on what people consider when choosing a
    software platform. This part was divided into three categories, which
    from now on will be called 'Group', 'Single' and 'Hobby'. Group: When
    working professionally and choosing a software platform for a group of
    people, Single: When professionally choosing a software platform solely
    for yourself, and Hobby: When working non-professionally on a hobby
    project. The questions were the same in these three prats, the only
    different was the context given.

    The second part of the survey focused on developer experience. It had
    two sub-parts, how likely a factor is to cause them to leave an
    interaction with a software platform with a positive feeling, and how
    likely a factor is to cause them to leave an interaction with a software
    platform with a negative feeling. The question were closely linked to
    the questions asked in the consideration part, but focused on the
    _feeling_ rather than if they usually consider the factor when choosing
    a software platform.

    \subsubsection{Survey 2 results}

    We got 39 responses in total. The results were loaded into Qlik Sense
    where we could find different groups and patterns. We looked at the
    results from three angels: The overall average result, result depending
    on your job title and result depending on how much experience you have
    in the industry.

    \subsubsubsection{Overall Result}

    In general, people consider things more often when they are choosing for
    a group than for themselves. For the three categories, Single is the
    closest to the average result, with Group and Hobby existing as
    opposites on either side of Single. In all but one case, if it's often
    considered when choosing for a group, it's _not_ important when choosing
    for a hobby project, and vice versa.

    Overall it can be said that there is a direct correlation between if an
    aspect in it's good form has a positive impact, that same aspect in it's
    bad form will have about the same level of negative impact. However, in
    all but once case, the positive effect is greater than the negative
    effect, if only slightly. In general, it's closely related to how often
    something is considered. If something has a strong DX-impact, it is
    something that is often considered, and vice versa. There are some
    outliers to this, but in general it seems that people actually consider
    things that will cause them to have a good DX.

    The top three most important aspects for each category can be seen in
    the table below.

    |                              Average                              |                               Group                               |                              Single                               |              Hobby              |
    |:------------------------------------------------------------------|:------------------------------------------------------------------|:------------------------------------------------------------------|:--------------------------------|
    | The API has code examples                                         | The API has code examples                                         | The API has code examples                                         | The pricing of the software     |
    | The API documentation gives thorough explanations on how it works | The API documentation gives thorough explanations on how it works | The API documentation gives thorough explanations on how it works | The API has code examples       |
    | I can have working code quickly                                   | The software is compatible with different platforms               | I can have working code quickly                                   | I can have working code quickly |


    The overall result concluded that _the_ most considered aspect overall
    is 'The API has code examples'. For hobby, it's the second most
    important aspect, losing the first place by only 0.03 points. For
    everyone else, it's the most important aspect. Further, good code
    examples had the biggest positive impact on developer experience, and
    bad code examples had the biggest negative impact on DX. The negative
    impact if the examples are bad are not as extreme as the positive impact
    is if they are good however. In conclusion, API examples are a key
    factor for software platforms' quality.

    The thoroughness of the documentation is also one of the most important
    aspects, coming in as the second most important aspect for all
    categories except for Hobby. The positive DX-impact is as big as the
    negative one, and it's also reflected in that it's consideration points
    are as high as the DX-impact points.

    Having working code quickly is also in the top three for all but the
    group category, where it's in fourth place. The positive DX-impact if
    you can have working code quickly is slightly higher than the negative
    DX-impact if it takes a long time before you have working code. The
    positive impact is also stronger than if you compare it to how often
    it's considered.

    Having the software be compatible with different platforms is a big
    divider. It's the third most important aspect for Group, but one the
    least important aspects for Hobby, and somewhere in the middle for
    Single. The positive impact is lower than how often it's considered, and
    the negative DX-impact is even less. If you exclude the Group-category,
    it places itself on average as the 11<sup>th</sup> most important
    aspect, out of 16.

    The pricing of the software is important to everyone, placing itself on
    average as the 4<sup>th</sup> most important aspect overall, but _the_
    most important aspect for Hobby. For group and single, it's both the
    5<sup>th</sup> and 4<sup>th</sup> most important aspect respectively.
    It's a little bit harder to make statements about the DX-impact since
    it's phrased a little differently. The question for consideration is
    simply how often they consider 'The pricing of the software'. In the
    DX-questions however, it's phrased as 'The pricing of the software was
    easy to find' and 'The pricing of the software has hard to find'. The
    positive and negative impact of those two are however less than how
    often it is considered.

    \subsubsubsection{Job Title}

    The job titles were divided into four groups: Architects, Developer and
    Engineers, Managers and Other. The quantity and percentage of total can
    be seen in the table below.

    |   Grouped Job Titles    | # of answers | % of total |
    |:------------------------|:-------------|:-----------|
    | Architects              | 10           | 25.6%      |
    | Developer and Engineers | 21           | 53.8%      |
    | Managers                | 4            | 10.3%      |
    | Other                   | 4            | 10.3%      |

    Below we can see all the aspects, what points each job title gave it and
    it's rank.

    |                               Question                                | Everyone Rank | Avg Everyone | Arch Rank | Architects | Dev&Eng Rank | Devs & Eng | Man Rank | Managers | Other Rank | Other |
    |:----------------------------------------------------------------------|--------------:|:-------------|----------:|:-----------|-------------:|:-----------|---------:|:---------|-----------:|:------|
    | The API has code examples                                             |             1 | 0.90         |         1 | 0.92       |            1 | 0.91       |        1 | 0.79     |        1-2 | 0.94  |
    | The API documentation gives thorough explanations on how it works     |             2 | 0.82         |         3 | 0.82       |            2 | 0.87       |        3 | 0.73     |          4 | 0.74  |
    | I can have working code quickly                                       |             3 | 0.80         |         2 | 0.85       |            4 | 0.80       |      4-5 | 0.71     |        1-2 | 0.94  |
    | The pricing of the software                                           |             4 | 0.80         |         5 | 0.76       |            3 | 0.84       |        2 | 0.75     |          3 | 0.85  |
    | The software uses the programming language I am most comfortable with |             5 | 0.68         |         6 | 0.69       |            5 | 0.73       |      6-7 | 0.63     |          5 | 0.69  |
    | The software is open source                                           |             6 | 0.68         |         4 | 0.77       |            6 | 0.69       |        8 | 0.60     |          8 | 0.58  |
    | There exists an active online community around the software           |             7 | 0.65         |         8 | 0.63       |            7 | 0.67       |     9-11 | 0.54     |          6 | 0.65  |
    | The documentation is easy to navigate                                 |             8 | 0.60         |         7 | 0.66       |            8 | 0.61       |       12 | 0.52     |         10 | 0.52  |
    | The official website looks professional                               |             9 | 0.60         |         9 | 0.63       |           10 | 0.56       |      4-5 | 0.71     |         12 | 0.44  |
    | The software is compatible with different platforms                   |            10 | 0.58         |        10 | 0.58       |           11 | 0.52       |     9-11 | 0.54     |          7 | 0.59  |
    | How often the software is updated                                     |            11 | 0.58         |        11 | 0.52       |            9 | 0.57       |      6-7 | 0.63     |         11 | 0.51  |
    | The documentation doesn't assume any prior expertise                  |            12 | 0.47         |        13 | 0.47       |           13 | 0.44       |     9-11 | 0.54     |          9 | 0.52  |
    | The software has the same features on all different platforms         |            13 | 0.46         |        14 | 0.41       |           12 | 0.45       |    13-15 | 0.44     |         13 | 0.42  |
    | The documentation has consistent language                             |            14 | 0.41         |        12 | 0.49       |           14 | 0.34       |       16 | 0.42     |         14 | 0.39  |
    | The software is offered in more than one programming language         |            15 | 0.36         |        15 | 0.37       |           15 | 0.31       |    13-15 | 0.44     |         16 | 0.23  |
    | The release- and change notes are thorough                            |            16 | 0.33         |        16 | 0.26       |           16 | 0.29       |    13-15 | 0.44     |         15 | 0.35  |

    ![consid_jobtitle2](./consid_jobtitle.png "Considerations grouped by different job titles")

    There are several angles you can view this at. Here we discuss a few.

    \subsubsubsubsection{Developer and Engineers compared with Architects}

    For example, if we compare Architects and Developer and Engineers, they
    are quite similar. On average, the score differences is 0.06. The
    biggest difference for consideration questions is 'The documentation has
    consistent language', where Architects average out at 0.49/1.00 in
    points, making it 'Sometimes considered', whereas Developer and
    Engineers only score 0.34/1.00, placing it between 'Sometimes consider'
    and 'Rarely consider'. Their ranking does not differ much, with the
    biggest ranking difference being two spots.

    Below we can see the different aspects, sorted by average most
    important, for Architects and Developers and Engineers.

    |                                Aspect                                 | Arch Rank | Architect Points | Devs & Eng Rank | Devs & Eng Points | Diff |
    |:----------------------------------------------------------------------|----------:|:-----------------|----------------:|:------------------|-----:|
    | The API has code examples                                             |         1 | 0.92             |               1 | 0.91              | 0.01 |
    | The API documentation gives thorough explanations on how it works     |         3 | 0.82             |               2 | 0.87              | 0.05 |
    | The pricing of the software                                           |         5 | 0.76             |               3 | 0.84              | 0.08 |
    | I can have working code quickly                                       |         2 | 0.85             |               4 | 0.80              | 0.05 |
    | The software is open source                                           |         4 | 0.77             |               6 | 0.69              | 0.08 |
    | The software uses the programming language I am most comfortable with |         6 | 0.69             |               5 | 0.73              | 0.05 |
    | There exists an active online community around the software           |         8 | 0.63             |               7 | 0.67              | 0.03 |
    | The documentation is easy to navigate                                 |         7 | 0.66             |               8 | 0.61              | 0.05 |
    | The official website looks professional                               |         9 | 0.63             |              10 | 0.56              | 0.07 |
    | The software is compatible with different platforms                   |        10 | 0.58             |              11 | 0.52              | 0.06 |
    | How often the software is updated                                     |        11 | 0.52             |               9 | 0.57              | 0.05 |
    | The software has the same features on all different platforms         |        14 | 0.41             |              12 | 0.45              | 0.04 |
    | The documentation doesn't assume any prior expertise                  |        13 | 0.47             |              13 | 0.44              | 0.03 |
    | The documentation has consistent language                             |        12 | 0.49             |              14 | 0.34              | 0.15 |
    | The software is offered in more than one programming language         |        15 | 0.37             |              15 | 0.31              | 0.06 |
    | The release- and change notes are thorough                            |        16 | 0.26             |              16 | 0.29              | 0.04 |


    \subsubsubsubsection{Architects compared with }

    If we compare Architects and Managers, the differences are bit more
    extreme. On average, they differ by 0.09 points. The biggest divider for
    them is 'The release- and change notes are thorough', where they differ
    by 0.18 points. The rank difference is only one spot though. Their
    biggest dividers in rank differ by five spots. They are 'The official
    website looks professional' and 'The documentation is easy to navigate'
    where the first one is more important to Architects, and the second one
    is more important to Managers.

    |                                Aspect                                 | Arch Rank | Architects | Manager Rank | Manager Points | Diff |
    |:----------------------------------------------------------------------|----------:|:-----------|-------------:|:---------------|-----:|
    | The API has code examples                                             |         1 | 0.92       |            1 | 0.79           | 0.13 |
    | The API documentation gives thorough explanations on how it works     |         2 | 0.82       |          4-5 | 0.71           | 0.14 |
    | The pricing of the software                                           |         3 | 0.76       |            3 | 0.73           | 0.09 |
    | I can have working code quickly                                       |         5 | 0.85       |            2 | 0.75           | 0.01 |
    | The software is open source                                           |         4 | 0.77       |            8 | 0.60           | 0.16 |
    | The software uses the programming language I am most comfortable with |         6 | 0.69       |          6-7 | 0.63           | 0.06 |
    | There exists an active online community around the software           |         9 | 0.63       |          4-5 | 0.71           | 0.08 |
    | The documentation is easy to navigate                                 |         8 | 0.66       |         9-11 | 0.54           | 0.09 |
    | The official website looks professional                               |         7 | 0.63       |           12 | 0.52           | 0.14 |
    | The software is compatible with different platforms                   |        10 | 0.58       |         9-11 | 0.54           | 0.04 |
    | How often the software is updated                                     |        11 | 0.52       |          6-7 | 0.63           | 0.11 |
    | The software has the same features on all different platforms         |        13 | 0.41       |         9-11 | 0.54           | 0.07 |
    | The documentation doesn't assume any prior expertise                  |        12 | 0.47       |           16 | 0.42           | 0.07 |
    | The documentation has consistent language                             |        14 | 0.49       |        13-15 | 0.44           | 0.03 |
    | The software is offered in more than one programming language         |        15 | 0.37       |        13-15 | 0.44           | 0.07 |
    | The release- and change notes are thorough                            |        16 | 0.26       |        13-15 | 0.44           | 0.18 |

    \subsubsubsubsection{Developer and Engineers compared with Managers}

    If we compare Developer and Engineer with Managers, we also find that
    they're quite different. On average, they differ in points by 0.10
    points. Their biggest differ in points is for 'The official website
    looks professional', where they differ by 0.15 points, which is also
    their biggest differ in ranking: 6 spots different.

    |                                Aspect                                 | Devs & Eng Rank | Devs & Eng | Manager Rank | Manager Points | Diff |
    |:----------------------------------------------------------------------|----------------:|:-----------|-------------:|:---------------|-----:|
    | The API has code examples                                             |               1 | 0.91       |            1 | 0.79           | 0.13 |
    | The API documentation gives thorough explanations on how it works     |               2 | 0.87       |          4-5 | 0.71           | 0.14 |
    | The pricing of the software                                           |               3 | 0.84       |            3 | 0.73           | 0.09 |
    | I can have working code quickly                                       |               4 | 0.80       |            2 | 0.75           | 0.01 |
    | The software is open source                                           |               5 | 0.73       |            8 | 0.60           | 0.16 |
    | The software uses the programming language I am most comfortable with |               6 | 0.69       |          6-7 | 0.63           | 0.06 |
    | There exists an active online community around the software           |               7 | 0.67       |          4-5 | 0.71           | 0.08 |
    | The documentation is easy to navigate                                 |               9 | 0.57       |         9-11 | 0.54           | 0.09 |
    | The official website looks professional                               |              10 | 0.56       |           12 | 0.52           | 0.14 |
    | The software is compatible with different platforms                   |               8 | 0.61       |         9-11 | 0.54           | 0.04 |
    | How often the software is updated                                     |              11 | 0.52       |          6-7 | 0.63           | 0.11 |
    | The software has the same features on all different platforms         |              12 | 0.45       |         9-11 | 0.54           | 0.07 |
    | The documentation doesn't assume any prior expertise                  |              13 | 0.44       |           16 | 0.42           | 0.07 |
    | The documentation has consistent language                             |              14 | 0.34       |        13-15 | 0.44           | 0.03 |
    | The software is offered in more than one programming language         |              15 | 0.31       |        13-15 | 0.44           | 0.07 |
    | The release- and change notes are thorough                            |              16 | 0.29       |        13-15 | 0.44           | 0.18 |


    \subsubsubsection{Experience}

    The responses were also divided into five groups, depending on how much
    experience in the software industry the had. There were 5 people with
    less than 5 years experience, 10 people with 5 - 10 years experience, 10
    people with 10 - 15 years experience, 12 people with 15 - 25 years
    experience and 2 people with 25+ years of experience in the software
    industry.

    |                         Less than 5 years                         |                           5 - 10 years                            |                           10 - 15 years                           |          15 - 25 years          |                             25+ years                             |
    |:------------------------------------------------------------------|:------------------------------------------------------------------|:------------------------------------------------------------------|:--------------------------------|:------------------------------------------------------------------|
    | The API documentation gives thorough explanations on how it works | The API has code examples                                         | The API has code examples                                         | The API has code examples       | The API has code examples                                         |
    | The API has code examples                                         | The API documentation gives thorough explanations on how it works | The API has code examples                                         | The pricing of the software     | I can have working code quickly                                   |
    | There exists an active online community around the software       | I can have working code quickly                                   | The API documentation gives thorough explanations on how it works | I can have working code quickly | The API documentation gives thorough explanations on how it works |

    \subsubsubsection{Decision Makers

    We divided the group into decision makers and non-decision makers. The
    groups are of comparable size: There are 22 decision makers for groups
    and 14 non-decision makers for groups, and 3 who answered that it is not
    applicable. The following are the top three most important aspects to
    the groups:

    |                          Decision Makers                          |                        Non-Decision Makers                        |
    |:------------------------------------------------------------------|:------------------------------------------------------------------|
    | The API has code examples                                         | The API has code examples                                         |
    | The API documentation gives thorough explanations on how it works | The pricing of the software                                       |
    | I can have working code quickly                                   | The API documentation gives thorough explanations on how it works |

    It could also be interesting to see who the decision makers are. Divided
    by job title and level, it looks like this:

    |        Job Title        | Level  | Decision Makers | Non-decision Makers |
    |:------------------------|:-------|:----------------|:--------------------|
    |                         |        | **22**          | **14**              |
    | Architect               | Middle | 0               | 0                   |
    | Architect               | Senior | 7               | 2                   |
    | Developer and Engineers | Middle | 3               | 4                   |
    | Developer and Engineers | Senior | 5               | 7                   |
    | Managers                | Middle | 1               | 0                   |
    | Managers                | Senior | 3               | 0                   |
    | Other                   | Middle | 0               | 0                   |
    | Other                   | Senior | 3               | 1                   |

    As we can see, and not surprising, all managers are decision makers.
    Somewhat more interesting, two architects claim they are not in a
    position to make decisions on what software others will use. We also see
    that 50% of Middle level people are decision makers, and 64% of Senior
    level people are decision makers.


    \subsubsubsection{Compared to Survey 1}

    The sample size is almost exactly the same in the two surveys. The first
    survey had 38 responses, whereas the second one had 39. There are two
    major differences between the two surveys. The first one had mostly just
    Developer and Engineers, 74%, only one manager, 3%, and just five
    architects, 13%. As a reminder, the second survey had 53.8% developer
    and engineers, 10.3% managers and 25.6% architects. The second
    difference is that in the second survey, the people were given a context
    for when they were considering the aspects.

    If we compare it to the first survey, we see some anomalies. To be able
    to compare the rankings, we will exclude the two newly added questions
    in this part. An interesting difference is the rank of 'The software is
    compatible with different platforms'. However, this aspect is very
    divided in the second survey depending on the situation. The rankings
    for this aspect look as following: Group ranks it as the third most
    important, whereas single ranks is at the 10th spot and finally hobby at
    place number 13. Similarly, the question 'The software has the same
    features on all different platforms' has a big point difference when
    comparing the first and second survey. However, this is also a divided
    question depending on situation in the second survey. It's ranked by
    Group at #9, Single is tied between the 11th and 12th spot and for Hobby
    it's #13. In the first survey, it's ranked as the #9 out of 14. The
    change in ranking in both these cases can therefor somewhat be explained
    by no context was given in the first survey.

    But maybe most interesting is that in the first survey, 'There exists an
    active online community around the software' ranked as the second most
    important aspect, but is on average ranked as the sixth most important
    aspect in the second survey. There is no difference on the context here,
    in the second survey it's ranked at #6 by both Single and Hobby, and is
    tied between 6th and 7th spot for Group. It _is_ slightly more important
    for Developer and Engineers, but only by 0.04 points, so the fact that
    the first survey had them as a majority cannot explain this either.
    There does not seem to be any obvious explanation for this shift.

    Below we can see the average result from the two surveys, with the
    rankings. The two questions that don't exist in the first survey have
    been excluded, and the ranking for the second survey recalculated.


    |                                Aspect                                 | Survey 2 - Everyone Rank | Survey 2 - Everyone	Points | Survey 1 - Everyone Rank | Survey 1 - Everyone Points | Diff Points | Diff Rank |
    |:----------------------------------------------------------------------|-------------------------:|:--------------------------|-------------------------:|:---------------------------|------------:|----------:|
    | The API has code examples                                             |                        1 | 0.90                      |                        1 | 0.87                       |       +0.03 |         0 |
    | The API documentation gives thorough explanations \[on how it works\] |                        2 | 0.82                      |                        3 | 0.78                       |       +0.05 |        +1 |
    | I can have working code quickly                                       |                        3 | 0.80                      |                        4 | 0.76                       |       -0.04 |        +1 |
    | The pricing of the software                                           |                        4 | 0.80                      |                        5 | 0.74                       |       +0.06 |        +1 |
    | The software uses the programming language I am most comfortable with |                        5 | 0.68                      |                        7 | 0.68                       |        0.00 |        +2 |
    | ~~The software is open source~~                                       |                        - | 0.68                      |                        - | -                          |           - |         - |
    | There exists an active online community around the software           |                        6 | 0.65                      |                        2 | 0.80                       |       -0.15 |        -4 |
    | The documentation is easy to navigate                                 |                        7 | 0.60                      |                       10 | 0.61                       |        0.00 |        +3 |
    | The official website looks professional                               |                        8 | 0.60                      |                       11 | 0.59                       |       +0.01 |        +3 |
    | The software is compatible with different platforms                   |                        9 | 0.58                      |                        6 | 0.72                       |       -0.14 |        -3 |
    | How often the software is updated                                     |                       10 | 0.58                      |                        8 | 0.63                       |       -0.05 |        -2 |
    | The documentation doesn't assume any prior expertise                  |                       11 | 0.47                      |                       12 | 0.50                       |       -0.03 |        +1 |
    | The software has the same features on all different platforms         |                       12 | 0.46                      |                        9 | 0.62                       |       -0.16 |        -3 |
    | The documentation has consistent language                             |                       13 | 0.41                      |                       13 | 0.45                       |       -0.04 |         0 |
    | ~~The software is offered in more than one programming language~~     |                        - | 0.36                      |                        - | -                          |           - |           |
    | The release- and change notes are thorough                            |                       14 | 0.33                      |                       14 | 0.41                       |       -0.09 |         0 |


    \subsubsubsection{Not considered}

    Because of the way the data pool was shaped, we did not look at two
    groups, which could have been interesting. The first one is company
    size. 30 out of 39 people worked in companies with more than 1000
    people, and only 4 out of 39 people worked in companies with less than
    100 people. Therefor the data pool is too small to make any larger
    claims on patterns.

    The other group that could have been interesting to look at is the
    professional level of the persons. However, 77% were senior level, and
    23% were middle, and no one was junior level. The job title level is
    somewhat arbitrary, and it can be argued that looking at years of
    experience, where the answers are more divided, can substitute this
    angle. Most of the people with a middle level have worked \< 5 years.
    How big part of the work experience group are made up of middle and
    senior level respectively can be seen in the table below.

    | Years of experience | Middle level | Senior level |
    |:--------------------|:-------------|:-------------|
    | < 5 years           | 100%         | 0%           |
    | 5 - 10 years        | 20%          | 80%          |
    | 10 - 15 years       | 20%          | 80%          |
    | 15 - 25 years       | 0%           | 100%         |
    | 25+ years           | 0%           | 100%         |

    \subsection{Interviews}

    \subsubsection{Reasons to do interviews}
    There are several reasons why interviews is a suitable method use in this reasearch project.
    The quantitative data collected
    by the two survey gives a good idea of what is needed, does not answer *why* it is needed.
    Interviews will therefor give a depth to the quantitative data that has been collected.
    Denscombe[9] recommends using interviews as a follow-up to questionnaires. As he puts it,
    questionnaires can generate some interesting results that interviews can pursue
    in greater detail and depth. He also states that interviews should be seen as a good way
    to corroborate data found with other methods. By using interviews, we can triangulate data
    collected by the questionnaires to confirm the facts once more with another approach.
    Furthermore, Dencsombe talks about interviews being well-suited for certain kinds of data.
    He gives three main data types when interviews are a good method,
    two of which are applicable to this research project. The first reason is
    when data is based on emotions. DX is, as described before, based on the *feeling*
    of a good interaction, something that is hard to get a understanding of through
    surveys. The second reason given by Denscombe is when you have access to 'key players'
    with in a field. That is, when you have the possibility to interview people
    that have great insight into a field, interviews are a good way to collect that data.
    With these interviews, we have access to people of different experience, job titles
    and level of decision power. By doing interviews, we can get an understanding
    of why different job titles desire different things, why more experienced people
    want certain aspects and why persons with a lot of power within a company
    have needs that others don't.

    \subsubsubsection{Interview instead of Observation Study}
    We also talked about doing

    \subsubsection{Interview Decisions}
    \subsubsubsection{Interview Style}
    There are many different ways of conducting interviews. Denscombe groups it into
    three different approaches: one-on-one interviews, group interviews and focus groups.
    These methods all have pros and cons, described by Denscombe[9].

    One-on-one interviews are easy to arrange, since
    only two people's (Interviewer and interviewee) schedules have to coincide.
    Compared to group interviews and focus groups, it's also easy to control the interview,
    dive deeper into questions when needed and it's easy to link the results
    to the specific person.

    Group interviews have advantages as well. It helps to find what a consensus around
    a topic is, where people's opinions can be challenged right away by other group members.
    There are also risks of group interviews, where 'quieter' people may be silenced by
    members of the group who are more dominant. Group interviews are also less suited
    for topics where answers that are more 'accepted' than others.

    Finally, focus groups have the same issues as group interviews does. It also
    requires a more skilled moderator, since focus groups is more of an ongoing
    discussion of experts within a field, that can easily get out of hand if the
    moderator does not know how to steer the group.  Transcribing focus groups are also
    more challenging since it is natural that people talk over each other. Linking opinions
    to certain people may also be harder when everyone's statements are blurred together.

    For this research project we chose to go ahead with one-on-one interviews.
    The disadvantages of them are small, and one the most challenging things
    about the interviews is finding time to arrange them. It was not feasible
    to try to find enough people for a group interview where everyone's schedules
    could coincide.

    \subsubsubsection{Interview Structure}
    Dencsombe describes three different types of interviews: structured, semi-structured and
    unstructured interviews. Structured interviews are much like questionnaires, where
    the interviewer holds a tight grip on the interview and does not expect free form answers.
    Structured interviews looks to standardize the results for easy comparison.
    This interview structure is more for 'checking' rather than 'discovery' and is
    therefor more of a quantitative data collection rather than qualitative.

    Semi-constructed interviews has, just like constructed interviews, a clear set
    list of questions. However, in a semi-structured interview, the interviewee
    is not presented with yes or no questions, or questions with alternatives.
    The questions are instead open-ended. As described by Denscombe, the interviewee
    is expected to develop their own ideas and speak widely about the given issue or question.
    Here, the emphasis is more on 'discovery' than checking.

    Unstructured interviews is the third alternative and is the loosest of the three styles.
    With this structure, the interviewer wants to get the interviewees general thoughts
    on a subject or issue and tries to be as unobtrusive as possible. As described by Denscombe,
    the interviewer presents a specific issue or subject and hopes to get a ball rolling.
    Semi-constructed and unstructured exist on a continuum. The more open-ended the questions
    are, the more you move from semi-structured to to unstructured.

    For this research project we are using the style of semi-constructed.
    We are trying to discover things rather than check, which removes the option
    of a structured interview. We however have a very clear set of questions
    we want answers on.

    \subsubsection{Interview Subjects}
    There are many things that be explored for these interviews. Due to limited
    time, we had to narrow it down to a few subjects. The three subjects
    we chose are related to release notes, APIs and online community.

    Release notes is noted as a very important aspect by literature, but
    the survey results show that this is one of the least considered aspects.
    Whereas this is something that maybe could be expected (people don't look
    at release notes until they are needed, and don't usually consider this
    when choosing platforms), the more surprising part is that it does *not*
    affect them negatively DX-wise either if the release notes are poorly written.
    To get a deeper understanding on why this is, we chose this as a subject
    for the interviews.

    The API documentation and examples are pointed as very important by the
    surveys. In the survey we used the vague term of 'Good' and 'Bad', and
    the survey takers applied their own definition of these two terms. We
    are interested of how we can define 'Good' and 'Bad'. One of the most important
    aspects according the survey is also to have working code quickly. Upon exploring
    possible aspects to explore, we found ourself asking if working code quickly
    is jus ta consequence of good API documentation and examples. Maybe it's not
    its own aspect at all? So this question also got merged into this question subject.

    The last subject we chose to explore is online communities. An anomaly found
    between the first and second survey is that online community was very important
    to people in the first survey, but only somewhat important in the second one.
    We want to understand the shift in this ranking.

    \subsubsection{Interview material}
    We concluded that it would be good to have reference points during the interview,
    that there should be some material that the interviewee could use when talking.
    We discussed using some API documentation and release notes from existing software platforms,
    more precisely Qlik Core. However, Qlik Core is a very advanced platform which takes a long time
    to grasp. Since the interviewees have limited time to put aside for the interviews, it seemed
    like making them read and understand documentation of a whole new software platform was too much to ask.
    Instead we created some API documentation and release notes for a made up software platform
    called 'MyBakery'. Everyone already have a good mental picture of how a bakery works,
    so even if you give them limited amount of documentation they still have a mental picture
    of what this software platform does. To make the request of making them read documentation less boring,
    we tried to make a gamification of the task. The interviewees were given the material, and
    three questions they needed to answer. This forced them to read and understand the material, rather
    than just read through it. The material can be seen in appendix X.

    \subsubsection{Interview Questions}

    Ultimately, we want to answer "Why are the aspects needed or not needed
    to give a good DX and what happens if they (don’t) exist or are poorly/well implemented?"
    The questions for the interviews are constructed to be able to answer this question.
    Since the interview is semi-constructed, all questions will not necessarily be asked.
    Depending on how the interviewee answer, some questions may already have been answered,
    some questions may be uninteresting to dive into, etc.
    \subsubsubsection{A - APIs}
    For the API documentation, API examples and to have working code quickly, we
    already know that is important. It's also not very surprising that it is.
    For this, we are rather trying to understand how we define 'Good' API related

    **AA - General**

    * AA1 - How important would you say API documentation and examples are?

    **AB - Api Documentation**

    *	AB1 - When you look at API documentation, what are you usually looking for?
    * What should the documentation look like?
    * Is there anything in the material you always look for, or something that is missing?
    * When you come across an API documentation, are there any red flags you look for?
    *	AB2 - How does the API documentation quality affect you in your work?
    * Do you abandon a software platform if the documentation is poor?

    **AC - Api Examples**

    *	AC1 - What is your goal when looking at API examples?
    * For copy-pasting?
    * To understand underlying structure?
    * To simply see how it's used?
    *   AC2 - What should API examples look like?
    * Short examples or long examples?
    * Should it be runnable or concise?
    * Within a big context or concise?
    *	AC3 - How does the API examples quality affect you in your work?
    * Do you abandon a software if the examples are poor?

    **AD - Working Code Quickly**

    *	AD1 - Is it important for you to have working code quickly?
    *	Why is it/is it not?
    *	AD2 - When can you accept to not have working code quickly?
    *	AD3 - Can you have working code quickly if the API examples and documentation is bad?

    \subsubsubsection{B - Release Notes}
    For release notes, we know through the survey that they are *not* considered.
    The questions have been divided into three groups: 'When are [release notes] needed",
    "What should [release notes] look like?" and "[What are the] consequences [of good/bad release notes]?"


    **BA - When are they used?**

    * BA1 - How often do you look at release notes?
    * BA2 - In what circumstances do you look at release notes?
    * Do you look at release notes before deciding to use a platform?
    * BA3 - What do you look for when looking at release notes?

    **BB - What should they look like?**

    * BB1 - Would you say it is important that software platforms have release notes?
    * Do you find that information in some other way?
    * BB2 - How detailed should they be? / What should they look like?
    * How important is it for you that the release notes are thoroughly written?
    * BB3 - Is it worth a company's time to make thorough release notes?
    * BB4 - How do poor release notes affect you?

    **BC - Survey Anomaly**
    * BC1 - Release notes is ranked as the least, or amongst the least important aspect for all groups. Why do you think that is?



    \subsubsubsection{C - Online Communities}
    For online community, we are trying to understand why it was ranked
    so differently in the two surveys. We also try to understand why, in general,
    an online community matters.

    **CA - General**
    *	CA1 - When talking about software platforms, what is an online community to you?
    *	CA2 - If the documentation was flawless, would you not need an online community?

    **CB - When are they used?**
    *	CB1 - How often would you say you take help from online communities?
    *   CB2 - Do you check out the online community before choosing a platform?

    **CC - What should it look like?**

    *	CC1 - What do you want from an online community / what should it look like?
    * How important are online communities?
    * What do communities that you like have in common?
    * Is it important that a community feels alive?
    * Is it important the community feels helpful?
    * Is it important that the tone used in the community is positive?
    * Is it important that the company behind the software are part of the community?
    *	CC2 - If we compare software platforms to something smaller, such as a library. Would you say it is more important or less important to have a software community around it?
    *	Why is it more/less important?

    **CD - Survey Anomaly**
    * CD1 - In the first survey, having an online community was ranked as the second most important aspect. In the second one, it’s ranked in the middle. Why do you think that is?

    \subsubsection{Interview Results}

    All of the interviews can be found in their full form in appendix X. In this section however
    we will discuss some interesting findings in the interviews.

    \subsubsubsection{API Documentation and Examples}

    Just like the surveys suggested, API documentation and examples are very important
    to everyone. It turns out that API examples are actually the first thing people
    look for when encountering new documentation. All the interview subjects, when
    encounter with the question "AB1 - When you look at API documentation, what are you usually looking for?"
    answered in some shape or form that they first of all look for examples.

    \subsection{Good Sources}

    <cite>[B2D practices and stratergies][4]</cite>. Some key notes are that
    in B2D is that the value of the product must shown before the purchase.
    Let developers use the product! Not demos or trials.

    <cite>[Developer Personas][5]</cite>. Defines different personas of
    developers. Who are the developers that will be using our product?

    <cite>[Best practices in API documentation][8]</cite>

    [1]: https://hackernoon.com/the-best-practices-for-a-great-developer-experience-dx-9036834382b0	"The Best Practices for a Great Developer Experience (DX)"
    [2]: http://everydeveloper.com/developer-experience/	"WHAT IS DEVELOPER EXPERIENCE?"
    [3]: https://www.sciencedirect.com/science/article/pii/S0164121218300323	"What happens when software developers are (un)happy"
    [4]: https://medium.com/@ApurvaBDave/lessons-in-business-to-developer-marketing-55c847300808	"Four Strategies for Business-to-Developer (B2D) Marketing"
    [5]: https://www.mnot.net/blog/2012/04/14/user_personas_for_http_apis	"User Personas for HTTP APIs"
    [6]: https://blog.restcase.com/internal-vs-external-apis/ "Internal vs External APIs"
    [7]: https://www.w3.org/TR/2004/NOTE-ws-arch-20040211/#relwwwrest "W3 REST"
    [8]: https://swagger.io/blog/api-documentation/best-practices-in-api-documentation/
    [9]: # (Denscombe, Martyn. Good Reseach Guide. McGraw-Hill eduction, 2007. p.165)


\end{multicols}
\end{document}